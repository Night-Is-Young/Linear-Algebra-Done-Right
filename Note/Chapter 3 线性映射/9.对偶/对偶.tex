\documentclass{ctexart}
\usepackage{geometry}
\usepackage[dvipsnames,svgnames]{xcolor}
\usepackage[strict]{changepage}
\usepackage{framed}
\usepackage{enumerate}
\usepackage{amsmath,amsthm,amssymb}
\usepackage{enumitem}
\usepackage{template}
\usepackage{nicematrix}

\allowdisplaybreaks
\linespread{1.5}
\geometry{left=2cm, right=2cm, top=2.5cm, bottom=2.5cm}

\begin{document}
\pagestyle{empty}
\begin{center}\large 对偶\end{center}
\tbf{1.对偶空间和线性映射}\\
映射到标量域$\F$的线性映射在线性代数中扮演着特殊的角色,于是,我们赋予它特殊的名称.
\begin{definition}[1.1 定义:线性泛函]
    $V$上的\tbf{线性泛函}是从$V$到$\F$的线性映射.
\end{definition}\noindent
换言之,$V$上的线性泛函$T$是$\mathcal{L}(V,\F)$的元素.同样地,$\mathcal{L}(V,\F)$也有特殊的名称和记号.
\begin{definition}[1.2 定义:对偶空间]
    $V$的对偶空间$V'$是$V$上全体线性泛函构成的向量空间.换言之,$V'=\mathcal{L}(V,\F)$.
\end{definition}\noindent
以及,对偶空间的维数和原空间相同.
\begin{formal}[1.3 对偶空间的维数]
    假设$V$是有限维向量空间,那么$V'$也是有限维的,且满足$\dim V=\dim V'$.
\end{formal}
\begin{solution}[Proof.]
    我们知道$\dim\left(\mathcal{L}(V,W)\right)=\left(\dim V\right)\left(\dim W\right)$.于是
    $$\dim V'=\dim V\cdot\dim\F=\dim V$$
\end{solution}
\begin{definition}[1.4 定义:对偶基]
    若$\li v,m$是$V$的一组基,那么$\li v,m$的对偶基是$V'$中的元素$\li\phi,m$构成的组,其中各$\phi_k$是满足
    $$\phi_k(v_j)=\left\{\begin{array}{l}
        1,j=k\\0,j\neq k
    \end{array}\right.$$
    的线性泛函.
\end{definition}\noindent
上述命题给了我们一种表示线性组合的系数的方法.
\begin{formal}[1.5 对偶基和线性组合的系数]
    假设$\li v,m$是$V$的基,$\li\phi,m$是其对偶基,那么对于任意$v\in V$都有
    $$v=\phi_1(v)v_1+\cdots+\phi_m(v)v_m$$
\end{formal}\noindent
上述命题是容易证明的.现在我们来说明对偶基是对偶空间的基.
\begin{formal}[1.6 对偶基是对偶空间的基]
    假设$V$是有限维的,那么$V$的基的对偶基是$V'$的基.
\end{formal}
\begin{solution}[Proof.]
    设$\li v,m$是$V$的基,$\li\phi,m$是其对偶基.设$\li a,m\in\F$满足
    $$a_1\phi_1+\cdots+a_m\phi_m=\mbf{0}$$
    对于各$k\in\left\{1,\cdots,m\right\}$有$$\left(a_1\phi_1+\cdots+a_m\phi_m\right)(v_k)=a_k$$
    即$\mbf{0}v_k=a_k$,于是$a_k=0$,进而$\li\phi,m$线性无关.\\
    由于$\li\phi,m$是$V'$中长度为$\dim V'$的线性无关组,进而$\li\phi,m$是$V'$的基.
\end{solution}
\begin{definition}[1.7 对偶映射]
    设$T\in\mathcal{L}(V,W)$,$T$的对偶映射是由下式定义的线性映射$T'\in\mathcal{L}(W',V')$:对任意$\phi\in W'$有
    $$T'(\phi)=\phi\circ T$$
\end{definition}
\begin{formal}[1.8 对偶映射的代数性质]
    设$T\in\mathcal{L}(V,W)$.那么
    \begin{enumerate}[label=\tbf{(\arabic*)}]
        \item 对于所有$S\in\mathcal{L}(V,W)$,都有$(S+T)'=S'+T'$.
        \item 对于所有$\lambda\in\F$,都有$(\lambda T)'=\lambda T'$.
        \item 对于所有$S\in\mathcal{L}(W,U)$,都有$(ST)'=T'S'$.
    \end{enumerate}
\end{formal}
\end{document}