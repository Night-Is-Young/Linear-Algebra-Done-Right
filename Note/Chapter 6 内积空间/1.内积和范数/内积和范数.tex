\documentclass{ctexart}
\usepackage{geometry}
\usepackage[dvipsnames,svgnames]{xcolor}
\usepackage[strict]{changepage}
\usepackage{framed}
\usepackage{enumerate}
\usepackage{amsmath,amsthm,amssymb}
\usepackage{enumitem}
\usepackage{template}

\allowdisplaybreaks
\geometry{left=2cm, right=2cm, top=2.5cm, bottom=2.5cm}

\begin{document}
\pagestyle{empty}
\begin{center}\large 内积和范数\end{center}
\tbf{1.内积}\\
首先说明定义内积的动机.对于二维空间中的向量$v=(x,y)$,其\tbf{范数}(即我们所说的模长)为
$$||v||=\sqrt{x^2+y^2}$$
尽管高于三维的情况我们并不能想象出来,但是我们仍然可以定义$\R^n$中的向量$v=\left(\li x,n\right)$的范数为
$$||v||=\sqrt{\sum_{i=1}^{n}x_i^2}$$
范数并不是线性的.为了将线性引入我们的讨论,我们引入向量的点积.
\begin{definition}[1.1 定义:点积]
    对$u,v\in\R^n$,$u$和$v$的点积记作$u\cdot v$,由
    $$u\cdot v=\sum_{i=1}^{n}x_iy_i$$
    定义.其中$u=(\li x,n),v=(\li y,n)$.
\end{definition}\noindent
$\R^n$上的点积满足如下性质.
\begin{enumerate}[label=\tbf{(\arabic*)}]
    \item 对所有$x\in\R^n$,$||x||^2=x\cdot x$.
    \item 对所有$x\in\R^n$,$x\cdot x\geqslant 0$,当且仅当$x=\mbf{0}$时等号成立.
    \item 对于固定的$y\in\R^n$,映射$T:x\mapsto x\cdot y$是线性映射.
    \item 对所有$x,y\in\R^n$,都有$x\cdot y=y\cdot x$.
\end{enumerate}
为了将我们的讨论扩展到复向量空间上,定义$z\in\C^n$的范数为
$$||z||=\sqrt{\sum_{i=1}^n|z_i|^2}=\sqrt{\sum_{i=1}^{n}z_i\bar{z_i}}$$
我们想让$||z||^2$视作$z$与自身的内积,于是上式表明$z,w\in\C^n$的内积应由
$$z\cdot w=z_1\bar{w_1}+\cdots+z_n\bar{w_n}$$
定义.这同时表明复向量空间上的内积不满足交换律,而满足$z\cdot w=\overline{w\cdot z}$.\\
由此,我们可以定义实向量空间或复向量空间上的内积.
\begin{definition}[1.2 定义:内积]
    $V$上的内积是一个函数,任意$u,v\in V$构成的有序对$(u,v)$对应至一个标量$\langle u,v\rangle\in\F$,并满足如下性质.
    \begin{enumerate}[label=\tbf{(\arabic*)}]
        \item \tbf{正性}:对所有$v\in V$,$\langle v,v\rangle\geqslant 0$.
        \item \tbf{定性}:$\langle v,v\rangle=0$当且仅当$v=\mbf{0}$.
        \item \tbf{第一位可加性}:对所有$u,v,w\in V$,都有$\langle u+v,w\rangle=\langle u,w\rangle+\langle v,w\rangle$.
        \item \tbf{第一位齐次性}:对所有$\lambda\in\F$和所有$u,v\in V$,均有$\langle\lambda u,v\rangle=\lambda\langle u,v\rangle$.
        \item \tbf{共轭对称性}:对所有$u,v\in V$,均有$\langle u,v\rangle=\overline{\langle v,u\rangle}$.
    \end{enumerate}
\end{definition}\noindent
有些时候,物理学家们要求齐次性在第二位上成立而不是第一位.\\
我们说到的点积就是内积的一个典型的例子,这也称为\tbf{欧几里得内积}.
\begin{definition}[1.3 定义:内积空间]
    一个\tbf{内积空间}是带有内积的向量空间$V$.
\end{definition}\noindent
当我们讨论$\F^n$时,如无特殊说明,都应假设其上定义的内积是欧几里得内积.
\begin{formal}[1.4 内积的性质]
    内积具有如下性质.
    \begin{enumerate}[label=\tbf{(\arabic*)}]
        \item 对于固定的$y\in V$,映射$T:x\mapsto \langle x,y\rangle$是线性映射.
        \item 对于任意$v\in V$,都有$\langle\mbf{0},v\rangle=\langle v,\mbf{0}\rangle=0$.
        \item 对所有$u,v,w\in V$,都有$\langle u,v+w\rangle=\langle u,v\rangle+\langle u,w\rangle$.
        \item 对所有$\lambda\in\F$和所有$u,v\in V$,均有$\langle u,\lambda v\rangle=\bar{\lambda}\langle u,v\rangle$.
    \end{enumerate}
\end{formal}\noindent
\tbf{2.范数}\\
我们定义内积的最初动机来源于$\R^2$上的向量的范数.现在我们会看到每种内积都能确定对应的范数.
\begin{definition}[2.1 定义:范数]
    对$v\in V$,$v$的\tbf{范数}$||v||$定义为
    $$||v||=\sqrt{\langle v,v\rangle}$$
\end{definition}\noindent
如同我们之前提到的模长的定义,范数具有以下的基本性质.
\begin{formal}[2.2 范数的基本性质]
    设$v\in V$.于是
    \begin{enumerate}[label=\tbf{(\arabic*)}]
        \item $||v||=0$当且仅当$v=\mbf{0}$.
        \item 对所有$\lambda\in\F$,均有$||\lambda v||=|\lambda|||v||$.
    \end{enumerate}
\end{formal}\noindent
上面的\tbf{(2)}告诉我们研究范数的平方通常比研究范数本身要容易一些.
现在我们给出一个关键的定义:正交.
\begin{definition}[2.3 定义:正交]
    称两个向量$u,v\in V$是\tbf{正交的},如果$\langle u,v\rangle=0$.
\end{definition}\noindent
根据点积的定义,$\R^2$和$\R^3$中垂直的向量就是正交的.于是在特殊的情况下,你也可以视正交为表达垂直的一种更为酷炫的说法.
我们从简单地结论开始研究正交性.在研究的过程中,可以时刻利用垂直的几何直观帮助我们理解问题.
\begin{formal}[2.4 正交性和$\mbf{0}$向量]
    $\mbf{0}$与$V$中的任意向量正交.特别地,它是$V$中唯一与自身正交的向量.
\end{formal}\noindent
以及,我们从勾股定理出发可以得到以下等式.
\begin{formal}[2.5 勾股定理(毕达哥拉斯定理)]
    若$u,v\in V$正交,那么$||u+v||^2=||u||^2+||v||^2$.
\end{formal}
\begin{proof}
    设$\langle u,v\rangle=0$,于是
    $$\begin{aligned}
        ||u+v||^2
        &= \langle u+v,u+v\rangle \\
        &= \langle u,u\rangle+\langle u,v\rangle+\langle v,u\rangle+\langle v,v\rangle \\
        &= ||u||^2+||v||^2
    \end{aligned}$$
\end{proof}\noindent
在这一定理的帮助下,我们可以对向量进行正交分解.设$u,v\in V$且均不为$\mbf{0}$.我们想要把$u$用$v$和与$v$正交的$w$线性表出.%
于是令$c\in\F$,则有$u=cv+(u-cv)$.只需令$v$与$u-cv$正交即可.于是
$$0=\langle u-cv,v\rangle=\langle u,v\rangle-c||v||^2$$
于是$c=\dfrac{\langle u,v\rangle}{||v||^2}$,即
$$u=\dfrac{\langle u,v\rangle}{||v||^2}+\left(u-\dfrac{\langle u,v\rangle}{||v||^2}v\right)$$
于是我们就得到了正交分解的形式.
\begin{formal}[2.6 正交分解]
    设$u,v\in V$且$v\neq\mbf{0}$.取$c=\dfrac{\langle u,v\rangle}{||v||^2}$且$w=u-\dfrac{\langle u,v\rangle}{||v||^2}v$,那么
    $$u-cv+w\text{  且  }\langle w,v\rangle=0$$
\end{formal}\noindent
我们利用正交分解来证明Cauchy-Schwarz不等式.
\begin{formal}[2.7 Cauchy-Schwarz不等式]
    设$u,v\in V$,那么$$|\langle u,v\rangle|\leqslant||u||||v||$$
    当且仅当$u,v$成标量倍的关系时等号成立.
\end{formal}
\begin{proof}
    若$v=\mbf{0}$,那么欲证结论当然成立.\\
    若$v\neq\mbf{0}$,那么考虑$u$的正交分解$u=\dfrac{\langle u,v\rangle}{||v||^2}v+w$.根据勾股定理有
    $$||u^2||=\left|\left|\dfrac{\langle u,v\rangle}{||v||^2}v\right|\right|^2+||w||^2=\dfrac{|\langle u,v\rangle|^2}{||v||^2}+||w^2||\geqslant\dfrac{|\langle u,v\rangle|^2}{||v||^2}$$
    整理后开平方根即可得欲证等式.特别地,等号成立当且仅当$w=\mbf{0}$,这表明$u=cv$,即$u$是$v$的标量倍.
\end{proof}\noindent
由此,我们可以推出两个重要的式子.
\begin{formal}[2.8 三角不等式与平行四边形等式]
    设$u,v\in V$,那么
    \begin{enumerate}[label=\tbf{(\arabic*)}]
        \item $||u+v||\leqslant ||u||+||v||$.
        \item $||u+v||^2+||u-v||^2=2\left(||u||^2+||v||^2\right)$.
    \end{enumerate}
\end{formal}
\end{document}