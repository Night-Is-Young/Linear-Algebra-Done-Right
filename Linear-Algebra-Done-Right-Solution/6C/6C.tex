\documentclass{ctexart}
\usepackage{geometry}
\usepackage[dvipsnames,svgnames]{xcolor}
\usepackage[strict]{changepage}
\usepackage{framed}
\usepackage{enumerate}
\usepackage{amsmath,amsthm,amssymb}
\usepackage{enumitem}
\usepackage{solution}

\allowdisplaybreaks
\geometry{left=2cm, right=2cm, top=2.5cm, bottom=2.5cm}

\begin{document}\pagestyle{empty}
\begin{center}
    \large\tbf{Linear Algebra Done Right 6C}
\end{center}
\begin{problem}[1.]
    设$\li v,m\in V$.试证明
    \[\left\{\li v,m\right\}^{\bot}=\left(\span\left(\li v,m\right)\right)^{\bot}\]
\end{problem}
\begin{proof}
    由于$\left\{\li v,m\right\}\subseteq\span\left(\li v,m\right)$,于是根据正交补的性质有$\left(\span\left(\li v,m\right)\right)^{\bot}\subseteq\left\{\li v,m\right\}^{\bot}$.\\
    对于任意$u\in\left\{\li v,m\right\}^\bot$,都满足$\inprod{u}{v_k}=0$对任意$k\in\{1,\cdots,m\}$成立.\\
    于是对于任意$v:=a_1v_1+\cdots+a_mv_m\in\span\left(\li v,m\right)$有
    \[\inprod{u}{v}=\inprod{u}{\sum_{k=1}^{m}a_kv_k}=\sum_{k=1}^{m}a_k\inprod{u}{v_k}=0\]
    从而$u\in\left(\span\left(\li v,m\right)\right)^{\bot}$,因而$\left\{\li v,m\right\}\subseteq\left(\span\left(\li v,m\right)\right)^{\bot}$.\\
    综上可知$\left\{\li v,m\right\}^{\bot}=\left(\span\left(\li v,m\right)\right)^{\bot}$.
\end{proof}
\begin{problem}[2.]
    设$U$是$V$的子空间,且有一组基$\li u,m$.向量组$\li u,m,\li v,n$是$V$的一组基.对上述$V$的基运用Gram-Schmidt过程得到$V$的规范正交基$\li e,m,\li f,n$.试证明:$\li e,m$是$U$的规范正交基,$\li f,n$是$U^\bot$的规范正交基.
\end{problem}
\begin{proof}
    对$\li u,m$应用Gram-Schmidt过程得到的$\li e,m$自然是$U$的规范正交基.\\
    对于任意$k\in\{1,\cdots,n\}$和任意$j\in\{1,\cdots,m\}$,都有$\inprod{f_k}{e_j}=0$,于是$f_k\in U^\bot$.\\
    于是$\li f,n$是$U^\bot$中的规范正交组.\\
    又因为$\dim U^\bot=\dim V-\dim U=n$,因而$\li f,n$是$U^\bot$的规范正交基.
\end{proof}
\begin{problem}[3.]
    设$U$是$\R^4$的子空间,其定义为
    \[U=\span\left(\left(1,2,3,-4\right),\left(-5,4,3,2\right)\right)\]
    求$U$的一规范正交基和$U^\bot$的一规范正交基.
\end{problem}
\begin{proof}
    将$\left(1,2,3,-4\right),\left(-5,4,3,2\right)$扩展为$V$的一组基
    \[\left(1,2,3,-4\right),\left(-5,4,3,2\right),\left(1,0,0,0\right),(0,1,0,0)\]
    对这组基运用Gram-Schmidt过程,得到
    \[e_1=\dfrac{1}{\sqrt{30}}\left(1,2,3,-4\right),e_2=\dfrac{1}{\sqrt{12030}}\left(-77,56,39,38\right)\]
    \[e_3=\dfrac{1}{\sqrt{76190}}\left(190,117,60,151\right),e_4=\dfrac{1}{\sqrt{190}}\left(0,9,-10,3\right)\]
    根据\tbf{6C.2}可知$e_1,e_2$是$U$的基,$e_3,e_4$是$U^\bot$的基.
\end{proof}
\begin{problem}[4.]
    设$\li e,n$是$V$中的一组向量,满足
    \begin{enumerate}[label=\tbf{(\alph*)}]
        \item 对任意$k\in\{1,\cdots,n\}$,都有$||e_k||=1$.
        \item 对任意$v\in V$,都有$||v||^2=\left|\inprod{v}{e_1}\right|^2+\cdots+\left|\inprod{v}{e_n}\right|^2$.
    \end{enumerate}
    试证明:$\li e,n$是$V$的规范正交基.
\end{problem}
\begin{proof}
    根据题设条件,对于任意$k\in\{1,\cdots,n\}$有
    \[||e_k||^2=\left|\inprod{e_k}{e_1}\right|^2+\cdots+\left|\inprod{e_k}{e_n}\right|^2=1\]
    又因为$\left|\inprod{e_k}{e_k}\right|=||e_k||=1$,于是
    \[\sum_{j=1,j\neq k}^{n}\left|\inprod{e_k}{e_j}\right|^2=0\]
    这表明对任意$j,k\in\{1,\cdots,n\}$且$j\neq k$都有$\inprod{e_k}{e_j}=0$.\\
    于是$\li e,n$是$V$中的规范正交组.\\
    考虑任意的$v\in V$,根据Bessel不等式可知
    \[\sum_{k=1}^{n}|\langle v,e_k\rangle|^2\leqslant||v||^2\]
    当且仅当$\displaystyle v=\sum_{k=1}^{n}|\langle v,e_k\rangle|e_k$时等号成立.\\
    于是$v\in\span\left(\li e,n\right)$,即$V=\span\left(\li e,n\right)$,从而$\li e,n$为$V$的规范正交基.
\end{proof}
\begin{problem}[5.]
    设$V$是有限维的,且$U$为$V$的子空间,试证明:$P_{U^\bot}=I-P_U$,其中$I$是$V$上的恒等算子.
\end{problem}
\begin{proof}
    由于$V=U\oplus U^\bot$,于是对于任意$v\in V$,其都可以被唯一分解为$v=u+w$,其中$u\in U,w\in U^\bot$.\\
    根据正交投影的定义,我们有$P_Uv=u,P_{U^\bot}v=w$,于是
    \[v=P_Uv+P_{U^\bot}v\]
    从而
    \[I=P_U+P_{U^\bot}\]
    移项即可得欲证等式.
\end{proof}
\begin{problem}[6.]
    设$V$是有限维的,且$T\in\L(V,W)$.试证明
    \[T=TP_{(\nul T)^\bot}=P_{\range T}T\]
\end{problem}
\begin{proof}
    根据\tbf{6C.5}有$P_{(\nul T)^\bot}=I-P_{\nul T}$.对任意$v\in V$,都有$P_{\nul T}v\in\nul T$,于是
    \[TP_{(\nul T)^\bot}v=T\left(I-P_{\nul T}\right)v=Tv-\mbf0=Tv\]
    于是$TP_{(\nul T)^\bot}=T$.\\
    对于任意$w\in\range T$,都有$P_{\range T}w=w$.于是$T=P_{\range T}T$.\\
    综上,命题得证.
\end{proof}
\begin{problem}[7.]
    设$X$和$Y$为$V$的有限维子空间.试证明:$P_XP_Y=\mbf0$当且仅当$\inprod xy=0$对所有$x\in X$和所有$y\in Y$都成立.
\end{problem}
\begin{proof}
    $\Rightarrow$:$P_XP_Y=\mbf0$即对任意$v\in V$有$P_XP_Yv=\mbf0$.又$\range P_Y=Y$,于是对于任意$y\in Y$有$P_Xy=\mbf0$.\\
    由于$V=X\oplus X^\bot$,于是存在唯一的分解$y=x+x'$使得$x\in X,x'\in X^\bot$.\\
    又因为$P_Xy=\mbf0$,即上述分解中$x=\mbf0$,于是$y=x'\in X^\bot$.即对于任意$x\in X$有$\inprod xy=0$.\\
    $\Leftarrow$:考虑$X$的规范正交基$\li e,n$.对于任意$v\in V$,有$P_Yv\in Y$,于是
    \[P_X\left(P_Yv\right)=\sum_{k=1}^{n}\inprod{P_Yv}{e_k}e_k=\mbf0\]
    于是$P_XP_Y=\mbf0$.
\end{proof}
\begin{problem}[8.]
    设$U$是$V$的有限维子空间,且$v\in V$.定义$U$上的线性泛函$\phi:U\to\F$为
    \(\phi(u)=\inprod uv\)
    对所有$u\in U$成立.根据Riesz表示定理,存在唯一$w\in U$使得
    \(\phi(u)=\inprod uw\)
    对所有$u\in U$成立.试证明:$w=P_Uv$.
\end{problem}
\begin{proof}
    因为$v-P_Uv\in U^\bot$,于是$\inprod{u}{v-P_Uv}$.于是
    \[\inprod uv=\inprod{u}{v-P_Uv}+\inprod{u}{P_Uv}=\inprod{u}{P_Uv}\]
    因此$\phi(u)=\inprod uv=\inprod{u}{P_Uv}$.而$P_Uv\in U$.根据Riesz表示定理,这样的向量是唯一存在的,于是$w=P_Uv$.
\end{proof}
\begin{problem}[9.]
    设$V$是有限维的,$P\in\L(V)$使得$P^2=P$且$\nul P$中的任意向量都正交与$\range P$中的任意向量.试证明:存在$V$的子空间$U$使得$P=P_U$.
\end{problem}
\begin{proof}
    根据\tbf{3B.27}可知$V=\range P\oplus\nul P$.又因为$V=\range P\oplus\left(\range P\right)^\bot$,于是
    \[\dim\nul P=\dim\left(\range P\right)^\bot\]
    根据题意可知$\nul P\subseteq\left(\range P\right)^{\bot}$.于是$\nul P=\left(\range P\right)^\bot$.\\
    令$U=\range P$.对于任意$v:=Px+w\in V$,其中$Px\in\range P,w\in\nul P$,有
    \[P_Uv=Px=P\left(Px+w\right)=Pv\]
    此时$P_U=P$.
\end{proof}
\begin{problem}[10.]
    设$V$是有限维的,$P\in\L(V)$使得$P^2=P$且$||Pv||\leqslant||v||$对任意$v\in V$成立.试证明:存在$V$的子空间$U$使得$P=P_U$.
\end{problem}
\begin{proof}
    考虑$w\in\nul P$和$Px\in\range P$,根据题意,对任意$\lambda\in\F$有
    \[||Px||=||P(Px+\lambda w)||\leqslant||Px+\lambda w||\]
    根据\tbf{6A.6}可知$\inprod{w}{Px}=0$,即$\nul P$中的任意向量都正交与$\range P$中的任意向量.\\
    根据\tbf{6C.9},取$U=\range P$即可使$P=P_U$.
\end{proof}
\begin{problem}[11.]
    设$T\in\L(V)$且$U$是$V$的有限维子空间.试证明:$U$在$T$下不变,当且仅当$P_UTP_U=TP_U$.
\end{problem}
\begin{proof}
    $\Rightarrow$:对于任意$v\in V$,都有$P_Uv\in U$.因为$U$在$T$下不变,于是$T\left(P_Uv\right)\in U$.\\
    于是$P_U\left(T\left(P_Uv\right)\right)=T\left(P_Uv\right)$.因而$P_UTP_U=TP_U$.\\
    $\Leftarrow$:如果$U$不在$T$下不变,那么存在$u\in U$使$Tu\notin U$.于是
    \[P_UTP_Uu=P_UTu\neq Tu=TP_Uu\]
    于是$P_UTP_U\neq TP_U$.
\end{proof}
\begin{problem}[12.]
    设$V$是有限维的,$T\in\L(V)$,且$U$是$V$的子空间.试证明$U$和$U^\bot$在$T$下不变,当且仅当$P_UT=TP_U$.
\end{problem}
\begin{proof}
    $\Rightarrow$:对任意$v:=u+w\in V$,其中$u\in U,w\in U^\bot$,都有$Tu\in U,Tw\in U^\bot$.于是
    \[P_UTv=P_U\left(Tu+Tw\right)=Tu=TP_Uv\]
    于是$P_UT=TP_U$.\\
    $\Leftarrow$:若存在$u\in U$使得$Tu\notin U$,那么
    \[TP_Uu=Tu\notin U,P_U(Tu)\in U\]
    于是$TP_U\neq P_UT$,与题设矛盾.从而$Tu\in U$对任意$u\in U$成立.\\
    若存在$w\in U^\bot$使得$Tw\notin U^\bot$,那么
    \[TP_Uw=T\mbf0=\mbf0,P_UTw\neq\mbf0\]
    同理可推出矛盾,因此$Tw\in U^\bot$对任意$w\in U^\bot$成立.\\
    于是$U$和$U^\bot$在$T$下不变.
\end{proof}
\begin{problem}[13.]
    设$\F=\R$,$V$是有限维向量空间.对任意$v\in V$,令$\phi_v$为$V$上的线性泛函,定义为$\phi_v(u)=\inprod uv$对所有$u\in V$成立.
    \begin{enumerate}[label=\tbf{(\arabic*)}]
        \item 试证明:$v\mapsto\phi_v$是$V$到$V'$的单的线性映射.
        \item 试证明:$v\mapsto\phi_v$是$V$到$V'$的同构.
    \end{enumerate}
\end{problem}
\begin{proof}
    令映射$T:V\to V'$为$Tv=\phi_v$对所有$v\in V$成立.
    \begin{enumerate}[label=\tbf{(\arabic*)}]
        \item 对于任意$v,w\in V$,对于任意$u\in V$有
            \[\phi_v(u)+\phi_w(u)=\inprod uv+\inprod uw=\inprod{u}{v+w}=\phi_{v+w}(u)\]
            于是$Tv+Tw=T(v+w)$.\\
            对于任意$v\in V$和$\lambda\in\F$,对于任意$u\in V$有
            \[\lambda\phi_v(u)=\lambda\inprod{u}{v}=\inprod{u}{\lambda v}=\phi_{\lambda v}(u)\]
            于是$\lambda Tv=T\left(\lambda v\right)$.\\
            因此$T$满足可加性和齐次性,是$V$到$V'$的线性映射.\\
            现在,假定存在$v,w\in V$使得$Tv=Tw$.于是对于任意$u\in V$有$\phi_v(u)=\phi_w(u)$.而
            \[\phi_v(u)=\phi_w(u)\Leftrightarrow \inprod uv=\inprod uw\Leftrightarrow \inprod{u}{v-w}=0\Leftrightarrow v-w=\mbf0\Leftrightarrow v=w\]
            于是$T$是单射.
        \item 注意到$\dim V=\dim V'$,又因为$T$是单射,于是$T$是$V$到$V'$的同构.
    \end{enumerate}
\end{proof}
\begin{problem}[14.]
    设$\li e,n$是$V$的规范正交基.对于$v\in V$,令$\phi_v(u)=\inprod uv$对所有$u\in U$成立.试证明:$\li e,n$的对偶基为$\phi_{e_1},\cdots,\phi_{e_n}$.
\end{problem}
\begin{proof}
    对于任意$j,k\in\{1,\cdots,n\}$且$j\neq k$有
    \[\phi_{e_k}\left(e_j\right)=\inprod{e_j}{e_k}=0\]
    又$\phi_{e_k}\left(e_k\right)=\inprod{e_k}{e_k}=1$.于是
    \[\phi_{e_k}\left(e_j\right)=\left\{\begin{array}{l}
        1,j=k\\0,j\neq k
    \end{array}\right.\]
    这符合$e_k$的对偶$\phi_k$的定义.于是$\phi_{e_1},\cdots,\phi_{e_n}$为$\li e,n$的对偶基.
\end{proof}
\end{document}