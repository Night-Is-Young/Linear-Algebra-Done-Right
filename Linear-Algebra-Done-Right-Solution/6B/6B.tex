\documentclass{ctexart}
\usepackage{geometry}
\usepackage[dvipsnames,svgnames]{xcolor}
\usepackage[strict]{changepage}
\usepackage{framed}
\usepackage{enumerate}
\usepackage{amsmath,amsthm,amssymb}
\usepackage{enumitem}
\usepackage{solution}

\allowdisplaybreaks
\geometry{left=2cm, right=2cm, top=2.5cm, bottom=2.5cm}

\begin{document}\pagestyle{empty}
\begin{center}
    \large\tbf{Linear Algebra Done Right 6B}
\end{center}
\begin{problem}[1.]
    设$\li e,n$是$V$中的向量组,使得
    \[||a_1e_1+\cdots+a_ne_n||^2=|a_1|^2+\cdots+|a_m|^2\]
    对任意$\li a,n\in\F$都成立.试证明:$\li e,n$是规范正交组.
\end{problem}
\begin{proof}
    对于任意$k\in\{1,\cdots,n\}$,令
    \[a_j=\left\{\begin{array}{l}
        0,j\neq k\\
        1,j=k
    \end{array}\right.\]
    可得$||e_k||^2=1$,即$\inprod{e_k}{e_k}=1$.\\
    对于任意$k,j\in\{1,\cdots,n\}$且$j\neq k$,令
    \[a_i=\left\{\begin{array}{l}
        0,i\neq j\text{或}i\neq k\\
        1,i=j\\
        t,i=k
    \end{array}\right.\]
    可得$||e_j+te_k||^2=1+|t|^2\geqslant||e_j||^2$.根据\tbf{6A.6}可知$\inprod{e_j}{e_k}=0$.\\
    综上可知$\li e,n$是规范正交组.
\end{proof}
\begin{problem}[2.]
    证明下列命题.
    \begin{enumerate}[label=\tbf{(\arabic*)}]
        \item 设$\theta\in\R$,试证明:$(\cos\theta,\sin\theta),(-\sin\theta,\cos\theta)$和$(\cos\theta,\sin\theta),(\sin\theta,-\cos\theta)$都是$\R^2$的规范正交基.
        \item 试证明:$\R^2$中的每个规范正交基都具有\tbf{(1)}中两个形式之一.
    \end{enumerate}
\end{problem}
\begin{proof}
    \begin{enumerate}[label=\tbf{(\arabic*)}]
        \item 对于任意$a,b\in\R$,都有
            \[\begin{aligned}
                ||a(\cos\theta,\sin\theta),b(-\sin\theta,\cos\theta)||^2
                &= ||a\cos\theta-b\sin\theta,a\sin\theta+b\cos\theta||^2 \\
                &= (a\cos\theta-b\sin\theta)^2+(a\sin\theta+b\cos\theta)^2 \\
                &= a^2+b^2
            \end{aligned}\]
            根据\tbf{6B.1}可知$(\cos\theta,\sin\theta),(-\sin\theta,\cos\theta)$是$\R^2$中的规范正交组.\\
            又因为$\dim\R^2=2$,于是$(\cos\theta,\sin\theta),(-\sin\theta,\cos\theta)$是$\R^2$中的规范正交基.\\
            另一组向量的证明过程相似,在此不再赘述.
        \item 考虑$\R^2$中所有满足$||u||=1$的向量$u$,都应当具有$(\cos\alpha,\sin\alpha)$的形式.\\
            考虑向量$u=(\cos\alpha,\sin\alpha),v=(\cos\beta,\sin\beta)$.令$\inprod uv=0$,则有
            \[\cos\alpha\cos\beta+\sin\alpha\sin\beta=0\]
            即\[\cos(\alpha-\beta)=0\]
            当且仅当$\beta=\alpha+\dfrac{\pi}{2}+k\pi$,$k\in\mathbb{Z}$时成立.\\
            若$k$是偶数,则$v=(-\sin\alpha,\cos\alpha)$.\\
            若$k$是奇数,则$v=(\sin\alpha,-\cos\alpha)$.\\
            于是命题得证.
    \end{enumerate}
\end{proof}
\begin{problem}[3.]
    设$\li e,m$是$V$中的一规范正交组,且$v\in V$.试证明:
    \[||v||^2=\left|\inprod{v}{e_1}\right|^2+\cdots+\left|\inprod{v}{e_m}\right|^2\Leftrightarrow v\in\span(\li e,m)\]
\end{problem}
\begin{proof}
    根据Bessel不等式,$\left|\inprod{v}{e_1}\right|^2+\cdots+\left|\inprod{v}{e_m}\right|^2\leqslant||v||^2$,当且仅当
    \[v-\inprod{v}{e_1}e_1-\cdots-\inprod{v}{e_m}e_m=\mbf0\]
    时等式成立.这等价于$v\in\span(\li e,m)$.
\end{proof}
\begin{problem}[4.]
    设$n\in\N^*$,试证明:
    \[\dfrac{1}{\sqrt{2\pi}},\dfrac{\cos x}{\sqrt{\pi}},\cdots,\dfrac{\cos nx}{\sqrt{\pi}},\dfrac{\sin x}{\sqrt{\pi}},\cdots,\dfrac{\sin nx}{\sqrt{\pi}}\]
    是$C[-\pi,\pi]$中的一规范正交组.\\
    $C[\pi,\pi]$是定义在$[-\pi,\pi]$上的连续实值函数构成的向量空间,其内积定义为$\displaystyle\inprod fg=\int_{-\pi}^{\pi}fg$.
\end{problem}
\begin{proof}
    首先有\[\inprod{\dfrac{1}{\sqrt{2\pi}}}{\dfrac{1}{\sqrt{2\pi}}}=\displaystyle\int_{-\pi}^{\pi}\dfrac{1}{2\pi}\di x=1\]
    又对于任意$k\in\{1,\cdots,n\}$有
    \[\inprod{\dfrac{\cos kx}{\sqrt{\pi}}}{\dfrac{\cos kx}{\sqrt{\pi}}}=\int_{-\pi}^{\pi}\dfrac{\cos^2kx}{\pi}\di x=\int_{-\pi}^{\pi}\dfrac{1+\cos 2kx}{2\pi}\di x=\left.\left(\dfrac{x}{2\pi}+\dfrac{\sin 2kx}{4k\pi}\right)\right|_{-\pi}^{\pi}=1\]
    \[\inprod{\dfrac{\sin kx}{\sqrt{\pi}}}{\dfrac{\sin kx}{\sqrt{\pi}}}=\int_{-\pi}^{\pi}\dfrac{\sin^2kx}{\pi}\di x=\int_{-\pi}^{\pi}\dfrac{1-\cos 2kx}{2\pi}\di x=\left.\left(\dfrac{x}{2\pi}-\dfrac{\sin 2kx}{4k\pi}\right)\right|_{-\pi}^{\pi}1\]
    于是这向量组中各向量的范数均为$1$.又因为
    \[\inprod{\dfrac{1}{\sqrt{2\pi}}}{\dfrac{\cos x}{\sqrt{\pi}}}=\dfrac{1}{2\pi}\int_{-\pi}^{\pi}\cos kx\dx=\left.\dfrac{\sin kx}{2k\pi}\right|_{-\pi}^{\pi}=0\]
    \[\inprod{\dfrac{1}{\sqrt{2\pi}}}{\dfrac{\sin x}{\sqrt{\pi}}}=\dfrac{1}{2\pi}\int_{-\pi}^{\pi}\sin kx\dx=0\]
    对于任意$j,k\in\{1,\cdots,n\}$且$j\neq k$有
    \[\inprod{\dfrac{\cos jx}{\sqrt{\pi}}}{\dfrac{\cos kx}{\sqrt{\pi}}}=\int_{-\pi}^{\pi}\dfrac{\cos(j-k)x+\cos(j+k)x}{2}\dx=\left.\left(\dfrac{\sin(j-k)x}{2(j-k)}+\dfrac{\sin(j+k)x}{2(j+k)}\right)\right|_{-\pi}^{\pi}=0\]
    \[\inprod{\dfrac{\sin jx}{\sqrt{\pi}}}{\dfrac{\sin kx}{\sqrt{\pi}}}=\int_{-\pi}^{\pi}\dfrac{\cos(j-k)x-\cos(j+k)x}{2}\dx=\left.\left(\dfrac{\sin(j-k)x}{2(j-k)}-\dfrac{\sin(j+k)x}{2(j+k)}\right)\right|_{-\pi}^{\pi}=0\]
    \[\inprod{\dfrac{\sin jx}{\sqrt{\pi}}}{\dfrac{\cos kx}{\sqrt{\pi}}}=\int_{-\pi}^{\pi}\dfrac{\sin(j-k)x+\sin(j+k)x}{2}\dx=\left.\left(\dfrac{\cos(j-k)x}{2(k-j)}-\dfrac{\cos(j+k)x}{2(j+k)}\right)\right|_{-\pi}^{\pi}=0\]
    于是这向量组满足规范正交组的定义,命题得证.
\end{proof}
\begin{problem}[5.]
    设$f:[-\pi,\pi]\to\R$是连续的.对于任意$k\in\N$,定义
    \[a_k=\dfrac{1}{\sqrt{\pi}}\int_{-\pi}^{\pi}f(x)\cos(kx)\di x\text{\ \ \ \ \ 和\ \ \ \ \ }b_k=\dfrac{1}{\sqrt{\pi}}\int_{-\pi}^{\pi}f(x)\sin(kx)\dx\]
    试证明:
    \[\dfrac{a_0^2}{2}+\sum_{k=1}^{\infty}\left(a_k^2+b_k^2\right)\leqslant\int_{-\pi}^{\pi}\left(f(x)\right)^2\dx\]
\end{problem}
\begin{proof}
    设$V=C[-\pi,\pi]$,于是$f\in C[-\pi,\pi]$.沿用\tbf{6B.4}中$V$上内积的定义,我们有
    \[\dfrac{a_0^2}{2}=\left|\dfrac{a_0}{\sqrt{2}}\right|^2=\left|\int_{-\pi}^{\pi}\dfrac{1}{\sqrt{2\pi}}f(x)\di x\right|^2=\left|\inprod{f}{\dfrac{1}{\sqrt{2\pi}}}\right|^2\]
    \[a_k^2=\left|a_k\right|^2=\left|\inprod{f}{\dfrac{\cos kx}{\sqrt{\pi}}}\right|^2\]
    \[b_k^2=\left|b_k\right|^2=\left|\inprod{f}{\dfrac{\sin kx}{\sqrt{\pi}}}\right|^2\]
    对于任意$n\in\N$,根据\tbf{6B.4}可知
    \[\dfrac{1}{\sqrt{2\pi}},\dfrac{\cos x}{\sqrt{\pi}},\cdots,\dfrac{\cos nx}{\sqrt{\pi}},\dfrac{\sin x}{\sqrt{\pi}},\cdots,\dfrac{\sin nx}{\sqrt{\pi}}\]
    是$C[-\pi,\pi]$上的规范正交组.根据Bessel不等式有
    \[\left|\inprod{f}{\dfrac{1}{\sqrt{2\pi}}}\right|^2+\left|\inprod{f}{\dfrac{\cos x}{\sqrt{\pi}}}\right|^2+\cdots+\left|\inprod{f}{\dfrac{\cos nx}{\sqrt{\pi}}}\right|^2+\left|\inprod{f}{\dfrac{\sin x}{\sqrt{\pi}}}\right|^2+\cdots+\left|\inprod{f}{\dfrac{\sin nx}{\sqrt{\pi}}}\right|^2\leqslant||f||^2\]
    即
    \[\dfrac{a_0^2}{2}+\sum_{k=1}^{n}\left(a_k^2+b_k^2\right)\leqslant\int_{-\pi}^{\pi}\left(f(x)\right)^2\dx\]
    因为上式对所有正整数$n$都成立,于是两边对$n$取极限有
    \[\dfrac{a_0^2}{2}+\sum_{k=1}^{\infty}\left(a_k^2+b_k^2\right)\leqslant\int_{-\pi}^{\pi}\left(f(x)\right)^2\dx\]
\end{proof}
\begin{problem}[6.]
    设$\li e,n$是$V$中的规范正交基.
    \begin{enumerate}[label=\tbf{(\arabic*)}]
        \item 试证明:如果$\li v,n$是$V$中的向量组,满足
            \[||e_k-v_k||<\dfrac{1}{\sqrt{n}}\]
            对于任意$k\in\{1,\cdots,n\}$成立,那么$\li v,n$是$V$的基.
        \item 试证明:存在$V$中的向量组$\li v,n$,满足
            \[||e_k-v_k||\leqslant\dfrac{1}{\sqrt{n}}\]
            对于任意$k\in\{1,\cdots,n\}$成立,且$\li v,n$不是线性无关组.
    \end{enumerate}
\end{problem}
\begin{proof}
    \begin{enumerate}[label=\tbf{(\arabic*)}]
        \item 设$\li a,n\in\F$使得
            \[a_1v_1+\cdots+a_nv_n=\mbf0\]
            则有
            \[\sum_{k=1}^{n}|a_k|^2=\left|\left|\sum_{k=1}^{n}a_ke_k\right|\right|^2
            =\left|\left|\sum_{k=1}^{n}a_k\left(e_k-v_k\right)\right|\right|^2
            \leqslant\sum_{k=1}^{n}|a_k|^2||e_k-v_k||^2
            \leqslant\sum_{k=1}^{n}|a_k|^2\cdot\sum_{k=1}^{n}||e_k-v_k||^2\]
            又
            \[\sum_{k=1}^{n}||e_k-v_k||^2<n\cdot\dfrac{1}{n}=1\]
            于是$\displaystyle\sum_{k=1}^{n}|a_k|^2=0$,即$\li a=n=0$.\\
            于是$\li v,n$线性无关.因而$\li v,n$是$V$的基.
        \item 令$v_k=e_k-\dfrac{\li e+n}{n}$.注意到
            \[\li v+n=\li e+n-n\left(\dfrac{\li e+n}{n}\right)=\mbf0\]
            于是$\li v,n$不是线性无关组.另一方面,有
            \[\left|\left|v_k-e_k\right|\right|=\left|\left|\dfrac{\li e+n}{n}\right|\right|=\dfrac{1}{n}||\li e+n||=\dfrac{\sqrt{n}}{n}\leqslant\dfrac{1}{\sqrt{n}}\]
            于是这样的$\li v,n$满足题意.
    \end{enumerate}
\end{proof}
\begin{problem}[7.]
    设$T\in\L(\R^3)$关于基$(1,0,0),(1,1,1),(1,1,2)$有上三角矩阵.求$\R^3$的一规范正交基,使得$T$关于其有上三角矩阵.
\end{problem}
\begin{solution}
    令$v_1=(1,0,0),v_2=(1,1,1),v_3=(1,1,2)$.\\
    对$v_1,v_2,v_3$应用Gram-Schmidt过程得到$\R^3$的一规范正交基
    \[e_1=(1,0,0),e_2=\left(0,\dfrac{1}{\sqrt{2}},\dfrac{1}{\sqrt{2}}\right),e_3=\left(0,-\dfrac{1}{\sqrt{2}},\dfrac{1}{\sqrt{2}}\right)\]
    且满足$\span(\li v,k)=\span(\li e,k)$.于是$T$在$\span(\li e,k)$下不变,即$T$关于$\li e,3$有上三角矩阵.
\end{solution}
\begin{problem}[8.]
    定义$\P_2(\R)$的内积为$\inprod pq=\displaystyle\int_0^1pq$,使得$\P_2(\R)$成为内积空间.回答下列问题.
    \begin{enumerate}[label=\tbf{(\arabic*)}]
        \item 对$\P_2(\R)$的标准基应用Gram-Schmidt过程,得到$\P_2(\R)$的一组规范正交基.
        \item $\P_2(\R)$上的微分算子$D$关于基$1,x,x^2$有上三角矩阵.求$\P_2(\R)$关于\tbf{(1)}中所求规范正交基的矩阵,并验证这矩阵是上三角矩阵.
    \end{enumerate}
\end{problem}
\begin{proof}
    \begin{enumerate}[label=\tbf{(\arabic*)}]
        \item 令$v_1=1,v_2=x,v_3=x^2$.我们有
            \[f_1=v_1=1\]
            \[f_2=v_2-\dfrac{\inprod{v_2}{f_1}}{||f_1||^2}f_1=x-\dfrac12\]
            \[f_3=v_3-\dfrac{\inprod{v_3}{f_1}}{||f_1||^2}f_1-\dfrac{\inprod{v_3}{f_2}}{||f_2||^2}f_2=x^2-\dfrac{1}{3}-\left(x-\dfrac12\right)=x^2-x+\dfrac{1}{6}\]
            于是
            \[e_1=\dfrac{f_1}{||f_1||}=1,e_2=\dfrac{f_2}{||f_2||}=2\sqrt{3}\left(x-\dfrac{1}{2}\right),e_3=\dfrac{f_3}{||f_3||}=6\sqrt{5}\left(x^2-x+\dfrac{1}{6}\right)\]
            即为所求的规范正交基.
        \item 不难得出,所求矩阵为
            \[\mathcal{M}(D)=\begin{pmatrix}
                0&2\sqrt{3}&0\\
                0&0&2\sqrt{15}\\
                0&0&0
            \end{pmatrix}\]
            这显然是一个上三角矩阵.
    \end{enumerate}
\end{proof}
\begin{problem}[9.]
    设$\li e,m$是对$V$中的线性无关组$\li v,m$运用Gram-Schmidt过程得到的规范正交组.试证明:对于任意$k\in\{1,\cdots,m\}$,都有$\inprod{v_k}{e_k}>0$.
\end{problem}
\begin{proof}
    我们有
    \[\begin{aligned}
        \inprod{v_k}{e_k}
        &= \dfrac{1}{||f_k||}\inprod{v_k}{f_k} \\
        &= \dfrac{1}{||f_k||}\inprod{v_k}{v_k-\sum_{j=1}^{k-1}\dfrac{\inprod{v_k}{f_j}}{||f_j||^2}f_j} \\
        &= \dfrac{1}{||f_k||}\left(||v_k||^2-\sum_{j=1}^{k-1}\dfrac{\left|\inprod{v_k}{f_j}\right|^2}{||f_j||^2}\right) \\
        &= \dfrac{1}{||f_k||}\left(||v_k||^2-\sum_{j=1}^{k-1}\left|\inprod{v_k}{e_j}\right|^2\right)
    \end{aligned}\]
    又因为$v_k\notin\span(\li e,{k-1})$,于是根据Bessel不等式有
    \[||v_k||^2>\sum_{j=1}^{k-1}\left|\inprod{v_k}{e_j}\right|^2\]
    于是$\inprod{v_k}{e_k}>0$.
\end{proof}
\begin{problem}[10.]
    设$\li v,m$是$V$中的线性无关组.试证明:通过Gram-Schmidt过程得到的规范正交组$\li e,m$是仅有的对任意$k\in\{1,\cdots,m\}$都有$\inprod{v_k}{e_k}>0$且$\span(\li v,k)=\span(\li e,k)$的规范正交组.
\end{problem}
\begin{lemma}[Lemma.10]
    设$\li v,n$是$V$中的线性无关组,对其应用Gram-Schmidt过程得到$V$上的规范正交组$\li e,n$.%
    令$S=\{\lambda\in\F:|\lambda|=1\}$和$m\in\N^*$,设$S^m$为所有将$\{1,\cdots,m\}$映射到$S$的函数的集合.%
    那么对于$V$的规范正交组$\li u,m$使得对于任意$k\in\{1,\cdots,n\}$都有$\span(\li u,k)=\span(\li v,k)$,%
    一定存在$f\in S^{m}$使得对于任意$k\in\{1,\cdots,n\}$有$u_k=f(k)e_k$.
\end{lemma}
\begin{proof}
    对于给定的$f\in S^m$,可知对于任意$j,k\in\{1,\cdots,m\}$且$j\neq k$有
    \[||f(j)e_j||=|f(j)|||e_j||=1\ \ \ \ \ \inprod{f(j)e_j}{f(k)e_k}=f(j)\overline{f(k)}\inprod{e_j}{e_k}=0\]
    又因为$0\notin S$,于是
    \[\span(f(1)e_1,\cdots,f(k)e_k)=\span(\li e,k)=\span(\li v,k)\]
    于是$f(1)e_1,\cdots,f(m)e_m$是满足
    \[\span(f(1)e_1,\cdots,f(k)e_k)=\span(\li v,k),\forall k\in\{1,\cdots,m\}\]
    的规范正交组.现在考虑题设中的规范正交组$\li u,m$,则有
    \[\span(\li u,k)=\span(\li v,k)=\span(\li e,k),\forall k\in\{1,\cdots,m\}\]
    于是$\span(u_1)=\span(e_1)$.令$u_1=\lambda_1e_1$.因为$||u_1||=||e_1||=1$,于是$|\lambda_1|=1$,即$\lambda_1\in S$.\\
    由于$\span(u_1,u_2)=\span(e_1,e_2)$,于是$u_2\in\span(e_1,e_2)$.于是
    \[u_2=\inprod{u_2}{e_1}e_1+\inprod{u_2}{e_2}e_2\]
    又因为$u_1,u_2$是规范正交组,于是
    \[0=\inprod{u_1}{u_2}=\inprod{\lambda_1e_1}{u_2}=\lambda_1\inprod{e_1}{u_2}\]
    因为$\lambda_1\neq0$,于是$\inprod{e_1}{u_2}=0$,于是
    \[u_2=\inprod{u_2}{e_2}e_2\]
    令$\lambda_2=\inprod{u_2}{e_2}$,同理可知$|\lambda_2|=1$,即$\lambda_2\in S$.\\
    依此过程构造$\li\lambda,m\in S$,令$f\in S^m$满足$f(k)=\lambda_k$对任意$k\in\{1,\cdots,m\}$成立.\\
    于是$\li u,m$就被写成$f(1)e_1,\cdots,f(m)e_m$的形式,命题得证.
\end{proof}
\begin{solution}
    回到\tbf{6B.10},假定存在$\li u,k$亦满足题设条件,则存在$f\in S^m$使得$u_k=f(k)e_k$对任意$k\in\{1,\cdots,m\}$成立.\\
    此时有
    \[\inprod{v_k}{u_k}=\inprod{v_k}{f(k)u_k}=\overline{f(k)}\inprod{v_k}{e_k}>0\]
    这要求$\overline{f(k)}\in\R$且$\overline{f(k)}>0$.又因为$f(k)\in S$,于是$f(k)=1$,从而表明$u_k=e_k$.\\
    于是只有$\li e,m$满足题意.
\end{solution}
\begin{problem}[11.]
    求多项式$q\in\P_2(\R)$使得$p\left(\dfrac12\right)=\displaystyle\int_0^1pq$对任意$p\in\P_2(\R)$都成立.
\end{problem}
\begin{solution}
    令$\phi\in\left(\P_2(\R)\right)'$为$\phi(p)=p\left(\dfrac12\right)$.根据\tbf{6B.8},令$e_1,e_2,e_3$为$\P_2(\R)$的标准基.令
    \[q=\phi(e_1)e_1+\phi(e_2)e_2+\phi(e_3)e_3=-15x^2+15x-\dfrac32\]
    于是
    \[\phi(p)=p\left(\dfrac12\right)=\inprod pq=\int_0^1 pq\]
\end{solution}
\begin{problem}[12.]
    求多项式$q\in\P_2(\R)$使得$\displaystyle\int_0^1p(x)\cos(\pi x)\di x=\int_0^1pq$对任意$p\in\P_2(\R)$都成立.
\end{problem}
\begin{solution}
    令$\phi\in\left(\P_2(\R)\right)'$为$\phi(p)=\displaystyle\int_0^1p(x)\cos(\pi x)\dx$,与\tbf{6B.11}同理令
    \[q=\phi(e_1)e_1+\phi(e_2)e_2+\phi(e_3)e_3=-\dfrac{24}{\pi^2}\left(x-\dfrac12\right)\]
    于是
    \[\phi(p)=\int_0^1p(x)\cos(\pi x)=\inprod pq=\int_0^1 pq\]
\end{solution}
\begin{problem}[13.]
    试证明:$V$中的一组向量组$\li v,m$线性相关,当且仅当使用Gram-Schmidt过程得到的某个$f_k=\mbf0(k\in\{1,\cdots,m\})$.
\end{problem}
\begin{proof}
    $\Leftarrow$:如果$\li v,m$线性无关,那么根据Gram-Schmidt过程,$\li f,k$必然都是非零的.\\
    取上面命题的逆否命题,即若存在$k\in\{1,\cdots,m\}$使得$f_k=\mbf0$,那么$\li v,m$线性相关.\\
    $\Rightarrow$:设$\li v,m$线性相关.若$v_1=\mbf0$,那么$f_1=\mbf0$.\\
    否则,令$k\in\{2,\cdots,m\}$是使得$v_k\in\span(\li v,{k-1})$成立的最小的数.\\
    于是$\li v,{k-1}$线性无关,并且根据Gram-Schmidt过程,可知$\span(\li v,{k-1})=\span(\li f,{k-1})$.\\
    于是$v_k\in\span(\li f,{k-1})$.不妨令$v_k=a_1f_1+\cdots+a_{k-1}f_{k-1}$.注意到对任意$j\in\{1,\cdots,k-1\}$有
    \[\inprod{v_k}{f_j}=\inprod{a_jf_j}{f_j}=a_j||f_j||^2\]
    于是
    \[v_k=\sum_{j=1}^{k-1}\dfrac{\inprod{v_k}{f_j}}{||f_j||^2}f_j\]
    于是
    \[f_k=v_k-\sum_{j=1}^{k-1}\dfrac{\inprod{v_k}{f_j}}{||f_j||^2}f_j=v_k-v_k=\mbf0\]
\end{proof}
\begin{problem}[14.]
    设$V$是实内积空间,且$\li v,m\in V$线性无关.试证明:$V$中恰好存在$2^m$个规范正交组$\li e,m$使得
    \[\span(\li v,k)=\span(\li e,k)\]
    对所有$k\in\{1,\cdots,m\}$成立.
\end{problem}
\begin{proof}
    首先,对$\li v,m$用Gram-Schmidt过程构造的规范正交向量组$\li e,m$是符合题意的.\\
    \tbf{Lemma.L.10}表明
    \[f(1)e_1,\cdots,f(m)e_m\]
    也是符合上述条件的规范正交组.对于任意$k\in\{1,\cdots,m\}$,$f(k)\in\{-1,1\}$,故而一共有$2^m$种这样的$f$.\\
    这就表明一共有$2^m$种这样的规范正交组符合题意.
\end{proof}
\begin{problem}[15.]
    设${\inprod\cdot\cdot}_1$和${\inprod\cdot\cdot}_2$是$V$上的内积,使得${\inprod uv}_1=0$当且仅当${\inprod uv}_2=0$.试证明:存在$c>0$使得${\inprod uv}_1=c{\inprod uv}_2$对任意$u,v\in V$成立.
\end{problem}
\begin{proof}
    如果$V=\{\mbf0\}$,那么我们可以取任意的$c>0$使命题成立.\\
    如果$V\neq\{\mbf0\}$,对于任意$v\in V$,定义$c_v=\dfrac{{\inprod vv}_1}{{\inprod vv}_2}$.\\
    根据正交分解,对于任意非零的$u,v\in V$,有
    \[{\inprod{u-\dfrac{{\inprod{u}{v}}_2}{{\inprod{v}{v}}_2}v}{v}}_2=0\]
    于是
    \[{\inprod{u-\dfrac{{\inprod{u}{v}}_2}{{\inprod{v}{v}}_2}v}{v}}_1=0\]
    即
    \[{\inprod uv}_1-\dfrac{{\inprod{u}{v}}_2}{{\inprod{v}{v}}_2}{\inprod vv}_2=0\]
    即
    \[{\inprod uv}_1=c_v{\inprod uv}_2\]
    交换$u,v$,我们可以得到${\inprod vu}_1=c_u{\inprod vu}_2$.于是
    \[c_v{\inprod uv}_2={\inprod uv}_1=\overline{{\inprod vu}_1}=\overline{c_u{\inprod vu}_2}=c_u{\inprod uv}_2\]
    于是,对于任意非零的$u,v\in V$有
    \[{\inprod uv}_1=c_u{\inprod uv}_2=c_v{\inprod uv}_2\]
    现在,我们考虑任意的$u,v\in V$,取$w\in V$使得${\inprod wu}_2\neq0$且${\inprod vw}_2\neq0$.于是
    \[c_w{\inprod wu}_2=c_u{\inprod wu}_2\text{\ \ \ \ \ 且\ \ \ \ \ }c_v{\inprod vw}_2=c_w{\inprod vw}_2\]
    于是上式表明$c_v=c_u=c_w$.这表示对于任意非零的$u,v\in V$,$c_u=c_v$.\\
    于是令$c=c_v$对于任意非零的$v\in V$成立.这就使得
    \[{\inprod uv}_1=c{\inprod uv}_2\]
    对所有$u,v\in V$都成立.
\end{proof}
\begin{problem}[16.]
    设$V$是有限维的,设${\inprod\cdot\cdot}_1,{\inprod\cdot\cdot}_2$都是$V$上的内积,与之关联的范数为$||\cdot||_1$和$||\cdot||_2$.试证明:存在$c>0$使得$||v||_1\leqslant c||v||_2$对任意$v\in V$成立.
\end{problem}
\begin{proof}
    令$\li e,n$为$V$关于${\inprod\cdot\cdot}_2$的规范正交基.对于任意$v:=a_1e_1+\cdots+a_ne_n\in V$,有
    \[|a_1|+\cdots+|a_n|\leqslant n\max\{|a_1|,\cdots,|a_n|\}\leqslant n\sqrt{|a_1|^2+\cdots+|a_n|^2}=n||v||_2\]
    不妨令$M=\max\left\{||e_1||_1,\cdots,||e_n||_1\right\}$.于是根据三角不等式有
    \[||v||_1\leqslant |a_1|||e_1||_1+\cdots+|a_n|||e_n||_1\leqslant M\left(|a_1|+\cdots+|a_n|\right)\leqslant nM||v||_2\]
    于是取$c=nM$即可使命题成立.
\end{proof}
\begin{problem}[17.]
    设$V$是有限维复向量空间.试证明:如果$T\in\L(V)$使得其唯一特征值为$1$且对任意$v\in V$都有$||Tv||\leqslant||v||$,那么$T$是恒等算子.
\end{problem}
\begin{proof}
    根据Schur定理,$T$关于$V$的某个规范正交基$\li e,n$具有上三角矩阵,不妨记为$A$.\\
    由于$T$有唯一特征值$1$,于是$A_{1,1}=\cdots=A_{n,n}=1$.\\
    对于任意$k\in\{1,\cdots,n\}$我们有
    \[||Te_k||
    =\left|\left|\sum_{j=1}^{k}A_{j,k}e_j\right|\right|
    =\sum_{j=1}^{k}\left|A_{j,k}\right|^2
    =1+\sum_{j=1}^{k-1}\left|A_{j,k}\right|^2\]
    由于$||Te_k||\leqslant||e_k||=1$,于是
    \[\sum_{j=1}^{k-1}\left|A_{j,k}\right|=0\]
    于是$A_{1,k}=\cdots=A_{k-1,k}=0$.这表明$A$是对角矩阵,且对角线上元素均为$1$.\\
    从而$A$是恒等矩阵,因此$T$是恒等算子.
\end{proof}
\begin{problem}[18.]
    设$\li u,m$是$V$中一线性无关组,试证明:存在$v\in V$使得$\inprod{u_k}{v}=1$对任意$k\in\{1,\cdots,m\}$成立.
\end{problem}
\begin{proof}
    令$U=\span\left(\li u,m\right)$,令$\li\phi,m$为其对偶基.\\
    令$\phi=\li\phi+m$,于是对于任意$k\in\{1,\cdots,m\}$都有$\phi\left(u_k\right)=1$.\\
    根据Riesz表示定理,存在$v\in U\subseteq V$使得对任意$u\in U$都有$\phi(u)=\inprod uv$.\\
    于是命题成立.
\end{proof}
\begin{problem}[19.]
    设$\li v,n$是$V$的基.试证明:存在$V$的一个基$\li u,n$使得
    \[\inprod{v_j}{u_k}=\left\{\begin{array}{l}
        0,j\neq k\\
        1,j=k
    \end{array}\right.\]
\end{problem}
\begin{proof}
    令$\li\phi,n$为$\li v,n$的对偶基.根据Riesz表示定理,存在$u_k\in V$使得
    \[\phi_k(v)=\inprod{v}{u_k}\]
    于是$\inprod{v_k}{u_k}=\phi_k(v_k)=1$.当$j\neq k$时$\inprod{v_j}{u_k}=\phi_k(v_j)=0$.\\
    我们只需证明$\li u,n$线性无关即可.为此,设$a_1u_1+\cdots+a_nu_n=\mbf0$.\\
    对于任意$k\in\{1,\cdots,n\}$,都有
    \[0=\inprod{v_k}{a_1u_1+\cdots+a_nu_n}=\sum_{j=1}^{n}a_j\phi_j(v_k)=a_k\]
    于是$\li a=n=0$,从而$\li u,n$线性无关,是$V$的一组基.
\end{proof}
\begin{problem}[20.]
    设$V$是有限维复向量空间,且$\mathcal{E}\subseteq\L(V)$使得$ST=TS$对所有$S,T\in\mathcal{E}$成立.试证明:存在$V$的一组规范正交基,使得$\mathcal{E}$中的每个元素关于其有上三角矩阵.
\end{problem}
\begin{proof}
    根据\tbf{5E.9}可知,存在$V$的一组基$\li v,n$使得$\mathcal{E}$中的每个元素关于其有上三角矩阵.\\
    对这组基应用Gram-Schmidt过程得到$V$的规范正交基$\li e,n$,满足对任意$k\in\{1,\cdots,n\}$有
    \[\span(\li v,k)=\span(\li e,k)\]
    从而$\mathcal{E}$中的各元素仍关于$\li e,n$有上三角矩阵.
\end{proof}
\begin{problem}[21.]
    设$V$是有限维复向量空间,$T\in\L(V)$且其特征值绝对值都小于$1$.试证明:对于任意$\ep>0$,存在$m\in\N$使得$||T^mv||\leqslant\ep||v||$对任意$v\in V$成立.
\end{problem}
\begin{proof}
    根据Schur定理,存在$V$的一组规范正交基$\li e,n$使得$T$关于其有上三角矩阵.
\end{proof}
\begin{problem}[22.]
    设$C[-1,1]$是区间$[-1,1]$上的连续实值函数构成的向量空间,其内积定义为
    \[\inprod fg=\displaystyle\int_{-1}^{1}fg\]
    令$\phi$为$C[-1,1]$上的线性泛函,定义为$\phi(f)=f(0)$.试证明:不存在$g\in C[-1,1]$使得
    \[\phi(f)=\inprod fg\]
    对任意$f\in C[-1,1]$成立.
\end{problem}
\begin{proof}
    我们假定这样的$g$存在,那么令$h(x)=x^2g(x)\in C[-1,1]$,则有
    \[0=h(0)=\phi(h)=\inprod hg=\int_{-1}^{1}x^2\left(g(x)\right)^2\di x\]
    当且仅当$g(x)=0$时上式成立.再定义$f(x)=1$,于是有
    \[\phi(f)=f(0)=1\neq0=\inprod fg\]
    于是产生矛盾,因而这样的$g$不存在.
\end{proof}
\end{document}