\documentclass{ctexart}
\usepackage{geometry}
\usepackage[dvipsnames,svgnames]{xcolor}
\usepackage[strict]{changepage}
\usepackage{framed}
\usepackage{enumerate}
\usepackage{amsmath,amsthm,amssymb}
\usepackage{enumitem}
\usepackage{solution}

\allowdisplaybreaks
\geometry{left=2cm, right=2cm, top=2.5cm, bottom=2.5cm}

\begin{document}\pagestyle{empty}
\begin{center}
    \large\tbf{Linear Algebra Done Right 5D}
\end{center}
\begin{problem}[1.]
    设$V$是有限维复向量空间,且$T\in\L(V)$.
    \begin{enumerate}[label=\tbf{(\arabic*)}]
        \item 证明:如果$T^4=I$,那么$T$可对角化.
        \item 证明:如果$T^4=T$,那么$T$可对角化.
        \item 给出一例$T\in\L(\C^2)$,使得$T^4=T^2$且$T$不可对角化.
    \end{enumerate}
\end{problem}
\begin{proof}
    \begin{enumerate}[label=\tbf{(\arabic*)}]
        \item 因为$T^4=I$,于是存在$p(z)=z^4-1=(z+1)(z-1)(z+\i)(z-\i)$使得$p(T)=\mbf0$.\\
            于是$p$是$T$的最小多项式$q$的多项式倍,因而$q$也具有$(z-\lambda_1)\cdots(z-\lambda_m)$的形式,其中各$\lambda$互异.\\
            因而$T$是可对角化的.
        \item 因为$T^4=T$,于是存在$p(z)=z^4-z=z(z-1)\left(z-\dfrac{1+\sqrt{3}\i}{2}\right)\left(z-\dfrac{1-\sqrt3\i}{2}\right)$使得$p(T)=\mbf0$.\\
            于是$p$是$T$的最小多项式$q$的多项式倍,因而$q$也具有$(z-\lambda_1)\cdots(z-\lambda_m)$的形式,其中各$\lambda$互异.\\
            因而$T$是可对角化的.
        \item 令$T(x,y)=(y,0)$,于是$T$的最小多项式为$p(z)=z^2$,满足$T^4-T^2=\mbf0$,且不可对角化($p(z)$有重根).
    \end{enumerate}
\end{proof}
\begin{problem}[2.]
    设$T\in\L(V)$关于$V$的一个基有对角矩阵$A$.试证明:若$\lambda\in\F$,那么$\lambda$在$A$的对角线上恰好出现$\dim E(\lambda,T)$次.
\end{problem}
\begin{proof}
    如果$\lambda$不是$T$的特征值,则$\lambda$不会出现在$A$的对角线上.而$E(\lambda,T)=\nul(T-\lambda I)=\left\{\mbf0\right\}$,于是命题成立.\\
    如果$\lambda$是$T$的特征值,考虑$A$对应的一组基$\li v,n$(其中$n=\dim V$)和对角线上的元素$\li\lambda,n$.我们有
    \[Tv_k=\lambda v_k,\forall k\in\left\{1,\cdots,n\right\}\]
    当且仅当$\lambda_k=\lambda$时有$(T-\lambda I)v_k=\mbf0$.这样的$v_{k_1},\cdots,v_{k_i}$构成了$E(\lambda,T)$的基,恰好对应$i$个$\lambda_k$.即$\lambda$恰好在$A$的对角线上出现$\dim E(\lambda,T)$次.于是命题成立.
\end{proof}
\begin{problem}[3.]
    设$V$是有限维的,且$T\in\L(V)$.试证明:如果$T$可对角化,那么$V=\nul T\oplus\range T$.
\end{problem}
\begin{proof}
    令$\li\lambda,m$表示$T$的非零互异特征值.于是有
    \[V=E(0,T)\oplus E(\lambda_1,T)\oplus\cdots\oplus E(\lambda_m,T)\]
    其中$E(0,T)=\nul T$.如果$E(0,T)=\{\mbf0\}$,那么$\range T=V$.\\
    如果$T$没有非零特征值,那么$\nul T=E(0,T)=V$.\\
    否则,令$W=E(\lambda_1,T)\oplus\cdots\oplus E(\lambda_m,T)$,则有$T=\nul T\oplus W$.我们只需证明$W=\range T$.\\
    对于任意$v\in V$,令$v=u+\li w+m\in\nul T\oplus W$,其中$u\in\nul T,w_k\in E(\lambda_k,T)$.于是
    \[Tv=Tu+\li{Tw}+m=\lambda_1w_1+\cdots+\lambda_mw_m\in W\]
    于是$\range T\subseteq W$.又对于任意$w:=w_1+\cdots+w_m\in W$有
    \[\li w+m=T\left(\dfrac{w_1}{\lambda_1}+\cdots+\dfrac{w_m}{\lambda_m}\right)\in\range T\]
    于是$W\subseteq\range T$.综上可知$W=\range T$,即$V=\nul T\oplus\range T$.
\end{proof}
\begin{problem}[4.]
    设$V$是有限维的,且$T\in\L(V)$.证明下列三个命题相互等价.
    \begin{enumerate}[label=\tbf{(\alph*)}]
        \item $V=\nul T\oplus\range T$.
        \item $V=\nul T+\range T$.
        \item $\nul T\cap\range T=\{\mbf0\}$.
    \end{enumerate}
\end{problem}
\begin{proof}
    \tbf{(a)}$\Rightarrow$\tbf{(b)}:显然.\\
    \tbf{(b)}$\Rightarrow$\tbf{(c)}:我们有
    \[\dim(\nul T\cap\range T)=\dim\nul T+\dim\range T-\dim(\nul T+\range T)\]
    根据线性映射基本定理,我们有
    \[\dim\nul T+\dim\range T=\dim V\]
    根据假设又有
    \[\dim(\nul T+\range T)=\dim V\]
    于是$\dim(\nul T\cap\range T)=0$,即$\nul T\cap\range T=\{\mbf0\}$,于是\tbf{(c)}成立.\\
    \tbf{(c)}$\Rightarrow$\tbf{(a)}:我们有
    \[\dim(\nul T+\range T)=\dim\nul T+\dim\range T-\dim(\nul T\cap\range T)=\dim V\]
    又因为$\nul T$与$\range T$是$T$的子空间,于是$V=\nul T+\range T$.\\
    又因为$\nul T\cap\range T=\{\mbf0\}$,于是$\nul T+\range T$是直和.于是$V=\nul T\oplus\range T$.
\end{proof}
\begin{problem}[5.]
    设$T$是有限维复向量空间,且$T\in\L(V)$.试证明:$T$可以对角化,当且仅当
    \[V=\nul(T-\lambda I)\oplus\range(T-\lambda I)\]
    对任意$\lambda\in\C$均成立.
\end{problem}
\begin{proof}
    $\Rightarrow$:若$T$可以对角化,不妨设$T$关于$V$的某组基的对角矩阵上的元素为$\li\lambda,n$.\\
    于是$T-\lambda I$也是对角矩阵,其对角线上的元素为$\lambda_1-\lambda,\cdots,\lambda_n-\lambda$.\\
    因此,对于任意$\lambda\in\C$,$T-\lambda I$都是可对角化算子.\\
    于是根据\tbf{5D.3},$V=\nul(T-\lambda I)\oplus\range(T-\lambda I)$对任意$\lambda\in\C$都成立.\\
    $\Leftarrow$:假设$T$不可以对角化,那么$T$的最小多项式$p$应当具有重根.不妨设$p=(z-\lambda)q$,其中$q\in\P(\C)$且$q(\lambda)=0$.\\
    对于任意$v\in V$,有
    \[p(T)v=\mbf0\Leftrightarrow (T-\lambda I)q(T)v=\mbf0\Leftrightarrow q(T)v\in\nul(T-\lambda I)\]
    \[q(\lambda)=0\Leftrightarrow\exists r\in\P(\C)\st q(z)=(z-\lambda)r(z)\Leftrightarrow q(T)v=(T-\lambda I)(r(T)v)\Rightarrow q(T)v\in\range(T-\lambda I)\]
    于是$q(T)v\in\nul(T-\lambda I)\cap\range(T-\lambda I)$.\\
    由于$p$是$T$的最小多项式,且$\deg q<\deg p$,因而$q(T)\neq\mbf0$.于是存在$v\in V$使得$q(T)v\neq\mbf0$.\\
    即$\nul(T-\lambda I)\cap\range(T-\lambda I)\neq\{\mbf0\}$,因而根据\tbf{5D.4}可知两者不是直和,与条件矛盾.于是$T$可以对角化.
\end{proof}
\begin{problem}[6.]
    设$T\in\L(\F^5)$且$\dim E(8,T)=4$.试证明$T-2I$或$T-6I$可逆.
\end{problem}
\begin{proof}
    假定$T-2I$和$T-6I$均不可逆,那么$2$和$6$均为$T$的特征值.于是
    \[\dim E(2,T)\geqslant 1,\dim E(6,T)\geqslant 1\]
    又因为$\dim E(8,T)>0$,于是$8$也为$T$的特征值.
    于是
    \[\dim V\geqslant\dim E(2,T)+\dim E(6,T)+\dim E(8,T)\geqslant 1+1+4=6\]
    而$\dim V=5$,于是推出矛盾.因而$T-2I$和$T-6I$至少有一个可逆.
\end{proof}
\begin{problem}[7.]
    设$T\in\L(V)$可逆,试证明
    \[E(\lambda,T)=E\left(\dfrac{1}{\lambda},T^{-1}\right)\]
    对任意$\lambda\in\F(\lambda\neq0)$都成立.
\end{problem}
\begin{proof}
    对于任意$v\in V$都有
    \[v\in E(\lambda,T)\Leftrightarrow Tv=\lambda v\Leftrightarrow v=\lambda T^{-1}v\Leftrightarrow T^{-1}v=\dfrac{1}{\lambda}v\Leftrightarrow v\in E\left(\dfrac{1}{\lambda},T^{-1}\right)\]
    于是$E(\lambda,T)=E\left(\dfrac{1}{\lambda},T^{-1}\right)$.
\end{proof}
\begin{problem}[8.]
    设$V$是有限维的,且$T\in\L(V)$.令$\li\lambda,m$表示$T$的非零互异特征值,试证明:
    \[\dim E(\lambda_1,T)+\cdots+\dim E(\lambda_m,T)\leqslant\dim\range T\]
\end{problem}
\begin{proof}
    我们有
    \[\dim E(0,T)+\dim E(\lambda_1,T)+\cdots+\dim E(\lambda_m,T)\leqslant\dim V\]
    而$E(0,T)=\nul T$,于是
    \[\dim E(\lambda_1,T)+\cdots+\dim E(\lambda_m,T)\leqslant\dim V-\dim E(0,T)=\dim V-\dim\nul T=\dim\range T\]
    于是命题成立.
\end{proof}
\begin{problem}[9.]
    设$R,T\in\L(\F^3)$,都有特征值$2,6,7$.证明:存在一可逆算子$S\in\L(\F^3)$使得$R=S^{-1}TS$.
\end{problem}
\begin{proof}
    设$R$关于$2,6,7$的特征向量为$u_1,u_2,u_3$,$T$关于$2,6,7$的特征向量为$v_1,v_2,v_3$.\\
    于是$u_1,u_2,u_3$和$v_1,v_2,v_3$是$\F^3$的基.令$Su_k=v_k$,则对于任意$v:=a_1u_1+a_2u_2+a_3u_3\in\F^3$有
    \[S^{-1}Sv=S^{-1}T(a_1v_1+a_2v_2+a_3v_3)=S^{-1}(2a_1v_1+6a_2v_2+7a_3v_3)=2a_1u_1+6a_2u_2+7a_3u_3=Rv\]
    于是$R=S^{-1}TS$.
\end{proof}
\begin{problem}[10.]
    
\end{problem}
\end{document}