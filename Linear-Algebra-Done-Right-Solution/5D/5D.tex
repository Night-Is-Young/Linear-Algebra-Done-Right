\documentclass{ctexart}
\usepackage{geometry}
\usepackage[dvipsnames,svgnames]{xcolor}
\usepackage[strict]{changepage}
\usepackage{framed}
\usepackage{enumerate}
\usepackage{amsmath,amsthm,amssymb}
\usepackage{enumitem}
\usepackage{solution}

\allowdisplaybreaks
\geometry{left=2cm, right=2cm, top=2.5cm, bottom=2.5cm}

\begin{document}\pagestyle{empty}
\begin{center}
    \large\tbf{Linear Algebra Done Right 5D}
\end{center}
\begin{problem}[1.]
    设$V$是有限维复向量空间,且$T\in\L(V)$.
    \begin{enumerate}[label=\tbf{(\arabic*)}]
        \item 证明:如果$T^4=I$,那么$T$可对角化.
        \item 证明:如果$T^4=T$,那么$T$可对角化.
        \item 给出一例$T\in\L(\C^2)$,使得$T^4=T^2$且$T$不可对角化.
    \end{enumerate}
\end{problem}
\begin{proof}
    \begin{enumerate}[label=\tbf{(\arabic*)}]
        \item 因为$T^4=I$,于是存在$p(z)=z^4-1=(z+1)(z-1)(z+\i)(z-\i)$使得$p(T)=\mbf0$.\\
            于是$p$是$T$的最小多项式$q$的多项式倍,因而$q$也具有$(z-\lambda_1)\cdots(z-\lambda_m)$的形式,其中各$\lambda$互异.\\
            因而$T$是可对角化的.
        \item 因为$T^4=T$,于是存在$p(z)=z^4-z=z(z-1)(z+\i)(z-\i)$使得$p(T)=\mbf0$.\\
            于是$p$是$T$的最小多项式$q$的多项式倍,因而$q$也具有$(z-\lambda_1)\cdots(z-\lambda_m)$的形式,其中各$\lambda$互异.\\
            因而$T$是可对角化的.
    \end{enumerate}
\end{proof}
\end{document}