\documentclass{ctexart}
\usepackage{geometry}
\usepackage[dvipsnames,svgnames]{xcolor}
\usepackage[strict]{changepage}
\usepackage{framed}
\usepackage{enumerate}
\usepackage{amsmath,amsthm,amssymb}
\usepackage{enumitem}
\usepackage{solution}

\allowdisplaybreaks
\geometry{left=2cm, right=2cm, top=2.5cm, bottom=2.5cm}

\begin{document}\pagestyle{empty}
\begin{center}
    \large\tbf{Linear Algebra Done Right 5D}
\end{center}
\begin{problem}[1.]
    设$V$是有限维复向量空间,且$T\in\L(V)$.
    \begin{enumerate}[label=\tbf{(\arabic*)}]
        \item 证明:如果$T^4=I$,那么$T$可对角化.
        \item 证明:如果$T^4=T$,那么$T$可对角化.
        \item 给出一例$T\in\L(\C^2)$,使得$T^4=T^2$且$T$不可对角化.
    \end{enumerate}
\end{problem}
\begin{proof}
    \begin{enumerate}[label=\tbf{(\arabic*)}]
        \item 因为$T^4=I$,于是存在$p(z)=z^4-1=(z+1)(z-1)(z+\i)(z-\i)$使得$p(T)=\mbf0$.\\
            于是$p$是$T$的最小多项式$q$的多项式倍,因而$q$也具有$(z-\lambda_1)\cdots(z-\lambda_m)$的形式,其中各$\lambda$互异.\\
            因而$T$是可对角化的.
        \item 因为$T^4=T$,于是存在$p(z)=z^4-z=z(z-1)\left(z-\dfrac{1+\sqrt{3}\i}{2}\right)\left(z-\dfrac{1-\sqrt3\i}{2}\right)$使得$p(T)=\mbf0$.\\
            于是$p$是$T$的最小多项式$q$的多项式倍,因而$q$也具有$(z-\lambda_1)\cdots(z-\lambda_m)$的形式,其中各$\lambda$互异.\\
            因而$T$是可对角化的.
        \item 令$T(x,y)=(y,0)$,于是$T$的最小多项式为$p(z)=z^2$,满足$T^4-T^2=\mbf0$,且不可对角化($p(z)$有重根).
    \end{enumerate}
\end{proof}
\begin{problem}[2.]
    设$T\in\L(V)$关于$V$的一个基有对角矩阵$A$.试证明:若$\lambda\in\F$,那么$\lambda$在$A$的对角线上恰好出现$\dim E(\lambda,T)$次.
\end{problem}
\begin{proof}
    如果$\lambda$不是$T$的特征值,则$\lambda$不会出现在$A$的对角线上.而$E(\lambda,T)=\nul(T-\lambda I)=\left\{\mbf0\right\}$,于是命题成立.\\
    如果$\lambda$是$T$的特征值,考虑$A$对应的一组基$\li v,n$(其中$n=\dim V$)和对角线上的元素$\li\lambda,n$.我们有
    \[Tv_k=\lambda v_k,\forall k\in\left\{1,\cdots,n\right\}\]
    当且仅当$\lambda_k=\lambda$时有$(T-\lambda I)v_k=\mbf0$.这样的$v_{k_1},\cdots,v_{k_i}$构成了$E(\lambda,T)$的基,恰好对应$i$个$\lambda_k$.即$\lambda$恰好在$A$的对角线上出现$\dim E(\lambda,T)$次.于是命题成立.
\end{proof}
\begin{problem}[3.]
    设$V$是有限维的,且$T\in\L(V)$.试证明:如果$T$可对角化,那么$V=\nul T\oplus\range T$.
\end{problem}
\begin{proof}
    令$\li\lambda,m$表示$T$的非零互异特征值.于是有
    \[V=E(0,T)\oplus E(\lambda_1,T)\oplus\cdots\oplus E(\lambda_m,T)\]
    其中$E(0,T)=\nul T$.如果$E(0,T)=\{\mbf0\}$,那么$\range T=V$.\\
    如果$T$没有非零特征值,那么$\nul T=E(0,T)=V$.\\
    否则,令$W=E(\lambda_1,T)\oplus\cdots\oplus E(\lambda_m,T)$,则有$T=\nul T\oplus W$.我们只需证明$W=\range T$.\\
    对于任意$v\in V$,令$v=u+\li w+m\in\nul T\oplus W$,其中$u\in\nul T,w_k\in E(\lambda_k,T)$.于是
    \[Tv=Tu+\li{Tw}+m=\lambda_1w_1+\cdots+\lambda_mw_m\in W\]
    于是$\range T\subseteq W$.又对于任意$w:=w_1+\cdots+w_m\in W$有
    \[\li w+m=T\left(\dfrac{w_1}{\lambda_1}+\cdots+\dfrac{w_m}{\lambda_m}\right)\in\range T\]
    于是$W\subseteq\range T$.综上可知$W=\range T$,即$V=\nul T\oplus\range T$.
\end{proof}
\begin{problem}[4.]
    设$V$是有限维的,且$T\in\L(V)$.证明下列三个命题相互等价.
    \begin{enumerate}[label=\tbf{(\alph*)}]
        \item $V=\nul T\oplus\range T$.
        \item $V=\nul T+\range T$.
        \item $\nul T\cap\range T=\{\mbf0\}$.
    \end{enumerate}
\end{problem}
\begin{proof}
    \tbf{(a)}$\Rightarrow$\tbf{(b)}:显然.\\
    \tbf{(b)}$\Rightarrow$\tbf{(c)}:我们有
    \[\dim(\nul T\cap\range T)=\dim\nul T+\dim\range T-\dim(\nul T+\range T)\]
    根据线性映射基本定理,我们有
    \[\dim\nul T+\dim\range T=\dim V\]
    根据假设又有
    \[\dim(\nul T+\range T)=\dim V\]
    于是$\dim(\nul T\cap\range T)=0$,即$\nul T\cap\range T=\{\mbf0\}$,于是\tbf{(c)}成立.\\
    \tbf{(c)}$\Rightarrow$\tbf{(a)}:我们有
    \[\dim(\nul T+\range T)=\dim\nul T+\dim\range T-\dim(\nul T\cap\range T)=\dim V\]
    又因为$\nul T$与$\range T$是$T$的子空间,于是$V=\nul T+\range T$.\\
    又因为$\nul T\cap\range T=\{\mbf0\}$,于是$\nul T+\range T$是直和.于是$V=\nul T\oplus\range T$.
\end{proof}
\begin{problem}[5.]
    设$T$是有限维复向量空间,且$T\in\L(V)$.试证明:$T$可以对角化,当且仅当
    \[V=\nul(T-\lambda I)\oplus\range(T-\lambda I)\]
    对任意$\lambda\in\C$均成立.
\end{problem}
\begin{proof}
    $\Rightarrow$:若$T$可以对角化,不妨设$T$关于$V$的某组基的对角矩阵上的元素为$\li\lambda,n$.\\
    于是$T-\lambda I$也是对角矩阵,其对角线上的元素为$\lambda_1-\lambda,\cdots,\lambda_n-\lambda$.\\
    因此,对于任意$\lambda\in\C$,$T-\lambda I$都是可对角化算子.\\
    于是根据\tbf{5D.3},$V=\nul(T-\lambda I)\oplus\range(T-\lambda I)$对任意$\lambda\in\C$都成立.\\
    $\Leftarrow$:假设$T$不可以对角化,那么$T$的最小多项式$p$应当具有重根.不妨设$p=(z-\lambda)q$,其中$q\in\P(\C)$且$q(\lambda)=0$.\\
    对于任意$v\in V$,有
    \[p(T)v=\mbf0\Leftrightarrow (T-\lambda I)q(T)v=\mbf0\Leftrightarrow q(T)v\in\nul(T-\lambda I)\]
    \[q(\lambda)=0\Leftrightarrow\exists r\in\P(\C)\st q(z)=(z-\lambda)r(z)\Leftrightarrow q(T)v=(T-\lambda I)(r(T)v)\Rightarrow q(T)v\in\range(T-\lambda I)\]
    于是$q(T)v\in\nul(T-\lambda I)\cap\range(T-\lambda I)$.\\
    由于$p$是$T$的最小多项式,且$\deg q<\deg p$,因而$q(T)\neq\mbf0$.于是存在$v\in V$使得$q(T)v\neq\mbf0$.\\
    即$\nul(T-\lambda I)\cap\range(T-\lambda I)\neq\{\mbf0\}$,因而根据\tbf{5D.4}可知两者不是直和,与条件矛盾.于是$T$可以对角化.
\end{proof}
\begin{problem}[6.]
    设$T\in\L(\F^5)$且$\dim E(8,T)=4$.试证明$T-2I$或$T-6I$可逆.
\end{problem}
\begin{proof}
    假定$T-2I$和$T-6I$均不可逆,那么$2$和$6$均为$T$的特征值.于是
    \[\dim E(2,T)\geqslant 1,\dim E(6,T)\geqslant 1\]
    又因为$\dim E(8,T)>0$,于是$8$也为$T$的特征值.
    于是
    \[\dim V\geqslant\dim E(2,T)+\dim E(6,T)+\dim E(8,T)\geqslant 1+1+4=6\]
    而$\dim V=5$,于是推出矛盾.因而$T-2I$和$T-6I$至少有一个可逆.
\end{proof}
\begin{problem}[7.]
    设$T\in\L(V)$可逆,试证明
    \[E(\lambda,T)=E\left(\dfrac{1}{\lambda},T^{-1}\right)\]
    对任意$\lambda\in\F(\lambda\neq0)$都成立.
\end{problem}
\begin{proof}
    对于任意$v\in V$都有
    \[v\in E(\lambda,T)\Leftrightarrow Tv=\lambda v\Leftrightarrow v=\lambda T^{-1}v\Leftrightarrow T^{-1}v=\dfrac{1}{\lambda}v\Leftrightarrow v\in E\left(\dfrac{1}{\lambda},T^{-1}\right)\]
    于是$E(\lambda,T)=E\left(\dfrac{1}{\lambda},T^{-1}\right)$.
\end{proof}
\begin{problem}[8.]
    设$V$是有限维的,且$T\in\L(V)$.令$\li\lambda,m$表示$T$的非零互异特征值,试证明:
    \[\dim E(\lambda_1,T)+\cdots+\dim E(\lambda_m,T)\leqslant\dim\range T\]
\end{problem}
\begin{proof}
    我们有
    \[\dim E(0,T)+\dim E(\lambda_1,T)+\cdots+\dim E(\lambda_m,T)\leqslant\dim V\]
    而$E(0,T)=\nul T$,于是
    \[\dim E(\lambda_1,T)+\cdots+\dim E(\lambda_m,T)\leqslant\dim V-\dim E(0,T)=\dim V-\dim\nul T=\dim\range T\]
    于是命题成立.
\end{proof}
\begin{problem}[9.]
    设$R,T\in\L(\F^3)$,都有特征值$2,6,7$.证明:存在一可逆算子$S\in\L(\F^3)$使得$R=S^{-1}TS$.
\end{problem}
\begin{proof}
    设$R$关于$2,6,7$的特征向量为$u_1,u_2,u_3$,$T$关于$2,6,7$的特征向量为$v_1,v_2,v_3$.\\
    于是$u_1,u_2,u_3$和$v_1,v_2,v_3$是$\F^3$的基.令$Su_k=v_k$,则对于任意$v:=a_1u_1+a_2u_2+a_3u_3\in\F^3$有
    \[S^{-1}Sv=S^{-1}T(a_1v_1+a_2v_2+a_3v_3)=S^{-1}(2a_1v_1+6a_2v_2+7a_3v_3)=2a_1u_1+6a_2u_2+7a_3u_3=Rv\]
    于是$R=S^{-1}TS$.
\end{proof}
\begin{problem}[10.]
    给出一例$R,T\in\L(\F^4)$都有且仅有特征值$2,6,7$,同时不存在可逆算子$S\in\L(\F^4)$使得$R=S^{-1}TS$.
\end{problem}
\begin{solution}
    设$R,T$对应$\F^4$的标准基$\li e,4$的矩阵为
    \[\mathcal{M}(R)=\begin{pmatrix}
        2&0&0&0\\0&6&0&0\\0&0&6&0\\0&0&0&7
    \end{pmatrix}\ \ \ \ \ 
    \mathcal{M}(T)=\begin{pmatrix}
        2&0&0&0\\0&2&0&0\\0&0&6&0\\0&0&0&7
    \end{pmatrix}\]
    假定存在$S\in\L(\F^4)$使得$R=S^{-1}TS$.设$v_1=S^{-1}e_1,v_2=S^{-1}e_2$,于是
    \[S^{-1}TSv_1=S^{-1}Te_1=2S^{-1}e_1=2v_1\]
    同理有
    \[S^{-1}TSv_2=2v_2\]
    若$v_1,v_2$线性相关,则存在$\lambda\in\F$使得$v_2=\lambda v_1$.又因为$S$可逆,于是$\nul S^{-1}=\{\mbf0\}$,于是
    \[v_2=\lambda v_1\Leftrightarrow S^{-1}e_2=\lambda S^{-1}e_1 \Leftrightarrow S^{-1}\left(e_2-\lambda e_1\right)=\mbf0\Leftrightarrow e_2-\lambda e_1=\mbf0\]
    于是$e_1,e_2$线性相关,这与$\li e,4$是$\F^4$的标准基不符.于是$\dim\span(v_1,v_2)=2$.\\
    即$\dim E(2,S^{-1}TS)=\span(v_1,v_2)=2$.而$\dim E(2,R)=1$,于是一定有$R\neq S^{-1}TS$.\\
    于是不存在可逆的$S\in\L(\F^4)$使得$R=S^{-1}TS$.
\end{solution}
\begin{problem}[11.]
    给出一例$T\in\L(\C^3)$使得$6$和$7$是$T$的特征值,且$T$不可对角化.
\end{problem}
\begin{solution}
    设$T$关于$\C^3$的标准基$\li e,3$的矩阵为
    \[\mathcal{M}(T)=\begin{pmatrix}
        6&0&7\\0&7&0\\0&0&6
    \end{pmatrix}\]
    这是一个上三角矩阵,因而$6$和$7$是$T$的特征值.然而
    \[\mathcal{M}(T-6I)=\begin{pmatrix}
        0&0&7\\0&1&0\\0&0&0
    \end{pmatrix}\ \ \ \ \ 
    \mathcal{M}(T-7I)=\begin{pmatrix}
        -1&0&7\\0&0&0\\0&0&-1
    \end{pmatrix}\]
    于是$\dim E(6,T)+\dim E(7,T)=1+1<\dim\C^3$,于是$T$不可对角化.
\end{solution}
\begin{problem}[12.]
    设$T\in\L(\C^3)$使得$6$和$7$是$T$的特征值,且$T$不可对角化.试证明:存在$(z_1,z_2,z_3)\in\C^3$使得
    \[T(z_1,z_2,z_3)=(6+8z_1,7+8z_2,13+8z_3)\]
\end{problem}
\begin{proof}
    由于$8$不是$T$的特征值,于是$T-8I$可逆,于是存在$(z_1,z_2,z_3)\in\C^3$使得
    \[(T-8I)(z_1,z_2,z_3)=(6,7,13)\]
    即
    \[T(z_1,z_2,z_3)=(6+8z_1,7+8z_2,13+8z_3)\]
    于是命题得证.
\end{proof}
\begin{problem}[13.]
    设$A$是对角线上元素互异的对角矩阵,$B$是与$A$大小相同的方阵.试证明:$AB=BA$当且仅当$B$是对角矩阵.
\end{problem}
\begin{proof}
    $\Rightarrow$:设$A$的对角线上元素为$\li\lambda,m$.若$B$不是对角矩阵,那么不妨假定存在$B_{i,j}\neq0$,其中$i\neq j$.于是
    \[(AB)_{i,j}=\sum_{k=1}^{m}A_{i,k}B_{k,j}=A_{i,i}B_{i,j}=\lambda_iB_{i,j}\]
    \[(BA)_{i,j}=\sum_{k=1}^{m}B_{i,k}A_{k,j}=B_{i,j}A_{j,j}=\lambda_jB_{i,j}\]
    因为$\lambda_i\neq\lambda_j$,于是$(AB)_{i,j}\neq(BA)_{i,j}$,即$AB\neq BA$.这与题设不符,于是$B$是对角矩阵.\\
    $\Leftarrow$:设$B$的对角线元素为$\li\mu,m$.于是
    \[(AB)_{i,j}=0=(BA)_{i,j}\]
    \[(AB)_{i,i}=\lambda_i\mu_i=(BA)_{i,i}\]
    即$AB=BA$.
\end{proof}
\begin{problem}[14.]
    回答下列问题.
    \begin{enumerate}[label=\tbf{(\arabic*)}]
        \item 给出一例有限维复向量空间$V$和$T\in\L(V)$使得$T^2$可对角化但$T$不可对角化.
        \item 设$\F=\C$,$k\in\N^*$,且$T\in\L(V)$可逆.试证明:$T$可对角化当且仅当$T^k$可对角化.
    \end{enumerate}
\end{problem}
\begin{solution}
    \begin{enumerate}[label=\tbf{(\arabic*)}]
        \item 设$V=\C^2,T(x,y)=(y,0)$.$T$关于$\C^2$的标准基的矩阵为
            \[\mathcal{M}(T)=\begin{pmatrix}
                0&1\\0&0
            \end{pmatrix}\]
            于是$T$有且仅有特征值$0$.而$\dim E(0,T)=1<\dim\C^2$,于是$T$不可对角化.\\
            然而$T^2=\mbf0$,于是$T^2$可对角化,其关于任意$\C^2$的基的矩阵都是元素均为$0$的对角矩阵.
        \item $\Rightarrow$:设$T$关于$V$的某组基具有对角矩阵
            \[\mathcal{M}(T)=\begin{pmatrix}
                \lambda_1&&0\\&\ddots&\\0&&\lambda_n
            \end{pmatrix}\]
            那么有
            \[\mathcal{M}(T^k)=\left(\mathcal{M}(T)\right)^k=\begin{pmatrix}
                \lambda_1^k&&0\\&\ddots&\\0&&\lambda_n^k
            \end{pmatrix}\]
            因而$T^k$关于$V$的这组基也具有对角矩阵,从而$T^k$可对角化.\\
            $\Leftarrow$:设$T^k$的最小多项式为$p(z)=(z-\lambda_1)\cdots(z-\lambda_m)$,则$\li\lambda,m$互异且非零(否则$T^k$不可逆).\\
            根据代数基本定理,$z^k-\lambda_j=0$有$k$重根,不妨记为$\mu_{j,1},\cdots,\mu_{j,k}$,这$k$个根各不相同.\\
            对于任意$a,b\in\{1,\cdots,m\},c,d\in\{1,\cdots,k\}$有
            \[\mu_{a,c}=\mu_{b,d}\Rightarrow \mu_{a,c}^k=\mu_{b,d}^k\Rightarrow \lambda_a=\lambda_b\]
            于是$\mu_{a,c}=\mu_{b,d}$当且仅当$a=b,c=d$.这表明$\mu_{1,1},\cdots,\mu_{m,k}$互异.令$q(z)=p\left(z^k\right)\in\P(\C)$,则
            \[q(z)=p\left(z^k\right)=\prod_{j=1}^{m}\left(z^k-\lambda_j\right)=\prod_{j=1}^{m}\prod_{i=1}^{k}\left(z-\mu_{j,i}\right)\]
            而$q(T)=p(T^k)=\mbf0$,于是$q$是$T$的最小多项式$r$的多项式倍.于是
            \[r(z)=(z-\xi_1)\cdots(z-\xi_n)\]
            其中各$\xi$均为$\mu_{1,1},\cdots,\mu_{m,k}$中的某个且互异.于是$T$是可对角化的.
    \end{enumerate}
\end{solution}
\begin{problem}[15.]
    设$V$是有限维复向量空间,$T\in\L(V)$,且$p$是$T$的最小多项式.证明下列命题相互等价.
    \begin{enumerate}[label=\tbf{(\alph*)}]
        \item $T$可对角化.
        \item 不存在$\lambda\in\C$使得$p$是$(z-\lambda)^2$的多项式倍.
        \item $p$和其导函数$p'$没有公共零点.
        \item $p$和其导函数$p'$的最大公因式是常数多项式$1$.
    \end{enumerate}
\end{problem}
\begin{proof}
    \tbf{(a)}$\Leftrightarrow$\tbf{(b)}:我们有
    \[T\text{可对角化}\Leftrightarrow p=(z-\lambda_1)\cdots(z-\lambda_m),\text{其中各}\lambda_k\text{不相同}\Leftrightarrow\text{不存在}\lambda\in\C\text{使}p\text{是}(z-\lambda)^2\text{的多项式倍}\]
    \tbf{(b)}$\Rightarrow$\tbf{(c)}:若存在$\lambda\in\C$使得$p(\lambda)=p'(\lambda)=0$,不妨令$p(z)=(z-\lambda)^kq(z)$,其中$k\geqslant1,q(\lambda)\neq0$.于是
    \[p'(z)=(z-\lambda)^kq'(z)+k(z-\lambda)^{k-1}q(z)=(z-\lambda)^{k-1}\left((z-\lambda)q'(z)+kq(z)\right)\]
    当$k=1$时,$p'(\lambda)=q(\lambda)\neq0$.\\
    当$k\geqslant2$时,$p'(\lambda)=0^{k-1}kq(\lambda)=0$.\\
    于是$k\geqslant2$.因而$p$是$(z-\lambda)^2$的$(z-\lambda)^{k-2}q$倍,与条件矛盾.因而$p$与$p'$没有公共零点.\\
    \tbf{(c)}$\Rightarrow$\tbf{(b)}:若存在$\lambda\in\C$使得$p=(z-\lambda)^2q$,其中$q\in\P(\C)$,那么
    \[p'(z)=2(z-\lambda)q(z)+(z-\lambda)^2q'(z)\]
    于是$p(\lambda)=p'(\lambda)=0$,与条件矛盾,因而不存在$\lambda\in\C$使得$p$是$(z-\lambda)^2$的多项式倍.\\
    \tbf{(c)}和\tbf{(d)}的等价性是显然的.
\end{proof}
\begin{problem}[16.]
    设$T\in\L(V)$可对角化,令$\li\lambda,m$表示$T$的互异特征值.证明:$V$的子空间$U$在$T$下不变,当且仅当存在$V$的子空间$\li U,m$使得$U_k\subseteq E(\lambda_k,T)$对每个$k$成立且$U=\li U\oplus m$.
\end{problem}
\begin{proof}
    $\Leftarrow$:假定存在这样的子空间$\li U,m$.对于任意$u\in U$,设$u=\li u+m$,其中$u_k\in U_k$对每个$k$都成立.\\
    那么我们有
    \[Tu=T(\li u+m)=\li{Tu}+m=\lambda_1u_1+\cdots+\lambda_mu_m\in\li U\oplus m\]
    于是$U=\li U\oplus m$在$T$下不变.\\
    $\Rightarrow$:令$U_k=U\cap E(\lambda_k,T)$对每个$k$成立,则$U_k\subseteq E(\lambda_k,T)$.\\
    由于$V=E(\lambda_1,T)\oplus\cdots\oplus E(\lambda_m,T)$,于是$\li U+m$是直和.\\
    对于任意$\li u,m$满足$u_k\in U_k$,由于$u_k\in U$对所有$k$成立,于是$\li u+m\in U$,即$\li U\oplus m\subseteq U$.\\
    另一方面,对于任意$u\in U$,都有
    \[u=\li v+m,v_k\in E(\lambda_k,T)\]
    
\end{proof}
\end{document}