\documentclass{ctexart}
\usepackage{geometry}
\usepackage[dvipsnames,svgnames]{xcolor}
\usepackage[strict]{changepage}
\usepackage{framed}
\usepackage{enumerate}
\usepackage{amsmath,amsthm,amssymb}
\usepackage{enumitem}
\usepackage{solution}

\allowdisplaybreaks
\geometry{left=2cm, right=2cm, top=2.5cm, bottom=2.5cm}

\begin{document}\pagestyle{empty}
\begin{center}
    \large\tbf{Linear Algebra Done Right 3E}
\end{center}
\begin{problem}[1.]
    设$T$是$V$到$W$的函数.$T$的\tbf{图(Graph)}是$V\times W$的子集,定义为
    $$\mathcal{G}(T)=\left\{(v,Tv)\in V\times W:v\in V\right\}$$
    证明:$T$是线性映射,当且仅当$\mathcal{G}(T)$是$V\times W$的子空间.
\end{problem}
\begin{proof}
    $\Rightarrow$:对任意$u,v\in V$,都有$Tu+Tv=T(u+v)$.于是对于任意$(v,Tv),(u,Tu)\in\mathcal{G}(T)$有
    $$(u,Tu)+(v,Tv)=(u+v,Tu+Tv)=(u+v,T(u+v))$$
    于是$\mathcal{G}(T)$对加法封闭.\\
    证明它对标量乘法也封闭的过程是类似的,在此不再赘述.\\
    $\Leftarrow$:对于任意$(u,Tu),(v,Tv)\in\mathcal{G}(T)$有$(u,Tu)+(v,Tv)=(u+v,Tu+Tv)\in\mathcal{G}(T)$.\\
    这就要求对于任意$u,v\in V$,$T(u+v)=Tu+Tv$,从而$T$满足可加性.\\
    证明$T$的齐次性的过程也是类似的,在此不再赘述.
\end{proof}
\begin{problem}[2.]
    设$\li V,m$是向量空间,使得$\li V\times m$是有限维的.试证明:对于任意$1\leqslant k\leqslant m$,$V_k$都是有限维的.
\end{problem}
\begin{proof}
    对于任意$V_k$,选取它的一组基.这组基的长度必然小于等于$\li V\times m$的基的长度.\\
    于是各$V_k$都是有限维的.
\end{proof}
\begin{problem}[3.]
    设$\li V,m$是向量空间.证明$\mathcal{L}(\li V\times m,W)$和$\mathcal{L}(V_1,W)\times\cdots\times\mathcal{L}(V_m,W)$是同构向量空间.
\end{problem}
\begin{proof}
    设$\Phi:\mathcal{L}(V_1,W)\times\cdots\times\mathcal{L}(V_m,W)\to\mathcal{L}(\li V\times m,W)$.\\
    其中$\Phi(\li T,m):\left(\li V\times m\right)\to W$满足$\Phi(\li T,m)(\li v,m)=T_1v_1+\cdots+T_mv_m$.\\
    不难验证$\Phi$是线性映射.\\
    再定义$\iota_k:V_k\to\li V\times m$,使得对于任意$v_k\in V_k$有$\iota_k(v_k)=(\mbf{0},\cdots,v_k,\cdots,\mbf{0})$.其中$v_k$出现在第$k$个位置.\\
    不难验证各$\iota_k$也是线性映射.\\
    再设$\Psi:\mathcal{L}(\li V\times m,W)\to\mathcal{L}(V_1,W)\times\cdots\times\mathcal{L}(V_m,W)$.\\
    其中,对于任意$T\in\mathcal{L}(\li V\times m,W)$有$\Psi(T)=\left(T\circ\iota_1,\cdots,T\circ\iota_m\right)$.\\
    现在,对于任意$\left(\li T,m\right)\in\mathcal{L}(V_1,W)\times\cdots\times\mathcal{L}(V_m,W)$都有
    $$\Psi\left(\Phi(\li T,m)\right)=(\Phi(\li T,m)\circ\iota_1,\cdots,\Phi(\li T,m)\circ\iota_m)$$
    对于任意$1\leqslant k\leqslant m$和任意$v_k\in V_k$有
    $$\left(\Phi(\li T,m)\circ\iota_k\right)(v_k)=T_1(\mbf{0})+\cdots+T_k(v_k)+\cdots+T_m(\mbf{0})=T_k(v_k)$$
    于是$\Phi(\li T,m)\circ\iota_k=T_k$,从而$\Psi\Phi=I$.\\
    现在,对于任意$T\in\mathcal{L}(\li V\times m,W)$和任意$(\li v,m)\in\li V\times m$有
    $$\begin{aligned}
        \Phi\left(\Psi(T)\right)(\li v,m)
        &= \Phi\left(T\circ\iota_1,\cdots,T\circ\iota_m\right)(\li v,m) \\
        &= (T\circ\iota_1)(v_1)+\cdots+(T\circ\iota_m)(v_m) \\
        &= T(v_1,\cdots,\mbf{0})+\cdots+T(\mbf{0},\cdots,v_m) \\
        &= T(\li v,m)
    \end{aligned}$$
    于是$(\Phi\Psi)(T)=T$,即$\Phi\Psi=I$.\\
    综上可知$\Phi$是可逆的,它的逆是$\Psi$.于是题中的两个向量空间同构.
\end{proof}
\begin{problem}[4.]
    设$\li W,m$是向量空间.证明$\mathcal{L}(V,\li W\times m)$和$\mathcal{L}(V,W_1)\times\cdots\times\mathcal{L}(V,W_m)$是同构向量空间.
\end{problem}
\begin{proof}
    这与\tbf{(3)}相类似,不再赘述.
\end{proof}
\begin{problem}[5.]
    对于$m\in\N^*$,定义$V^m$为
    $$V^m=\underbrace{V\times\cdots\times V}_{m\text{个}V}$$
    试证明:$V^m$与$\mathcal{L}(\F^m,V)$是同构向量空间.
\end{problem}
\begin{proof}
    我们有$$\dim V^m=\underbrace{\dim V+\cdots+\dim V}_{m\text{个}V}=m\dim V$$
    又有$$\dim(\mathcal{L}(\F^m,V))=(\dim(\F^n))(\dim V)=m\dim V$$
    于是两者维数相同,因而它们同构.
\end{proof}
\begin{problem}[6.]
    设$v,x\in V$,$U,W$是$V$的子空间,使得$v+U=x+W$.试证明:$U=W$.
\end{problem}
\begin{proof}
    由$v+U=x+W$可知对于任意$u\in U$,存在$w\in W$使得$v+u=x+w$,反之亦是同理.\\
    令$u=\mbf{0}$可知存在$w\in W$使得$v=x+w$.同理存在$u\in U$使得$v+u=x$.\\
    于是$v-x=w\in W$且$v-x=-u\in U$.\\
    于是对于任意$u\in U$,存在$w$使得$v+u=x+w$,就有$u=x+w-v=w-(v-x)\in W$,于是$U\subseteq W$.\\
    同理可以证明$W\subseteq U$.综上可知$U=W$,命题得证.
\end{proof}
\begin{problem}[7.]
    令$U=\left\{(x,y,z)\in\R^3:2x+3y+5z=0\right\}$.设$V\subseteq\R^3$.试证明:$V$是$U$的平移,当且仅当存在$c\in\R$使得%
    $V=\left\{(x,y,z)\in\R^3:2x+3y+5z=c\right\}$.
\end{problem}
\begin{proof}
    $\Rightarrow$:由于$A$是$U$的平移,不妨设$p:=(p_1,p_2,p_3)\in\R^3$使得$V=p+U$.\\
    于是对于任意$v:=(v_1,v_2,v_3)\in\R^3$都存在$u:=(u_1,u_2,u_3)\in\R^3$使得$v=p+u$.\\
    于是任意$v\in V$都满足$v=(p_1+u_1,p_2+u_2,p_3+u_3)$,而
    $$2(p_1+u_1)+5(p_2+u_2)+3(p_3+u_3)=(2p_1+5p_2+3p_3)+(2u_1+5u_2+3u_3)=(2p_1+5p_2+3p_3)$$
    于是令$c=2p_1+5p_2+3p_3$,对于任意$v=(v_1,v_2,v_3)$都有$v_1+v_2+v_3=c$.这就要求$V$具有题设的形式.\\
    $\Leftarrow$:设$p=\left(-\dfrac{c}{10},-\dfrac{c}{10},-\dfrac{c}{10}\right)$.\\
    对于任意$v:=(v_1,v_2,v_3)\in V$,都有$v+p=\left(v_1-\dfrac{c}{10},v_2-\dfrac{c}{10},v_3-\dfrac{c}{10}\right)$.而
    $$2\left(v_1-\dfrac{c}{10}\right)+5\left(v_2-\dfrac{c}{10}\right)+3\left(v_3-\dfrac{c}{10}\right)=2v_1+5v_2+3v_3-c=0$$
    这就说明$v+p\in U$.\\
    对于任意$u:=(u_1,u_2,u_3)$,亦有$u-p=\left(u_1+\dfrac{c}{10},u_2+\dfrac{c}{10},u_3+\dfrac{c}{10}\right)\in V$.\\
    这就表明$p+V=U$,即$V$是$U$的一个平移.
\end{proof}
\begin{problem}[8.]
    回答下列问题.
    \begin{enumerate}[label=\tbf{(\arabic*)}]
        \item 设$T\in\mathcal{L}(V,W),c\in W$.试证明:$\left\{x\in V:Tx=c\right\}$是$\varnothing$或$\nul T$的平移.
        \item 解释线性方程组的解集为什么是空集或者$\F^n$的某个子空间的平移.
    \end{enumerate}
\end{problem}
\begin{proof}
    \begin{enumerate}[label=\tbf{(\arabic*)}]
        \item 若$T$是单射,那么存在唯一$v\in V$使得$Tv=c$.这时$\left\{x\in V:Tx=c\right\}=v+\varnothing$.\\
            若$T$不是单射,假定某一$v\in V$满足$Tv=c$,那么对于任意$u\in\nul T$都有$T(u+v)=Tu+Tv=c$.\\
            又对于任意$u\in V$满足$Tu=c$,总有$T(u-v)=Tu-Tv=\mbf{0}$.于是总有$u-v\in\nul T$.\\
            综上可知$\left\{x\in V:Tx=c\right\}=v+\nul T$.
        \item 考虑一个有$n$个方程和$m$个未知数的线性方程组.\\
            对于$T\in\mathcal{L}(\F^m,\F^n)$和$c\in\F^n$,解集为$\left\{x\in\F^m:Tx=c\right\}$.\\
            于是该解集要么为空集的平移,要么为$\F^m$的子空间的平移.
    \end{enumerate}
\end{proof}
\begin{problem}[9.]
    证明:$V$的一非空子集$A$是$V$的某个子空间的平移,当且仅当$\lambda v+(1-\lambda)w\in A$对所有$v,w\in A$和所有$\lambda\in\F$成立.
\end{problem}
\begin{proof}
    $\Rightarrow$:设$V=x+U$,其中$x\in V$,$U$是$V$的子空间.\\
    对于任意$v,w\in A$,都存在$u_1,u_2\in U$使得$x+u_1=v,x+u_2=w$.于是
    $$\lambda v+(1-\lambda)w=\lambda(x+u_1)+(1-\lambda)(x+u_2)=x+\lambda u_1+(1-\lambda)u_2$$
    而$\lambda u_1+(1-\lambda)u_2\in U$,于是$x+\lambda u_1+(1-\lambda)u_2\in A$,命题得证.\\
    $\Leftarrow$:我们只需证明$\exists v\in A$使得$-v+A$是子空间.\\
    首先,由于$v\in A$,于是$v-v=\mbf{0}\in -v+A$.\\
    设$x\in -v+A$,于是存在$u\in A$使得$x=u-v$.对于任意$\lambda\in\F$有
    $$\lambda u+(1-\lambda)v=\lambda(u-v)+v=\lambda x+v\in A$$
    于是$\lambda x\in -v+A$,即$-v+A$对标量乘法封闭.\\
    对于任意$x_1,x_2\in -v+A$,总存在$u_1,u_2\in A$使得$x_1=u_1-v,x_2=u_2-v$.令$\lambda=\dfrac{1}{2}$,于是
    $$\dfrac{1}{2}u_1+\dfrac{1}{2}u_2=\dfrac{1}{2}(x_1+v)+\dfrac{1}{2}(x_2+v)=\left(\dfrac{1}{2}x_1+\dfrac{1}{2}x_2\right)+v\in A$$
    于是$\dfrac{1}{2}x_1+\dfrac{1}{2}x_2\in -v+A$.由于$-v+A$对标量乘法封闭,于是$x_1+x_2\in -v+A$,于是$-v+A$对加法封闭.\\
    综上可知$-v+A$是子空间,从而$A=v+(-v+A)$是子空间的平移.
\end{proof}
\begin{problem}[10.]
    设$A_1=v+U_1$且$A_2=w+U_2$,其中$v,w\in V$,$U_1,U_2$是$V$的子空间.试证明:$A_1\cap A_2$是$V$的某个子空间的平移或空集.
\end{problem}
\begin{proof}
    若$v-w\in U_1\cap U_2$,则$A_1\cap A_2=v+(U_1\cap U_2)=w+(U_1\cap U_2)$,自然是子空间的平移.\\
    若$v-w\notin U_1\cap U_2$,则$A_1\cap A_2=\varnothing$.
\end{proof}
\begin{problem}[11.]
    设$U=\left\{(x_1,x_2,\cdots)\in\F^{\infty}:x_k\neq 0\text{仅对有限多个}k\text{成立}\right\}$.
    \begin{enumerate}[label=\tbf{(\arabic*)}]
        \item 试证明$U$是$\F^\infty$的子空间.
        \item 试证明$\F^{\infty}/U$是无限维的.
    \end{enumerate}
\end{problem}
\begin{proof}
    \begin{enumerate}[label=\tbf{(\arabic*)}]
        \item 首先有$\mbf{0}=(0,0,\cdots)\in U$.\\
            对于任意$v,u\in U$,假设$v,u$中的非零元素分别有$m,n$个.那么$v+u$中的非零元素数目必然不超过$m+n$个,于是$u+v\in U$.\\
            对于任意$v\in U$和任意$\lambda\in\F$,$\lambda v$中的非零元素要么有$m$个,要么没有非零元素.于是$\lambda v\in U$.\\
            综上,$U$对加法和标量乘法封闭且有加法恒等元,于是$U$是$\F^{\infty}$的子空间.
        \item 考虑一组向量$(0,1,1,\cdots),(1,0,1,\cdots),\cdots$.各$v_k$的第$k$个元素为$0$,其余均为$1$.\\
            显然$v_1,\cdots\notin V$且线性无关.于是$v_1+U,\cdots$可以作为$\F^\infty/U$的一组基.这基的长度是无限的,于是$\F^\infty/U$自然是无限维的.
    \end{enumerate}
\end{proof}
\begin{problem}[12.]
    设$\li v,m\in V$.令$$A=\left\{\lambda_1v_1+\cdots+\lambda_mv_m:\li \lambda,m\in\F\text{且}\li \lambda+m=1\right\}$$
    \begin{enumerate}[label=\tbf{(\arabic*)}]
        \item 证明:$A$是$V$的某个子空间的平移.
        \item 证明:如果$B$是$V$的某个子空间的平移且$\left\{\li v,m\right\}\subseteq B$,那么$A\subseteq B$.
        \item 证明:$A$是$V$的某个子空间的平移,且该子空间的维数小于$m$.
    \end{enumerate}
\end{problem}
\begin{proof}
    \begin{enumerate}[label=\tbf{(\arabic*)}]
        \item 我们可以利用\tbf{(9)}的结论证明之.\\
            设$u_1=\alpha_1v_1+\cdots+\alpha_mv_m,u_2=\beta_1v_1+\cdots+\beta_mv_m$,其中$\displaystyle\sum_{i=1}^{m}\alpha_i=\sum_{i=1}^{n}\beta_i=1$.于是$u_1,u_2\in A$.\\
            我们有$\displaystyle\lambda u_1+(1-\lambda)u_2=\sum_{i=1}^{m}\left(\lambda\alpha_i+(1-\lambda)\beta_i\right)v_i$.\\
            又$\displaystyle\sum_{i=1}^{m}\lambda\alpha_i+(1-\lambda)\beta_i=\lambda\sum_{i=1}^{m}\alpha_i+(1-\lambda)\sum_{i=1}^{m}\beta_i=\lambda+(1-\lambda)=1$,于是$\lambda u_1+(1-\lambda)u_2\in A$.\\
            根据\tbf{(9)}可知$A$是$V$中某个子空间的平移.
        \item 假定$B=v+U$,记$u_k=v_k-v\in U,1\leqslant k\leqslant m$.\\
            现在,对于任意$\displaystyle w:=\sum_{i=1}^{m}\lambda_iv_i\in A$,都有
            $$w=\sum_{i=1}^{m}\lambda_i(v+u_i)=v+\sum_{i=1}^{m}u_i$$
            又$\sum_{i=1}^{m}u_i\in U$,于是$w\in B$,进而$A\subseteq B$.
        \item 注意到$v_m\in A$.设$U$满足$U+v_m=A$,即$U=A-v_m$,于是对于任意$\displaystyle u:=\sum_{i=1}^{m}\lambda_iv_i-v_m\in U$有
            $$u=\sum_{i=1}^{m}\lambda_iv_i-v_m=\sum_{i=1}^{m}\lambda_iv_i-\sum_{i=1}^{m}\lambda_iv_m=\sum_{i=1}^{m-1}\lambda_i(v_i-v_m)$$
            其中$\li \lambda,{m-1}\in\F$.这表明$U=\text{span}(\li v,{m-1})$,于是$\dim U\leqslant m-1<m$.
    \end{enumerate}
\end{proof}
\begin{problem}[13.]
    设$U$是$V$的子空间,使得$V/U$是有限维的.证明$V$和$U\times (V/U)$同构.
\end{problem}
\begin{proof}
    假定$v_1+U,\cdots,v_m+U$是$V/U$的一组基.于是$\li v,m$线性无关且均不属于$U$.\\
    考虑$\displaystyle x:=\sum_{i=1}^{m}c_iv_i$.假定$x\in U$,那么$x+U=U=\mbf{0}_{V/U}$.\\
    又$\displaystyle x+U=\sum_{i=1}^{m}c_i(v_i+U)=\mbf{0}_{V/U}$当且仅当$\li c=m=0$.于是当且仅当$\li c=m=0$时$x\in U$.\\
    现在,对于任意$v\in V$,我们证明存在$u\in U$和$\li a,m\in\F$使得$\displaystyle v=u+\sum_{i=1}^{m}a_iv_i$.\\
    假定$u_0\in U$,于是$v+u_0\in v+U$.\\
    又知存在$\li a,m\in\F$使得$\displaystyle v+U=\sum_{i=1}^{m}a_i(v_i+U)=\sum_{i=1}^{m}a_iv_i+U$.\\
    于是存在$u_1\in U$使得$\displaystyle v+u_0=\sum_{i=1}^{m}a_iv_i+u_1$.\\
    于是存在$u:=u_1-u_0\in U$使得$\displaystyle v=u+\sum_{i=1}^{n}a_iv_i$.\\
    现在我们证明上述表示方法是唯一的.设$v\in V$使得$\displaystyle v=u_1+\sum_{i=1}^{m}a_iv_i=u_2+\sum_{i=1}^{m}b_iv_i$.\\
    两式相减有$\displaystyle u_2-u_1=\sum_{i=1}^{m}(a_i-b_i)v_i$.而$u_2-u_1\in U$,于是$\displaystyle\sum_{i=1}^{m}(a_i-b_i)v_i=\mbf{0}$.\\
    从而只能有各$a_k=b_k$,$u_1=u_2$,于是上述表示方法是唯一的.\\
    现在,我们证明映射$\displaystyle T:v:=u+\sum_{i=1}^{m}a_iv_i\mapsto \left(u,\sum_{i=1}^{m}a_i(v_i+U)\right)$是线性的.
    \begin{enumerate}[label=\tbf{(\alph*)}]
        \item $T\mbf{0}=(\mbf{0}_U,\mbf{0}_{V/U})=\mbf{0}_{U\times(V/U)}$.
        \item 对于任意$\displaystyle v_1:=u_1+\sum_{i=1}^{m}a_iv_i,v_2:=u_2+\sum_{i=1}^{m}b_iv_i\in V$有
            $$\begin{aligned}
                T(v_1+v_2)
                &= T((u_1+u_2)+\sum_{i=1}^m(a_i+b_i)v_i) \\
                &= \left(u_1+u_2,\sum_{i=1}^{m}(a_i+b_i)(v_i+U)\right) \\
                &= \left(u_1,\sum_{i=1}^{m}a_i(v_i+U)\right)+\left(u_2,\sum_{i=1}^{m}b_i(v_i+U)\right) \\
                &= Tv_1+Tv_2
            \end{aligned}$$
            于是$T$具有可加性.
        \item 证明$T$具有齐次性的过程是类似的,在此便不再赘述.
    \end{enumerate}
    于是$T$是$V$到$U\times(V/U)$的线性映射.\\
    观察$T$的定义,可知$T$既是单射又是满射.于是$T$是这两个空间的一个同构,即$V$和$U\times(V/U)$是同构的.
\end{proof}
\begin{problem}[14.]
    设$U$和$W$是$V$的子空间,且$V=U\oplus W$.设$\li w,m$是$W$的基,证明$w_1+U,\cdots,w_m+U$是$V/U$的基.
\end{problem}
\begin{proof}
    对于任意$v\in V$,由于$V=U\oplus W$,于是存在唯一的$w\in W$和$u\in U$使得$v=w+u$.\\
    于是$v+U=w+u+U=w+U$.对于任意$v+U\in V/U$,设此$v$对应的$w:=a_1w_1+\cdots+a_mw_m\in W$,则
    $$v+U=w+U=(a_1w_1+\cdots+a_mw_m)+U=\sum_{i=1}^{m}a_i(w_i+U)$$
    下面证明$w_1+U,\cdots,w_i+U$线性无关.在上式中令$v=\mbf{0}$,则
    $$\mbf{0}+U=\mbf{0}_{V/U}=\sum_{i=1}^{m}a_i(w_i+U)$$
    当且仅当$\li a=m=0$成立,于是$w_1+U,\cdots,w_m+U$线性无关.\\
    综上可知$w_1+U,\cdots,w_m+U$是$V/U$的基.命题得证.
\end{proof}
\begin{problem}[15.]
    设$U$是$V$的子空间,$v_1+U,\cdots,v_m+U$是$V/U$的基,$\li u,n$是$U$的基.试证明$\li v,m\li u,n$是$V$的基.
\end{problem}
\begin{proof}
    不难得知$\li v,m,\li u,n$线性无关.我们只需证明这向量组张成$V$即可.\\
    对于任意$v\in V$,设$\displaystyle v+U=\sum_{i=1}^{m}a_i(v_i+U)=\sum_{i=1}^ma_iv_i+U$.\\
    于是$u:=\displaystyle v-\sum_{i=1}^ma_iv_i\in U$.设$u=\sum_{j=1}^{n}b_ju_j$,则有
    $$v=\sum_{i=1}^ma_iv_i+\sum_{j=1}^nb_ju_j$$
    于是$V=\text{span}\left(\li v,m,\li u,n\right)$,进而这向量组是$V$的基.
\end{proof}
\begin{problem}[16.]
    设$\varphi\in\mathcal{L}(V,\F)$,$\varphi\neq\mbf{0}$.证明$\dim V/(\nul\varphi)=1$.
\end{problem}
\begin{proof}
    考虑$\tilde{\varphi}:V/(\nul\varphi)\to W$满足$\tilde{\varphi}(v+\nul\varphi)=\varphi v$.\\
    根据前面的定理可知$\tilde{\varphi}$是单射且$\range\varphi=\range\tilde{\varphi}$.\\
    这表明$\tilde{\varphi}$是将$V/(\nul\varphi)$映射成$\range\varphi$的同构.又$\dim\range\varphi=\dim\F=1$,于是$\dim V/(\nul\varphi)=1$.
\end{proof}
\begin{problem}[17.]
    设$U$是$V$的子空间,使得$\dim V/U=1$.证明:存在$\varphi\in\mathcal{L}(V,\F)$使得$\nul\varphi=U$.
\end{problem}
\begin{proof}
    设$V/U$的基为$v+U$.对于任意$w\in V$,都有$w+U=a(v+U)=av+U$.令$\varphi$满足
    $$\varphi w=a:w+U=av+U$$
    于是$\varphi w=\mbf{0}$当且仅当$w+U=U$,即$w\in U$.于是$\nul\varphi=U$.命题得证.
\end{proof}
\begin{problem}[18.]
    设$U$是$V$的子空间,使得$V/U$是有限维的.
    \begin{enumerate}[label=\tbf{(\arabic*)}]
        \item 证明:若$W$是$V$的有限维子空间使得$V=U+W$,那么$\dim W\geqslant\dim V/U$.
        \item 证明:存在$V$的有限维子空间$W$使得$\dim W=\dim V/U$且$V=U\oplus W$.
    \end{enumerate}
\end{problem}
\begin{proof}
    \begin{enumerate}[label=\tbf{(\arabic*)}]
        \item 对于任意$v\in V$都存在$w\in W,u\in U$使得$v=w+u$.\\
            不妨设$V/U$的一组基为$v_1+U,\cdots,v_m+U$.\\
            由于$V=U+W$,于是对于任意$1\leqslant k\leqslant m$,都存在$w_k\in W,u_k\in U$使得$v_k=w_k+u_k$.\\
            即$v_k+U=(w_k+u_k)+U=w_k+U$.对于任意$v+U\in V/U$有
            $$v+U=\sum_{i=1}^{m}a_i(v_i+U)=\sum_{i=1}^{m}a_i(w_i+U)=\left(\sum_{i=1}^{m}a_iw_i\right)+U$$
            于是$\li w,m$线性无关,否则表出$\mbf{0}_{V/U}$的标量$\li a,m$将不唯一.\\
            于是$\li w,m$是$W$中长度为$m$的线性无关组,因而$\dim W\geqslant m=\dim V/U$.
        \item 在\tbf{(1)}中令$W=\text{span}(\li w,m)$即可.\\
            如此,对于任意$v\in V$,都有$\displaystyle v+U=\left(\sum_{i=1}^{m}a_iw_i\right)+U$,即$\displaystyle v-\sum_{i=1}^{m}w_k\in U$.\\
            又$v=\mbf0$当且仅当$\li a=m=0$,这表明$U+W$是直和.
    \end{enumerate}
\end{proof}
\begin{problem}[19.]
    设$T\in\mathcal{L}(V,W)$且$U$是$V$的子空间.令$\pi$表示$V$到$V/U$的商映射.证明:存在$S\in\mathcal{L}(V/U,W)$使得$T=S\circ\pi$当且仅当$U\subseteq\nul T$.
\end{problem}
\begin{proof}
    $\Rightarrow$:对于任意$u\in U$,都有$Tu=S(\pi u)=S\mbf{0}=\mbf{0}$.于是$u\in\nul T$,这表明$U\subseteq\nul T$.\\
    $\Leftarrow$:设对于任意$v\in V$都有$S(v+U)=Tv$.我们现在证明$S$是线性映射.
    \begin{enumerate}[label=\tbf{(\alph*)}]
        \item $S\mbf{0}_{V/U}=S(\mbf{0}_V+U)=T\mbf{0}_V=\mbf{0}_W$.于是$S\mbf{0}=\mbf{0}$.
        \item 对于任意$v,w\in V$,都有$S((v+w)+U)=T(v+w)=Tv+Tw=S(v+U)+T(w+U)$,于是$S$满足可加性.
        \item 对于任意$v\in V$和任意$\lambda\in\F$,都有$S(\lambda v+U)=T(\lambda v)=\lambda Tv=\lambda S(v+U)$,于是$S$满足齐次性.
    \end{enumerate}
    这样的$S$自然是线性映射.
\end{proof}
\end{document}