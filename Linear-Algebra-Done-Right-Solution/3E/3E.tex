\documentclass{ctexart}
\usepackage{geometry}
\usepackage[dvipsnames,svgnames]{xcolor}
\usepackage[strict]{changepage}
\usepackage{framed}
\usepackage{enumerate}
\usepackage{amsmath,amsthm,amssymb}
\usepackage{enumitem}
\usepackage{solution}

\allowdisplaybreaks
\geometry{left=2cm, right=2cm, top=2.5cm, bottom=2.5cm}

\begin{document}\pagestyle{empty}
\begin{problem}[1.]
    设$T$是$V$到$W$的函数.$T$的\tbf{图(Graph)}是$V\times W$的子集,定义为
    $$\mathcal{G}(T)=\left\{(v,Tv)\in V\times W:v\in V\right\}$$
    证明:$T$是线性映射,当且仅当$\mathcal{G}(T)$是$V\times W$的子空间.
\end{problem}
\begin{proof}
    $\Rightarrow$:对任意$u,v\in V$,都有$Tu+Tv=T(u+v)$.于是对于任意$(v,Tv),(u,Tu)\in\mathcal{G}(T)$有
    $$(u,Tu)+(v,Tv)=(u+v,Tu+Tv)=(u+v,T(u+v))$$
    于是$\mathcal{G}(T)$对加法封闭.\\
    证明它对标量乘法也封闭的过程是类似的,在此不再赘述.\\
    $\Leftarrow$:对于任意$(u,Tu),(v,Tv)\in\mathcal{G}(T)$有$(u,Tu)+(v,Tv)=(u+v,Tu+Tv)\in\mathcal{G}(T)$.\\
    这就要求对于任意$u,v\in V$,$T(u+v)=Tu+Tv$,从而$T$满足可加性.\\
    证明$T$的齐次性的过程也是类似的,在此不再赘述.
\end{proof}
\begin{problem}[2.]
    设$\li V,m$是向量空间,使得$\li V\times m$是有限维的.试证明:对于任意$1\leqslant k\leqslant m$,$V_k$都是有限维的.
\end{problem}
\begin{proof}
    对于任意$V_k$,选取它的一组基.这组基的长度必然小于等于$\li V\times m$的基的长度.\\
    于是各$V_k$都是有限维的.
\end{proof}
\begin{problem}[3.]
    设$\li V,m$是向量空间.证明$\mathcal{L}(\li V\times m,W)$和$\mathcal{L}(V_1,W)\times\cdots\times\mathcal{L}(V_m,W)$是同构向量空间.
\end{problem}
\begin{proof}
    我们有$$\begin{aligned}
        \dim(\mathcal{L}(\li V\times m),W)
        &= \dim(\li V\times m)\cdot(\dim W) \\
        &= (\li {\dim V}+m)(\dim W)
    \end{aligned}$$
    又有$$\begin{aligned}
        \dim\left(\mathcal{L}(V_1,W)\times\cdots\times\mathcal{L}(V_m,W)\right)
        &= \dim(\mathcal{L}(V_1,W))+\cdots+\dim(\mathcal{L}(V_m,W)) \\
        &= (\dim V_1)(\dim W)+\cdots+(\dim V_m)(\dim W) \\
        &= (\li {\dim V}+m)(\dim W)
    \end{aligned}$$
    于是两者维数相同,因而它们同构.
\end{proof}
\begin{problem}[4.]
    设$\li W,m$是向量空间.证明$\mathcal{L}(V,\li W\times m)$和$\mathcal{L}(V,W_1)\times\cdots\times\mathcal{L}(V,W_m)$是同构向量空间.
\end{problem}
\begin{proof}
    这与\tbf{(3)}相类似,不再赘述.
\end{proof}
\begin{problem}[5.]
    对于$m\in\N^*$,定义$V^m$为
    $$V^m=\underbrace{V\times\cdots\times V}_{m\text{个}V}$$
    试证明:$V^m$与$\mathcal{L}(\F^m,V)$是同构向量空间.
\end{problem}
\begin{proof}
    我们有$$\dim V^m=\underbrace{\dim V+\cdots+\dim V}_{m\text{个}V}=m\dim V$$
    又有$$\dim(\mathcal{L}(\F^m,V))=(\dim(\F^n))(\dim V)=m\dim V$$
    于是两者维数相同,因而它们同构.
\end{proof}
\begin{problem}[6.]
    设$v,x\in V$,$U,W$是$V$的子空间,使得$v+U=x+W$.试证明:$U=W$.
\end{problem}
\begin{proof}
    由$v+U=x+W$可知对于任意$u\in U$,存在$w\in W$使得$v+u=x+w$,反之亦是同理.\\
    令$u=\mbf{0}$可知存在$w\in W$使得$v=x+w$.同理存在$u\in U$使得$v+u=x$.\\
    于是$v-x=w\in W$且$v-x=-u\in U$.\\
    于是对于任意$u\in U$,存在$w$使得$v+u=x+w$,就有$u=x+w-v=w-(v-x)\in W$,于是$U\subseteq W$.\\
    同理可以证明$W\subseteq U$.综上可知$U=W$,命题得证.
\end{proof}
\begin{problem}[7.]
    令$U=\left\{(x,y,z)\in\R^3:2x+3y+5z=0\right\}$.设$V\subseteq\R^3$.试证明:$V$是$U$的平移,当且仅当存在$c\in\R$使得%
    $V=\left\{(x,y,z)\in\R^3:2x+3y+5z=c\right\}$.
\end{problem}
\begin{proof}
    $\Rightarrow$:由于$A$是$U$的平移,不妨设$p:=(p_1,p_2,p_3)\in\R^3$使得$V=p+U$.\\
    于是对于任意$v:=(v_1,v_2,v_3)\in\R^3$都存在$u:=(u_1,u_2,u_3)\in\R^3$使得$v=p+u$.\\
    于是任意$v\in V$都满足$v=(p_1+u_1,p_2+u_2,p_3+u_3)$,而
    $$2(p_1+u_1)+5(p_2+u_2)+3(p_3+u_3)=(2p_1+5p_2+3p_3)+(2u_1+5u_2+3u_3)=(2p_1+5p_2+3p_3)$$
    于是令$c=2p_1+5p_2+3p_3$,对于任意$v=(v_1,v_2,v_3)$都有$v_1+v_2+v_3=c$.这就要求$V$具有题设的形式.\\
    $\Leftarrow$:设$p=\left(-\dfrac{c}{10},-\dfrac{c}{10},-\dfrac{c}{10}\right)$.\\
    对于任意$v:=(v_1,v_2,v_3)\in V$,都有$v+p=\left(v_1-\dfrac{c}{10},v_2-\dfrac{c}{10},v_3-\dfrac{c}{10}\right)$.而
    $$2\left(v_1-\dfrac{c}{10}\right)+5\left(v_2-\dfrac{c}{10}\right)+3\left(v_3-\dfrac{c}{10}\right)=2v_1+5v_2+3v_3-c=0$$
    这就说明$v+p\in U$.\\
    对于任意$u:=(u_1,u_2,u_3)$,亦有$u-p=\left(u_1+\dfrac{c}{10},u_2+\dfrac{c}{10},u_3+\dfrac{c}{10}\right)\in V$.\\
    这就表明$p+V=U$,即$V$是$U$的一个平移.
\end{proof}
\begin{problem}[8.]
    回答下列问题.
    \begin{enumerate}[label=\tbf{(\arabic*)}]
        \item 设$T\in\mathcal{L}(V,W),c\in W$.试证明:$\left\{x\in V:Tx=c\right\}$是$\varnothing$或$\nul T$的平移.
        \item 解释线性方程组的解集为什么是空集或者$\F^n$的某个子空间的平移.
    \end{enumerate}
\end{problem}
\begin{proof}
    \begin{enumerate}[label=\tbf{(\arabic*)}]
        \item 若$T$是单射,那么存在唯一$v\in V$使得$Tv=c$.这时$\left\{x\in V:Tx=c\right\}=v+\varnothing$.\\
            若$T$不是单射,假定某一$v\in V$满足$Tv=c$,那么对于任意$u\in\nul T$都有$T(u+v)=Tu+Tv=c$.\\
            又对于任意$u\in V$满足$Tu=c$,总有$T(u-v)=Tu-Tv=\mbf{0}$.于是$u-v\in\nul T$.\\
            综上可知$\left\{x\in V:Tx=c\right\}=v+\nul T$.
    \end{enumerate}
\end{proof}
\end{document}