\documentclass{ctexart}
\usepackage{geometry}
\usepackage[dvipsnames,svgnames]{xcolor}
\usepackage[strict]{changepage}
\usepackage{framed}
\usepackage{enumerate}
\usepackage{amsmath,amsthm,amssymb}
\usepackage{enumitem}
\usepackage{solution}

\allowdisplaybreaks
\geometry{left=2cm, right=2cm, top=2.5cm, bottom=2.5cm}

\begin{document}\pagestyle{empty}
\begin{center}
    \large\tbf{Linear Algebra Done Right 7C}
\end{center}
\begin{problem}[1.]
    设$T\in\L(V)$.试证明:如果$T$和$-T$都是正算子,那么$T=\mbf0$.
\end{problem}
\begin{proof}
    由题意可知对任意$v\in V$有
    \[\inprod{Tv}{v}\geqslant0\]
    且
    \[\inprod{-Tv}{v}=-\inprod{Tv}{v}\geqslant\]
    于是$\inprod{Tv}{v}=0$,因而$T=\mbf0$(在实内积空间上需要$T$自伴,这由正算子的定义可得).
\end{proof}
\begin{problem}[2.]
    设$T\in\L(\F^4)$关于其标准基的矩阵为
    \[\begin{pmatrix}
        2&-1&0&0\\
        -1&2&-1&0\\
        0&-1&2&-1\\
        0&0&-1&2
    \end{pmatrix}\]
    试证明:$T$是可逆正算子.
\end{problem}
\begin{proof}
    $\F^4$的标准基也是$\F^4$的规范正交基.观察这矩阵,可知$(\M(T))^*=\M(T)$,于是$T^*=T$,即$T$自伴.\\
    一些计算表明$T$的特征值为
    \[\dfrac{3\pi\sqrt5}{2},\dfrac{5\pm\sqrt5}{2}\]
    于是$T$的特征值均非零且非负,因而它是可逆正算子.
\end{proof}
\begin{problem}[3.]
    设$n\in\N^*$,算子$T\in\L(\F^n)$关于其标准基的矩阵中的元素均为$1$.试证明:$T$是正算子.
\end{problem}
\begin{proof}
    对于任意$v:=(\li x,n)\in\F^n$有
    \[\begin{aligned}
        \inprod{Tv}{v}
        &= \inprod{T(\li x,n)}{(\li x,n)} \\
        &= \left(\li x+n\right)^2 \\
        &\geqslant 0
    \end{aligned}\]
    于是$T$是正算子.
\end{proof}
\begin{problem}[4.]
    设$n\in\N^*$且$n>1$,试证明:存在$n\times n$矩阵$A$,其所有元素都是正数且$A=A^*$,%
    但$\F^n$上关于其标准基的矩阵为$A$的算子不是正算子.
\end{problem}
\begin{proof}
    考虑矩阵
    \[A=\begin{pmatrix}
        1&2&\cdots&2\\
        2&1&\cdots&2\\
        \vdots&\vdots&\ddots&
    \end{pmatrix}\]
    令$A$为$T\in\L(\F^2)$对应于$\F^2$的标准基的矩阵.于是
    \[\inprod{T(x,y)}{(x,y)}
    =\inprod{(x+2y,2x+y)}{(x,y)}
    =(x+y)^2+2xy\]
    令$(x,y)=(1,-1)$,则有$\inprod{T(x,y)}{(x,y)}<0$,于是$T$不是正算子.
\end{proof}
\begin{problem}[5.]
    设$T\in\L(V)$是自伴的.试证明:$T$是正算子当且仅当对于$V$的任意规范正交基$\li e,n$都有$T$关于其的矩阵的对角线元素全为非负数.
\end{problem}
\begin{proof}
    $\Rightarrow$:对于任意$V$的规范正交基$\li e,n$和任意$k\in\{1,\cdots,n\}$,由于$T$是正算子,于是
    \[\inprod{Te_k}{e_k}\geqslant0\]
    又因为$Te_k=\inprod{Te_k}{e_1}e_1+\cdots+\inprod{Te_k}{e_n}e_n$,于是
    \[\M(T,(\li e,n))_{k,k}=\inprod{Te_k}{e_k}\geqslant0\]
    于是此矩阵的对角线上均为非负数.\\
    $\Leftarrow$:由于$T$是自伴的,于是存在$V$的规范正交基$\li e,n$使得$T$关于其有对角矩阵$A$.\\
    根据题意,$A_{1,1},\cdots,A_{n,n}\geqslant0$,从而对于任意$v:=a_1e_1+\cdots+a_ne_n\in V$有
    \[\begin{aligned}
        \inprod{Tv}{v}
        &= \inprod{A_{1,1}a_1e_1}{a_1e_1}+\cdots+\inprod{A_{n,n}a_ne_n}{a_ne_n} \\
        &= A_{1,1}a_1^2+\cdots+A_{n,n}a_n^2 \\
        &\geqslant 0
    \end{aligned}\]
    从而$T$为正算子.
\end{proof}
\begin{problem}[6.]
    试证明:$V$上两正算子之和为正算子.
\end{problem}
\begin{proof}
    对于任意正的$S,T\in\L(V)$和任意$v\in V$有
    \[\inprod{(S+T)v}{v}=\inprod{Sv+Tv}{v}=\inprod{Sv}{v}+\inprod{Tv}{v}\geqslant 0\]
    于是$S+T$是正的.
\end{proof}
\begin{problem}[7.]
    设$S\in\L(V)$是可逆正算子,$T\in\L(V)$是正算子,试证明:$S+T$可逆.
\end{problem}
\begin{lemma}[Lemma.L.11]
    设$T\in\L(V)$是正算子,那么$T$可逆当且仅当$\inprod{Tv}{v}>0$对所有非零的$v\in V$成立.
\end{lemma}
\begin{proof}
    $\Rightarrow$:由于$T$可逆,于是对于任意非零的$v\in V$有$Tv\neq\mbf0$.于是$\inprod{Tv}{v}>0$.\\
    $\Leftarrow$:如果$T$不可逆,那么存在非零的$v\in V$使得$Tv=\mbf0$,从而$\inprod{Tv}{v}=0$,这与条件矛盾.于是$T$可逆.
\end{proof}
\begin{proof}
    根据\tbf{Lemma.L.11}可知$\inprod{Sv}{v}>0$对所有非零的$v\in V$成立.于是对于任意非零的$v\in V$有
    \[\inprod{(S+T)v}{v}=\inprod{Sv}{v}+\inprod{Tv}{v}>0\]
    从而$S+T$是可逆的.
\end{proof}
\begin{problem}[8.]
    设$T\in\L(V)$.试证明:$T$是正算子,当且仅当$T^\dagger$是正算子.
\end{problem}
\begin{proof}
    由于$T$是正算子,于是令$V$有一规范正交基$\li e,n$是分别对应于$T$的特征值$\li\lambda,n$的特征向量.\\
    于是$\li\lambda,n\geqslant0$.根据\tbf{7B.25}有
    \[T^\dagger e_k=\left\{\begin{array}{l}
        \frac{1}{\lambda_k}e_k,\lambda_k\neq0\\\mbf0,\lambda_k=0
    \end{array}\right.\]
    对所有$k\in\{1,\cdots,n\}$成立.于是$T^\dagger$关于$\li e,n$的矩阵是对角线上均为非负数的对角矩阵,从而$T^\dagger$是正算子.\\
    交换$T$和$T^\dagger$即可证得另一方向.于是命题得证.
\end{proof}
\begin{problem}[9.]
    设$T\in\L(V)$是正算子,$S\in\L(W,V)$.试证明:$S^*TS$是$W$上的正算子.
\end{problem}
\begin{proof}
    对于任意$w\in W$有
    \[\inprod{S^*TSw}{w}=\inprod{T(Sw)}{Sw}\geqslant0\]
    于是$S^*TS$是$W$上的正算子.
\end{proof}
\begin{problem}[10.]
    设$T$是$V$上的正算子,设$v,w\in V$满足$Tv=w$且$Tw=v$.试证明:$v=w$.
\end{problem}
\begin{proof}
    我们有
    \[\inprod{T(v-w)}{v-w}=\inprod{w-v}{v-w}=-||v-w||^2\geqslant0\]
    于是$||v-w||=0$,即$v=w$.
\end{proof}
\begin{problem}[11.]
    设$T$是$V$上的正算子,$U$是$V$在$T$下不变的子空间,试证明:$T|_U\in\L(U)$是$U$上的正算子.
\end{problem}
\begin{proof}
    对于任意$u\in U$都有$Tu\in U\subseteq V$,又因为$T$是正的,于是
    \[\inprod{T|_Uu}{u}=\inprod{Tu}{u}\geqslant0\]
    于是$T|_U$是$U$上的正算子.
\end{proof}
\begin{problem}[12.]
    设$T\in\L(V)$是正算子,试证明:对于任意$k\in\N^*$,$T^k$都是正算子.
\end{problem}
\begin{proof}
    由于$T$是正算子,于是$T$关于$V$的某组规范正交基$\li e,n$有对角矩阵
    \[\M(T)=\begin{pmatrix}
        \lambda_1&\cdots&0\\
        \vdots&\ddots&\vdots\\
        0&\cdots&\lambda_n
    \end{pmatrix}\]
    且$\li\lambda,n\geqslant0$.\\
    于是$T^k$关于这基的矩阵为
    \[\M(T^k)=(\M(T))^k=\begin{pmatrix}
        \lambda_1^k&\cdots&0\\
        \vdots&\ddots&\vdots\\
        0&\cdots&\lambda_n^k
    \end{pmatrix}\]
    这是对角线均为非负数的对角矩阵,从而$T^k$也是正算子.
\end{proof}
\begin{problem}[13.]
    设$T\in\L(V)$是自伴的,设$\alpha\in\R$.回答下列问题.
    \begin{enumerate}[label=\tbf{(\arabic*)}]
        \item 试证明:$T-\alpha I$是正算子,当且仅当$\alpha$不大于$T$的任意特征值.
        \item 试证明:$\alpha I-T$是正算子,当且仅当$\alpha$不小于$T$的任意特征值.
    \end{enumerate}
\end{problem}
\begin{proof}
    \begin{enumerate}[label=\tbf{(\arabic*)}]
        \item 由于$T$是自伴算子,于是考虑$V$的某个规范正交基$\li e,n$使得$T$关于其有对角矩阵
            \[\M(T)=\begin{pmatrix}
                \lambda_1&\cdots&0\\
                \vdots&\ddots&\vdots\\
                0&\cdots&\lambda_n
            \end{pmatrix}\]
            则$\li\lambda,n$为$T$的特征值.\\
            于是$T-\alpha I$关于这基的矩阵为
            \[\M(T-\alpha I)=\begin{pmatrix}
                \lambda_1-\alpha&\cdots&0\\
                \vdots&\ddots&\vdots\\
                0&\cdots&\lambda_n-\alpha
            \end{pmatrix}\]
            因为$\alpha\leqslant\displaystyle\min_{k\in\{1,\cdots,n\}}\{\lambda_k\}$,于是$\lambda_1-\alpha,\cdots,\lambda_n-\alpha\geqslant0$.\\
            于是$T-\alpha I$关于这基有对角线元素非负的对角矩阵,于是$T-\alpha I$是正算子.
        \item 与\tbf{(1)}类似,不再赘述.
    \end{enumerate}
\end{proof}
\end{document}