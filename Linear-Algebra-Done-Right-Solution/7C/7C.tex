\documentclass{ctexart}
\usepackage{geometry}
\usepackage[dvipsnames,svgnames]{xcolor}
\usepackage[strict]{changepage}
\usepackage{framed}
\usepackage{enumerate}
\usepackage{amsmath,amsthm,amssymb}
\usepackage{enumitem}
\usepackage{solution}

\allowdisplaybreaks
\geometry{left=2cm, right=2cm, top=2.5cm, bottom=2.5cm}

\begin{document}\pagestyle{empty}
\begin{center}
    \large\tbf{Linear Algebra Done Right 7C}
\end{center}
\begin{problem}[1.]
    设$T\in\L(V)$.试证明:如果$T$和$-T$都是正算子,那么$T=\mbf0$.
\end{problem}
\begin{proof}
    由题意可知对任意$v\in V$有
    \[\inprod{Tv}{v}\geqslant0\]
    且
    \[\inprod{-Tv}{v}=-\inprod{Tv}{v}\geqslant\]
    于是$\inprod{Tv}{v}=0$,因而$T=\mbf0$(在实内积空间上需要$T$自伴,这由正算子的定义可得).
\end{proof}
\begin{problem}[2.]
    设$T\in\L(\F^4)$关于其标准基的矩阵为
    \[\begin{pmatrix}
        2&-1&0&0\\
        -1&2&-1&0\\
        0&-1&2&-1\\
        0&0&-1&2
    \end{pmatrix}\]
    试证明:$T$是可逆正算子.
\end{problem}
\begin{proof}
    $\F^4$的标准基也是$\F^4$的规范正交基.观察这矩阵,可知$(\M(T))^*=\M(T)$,于是$T^*=T$,即$T$自伴.\\
    一些计算表明$T$的特征值为
    \[\dfrac{3\pi\sqrt5}{2},\dfrac{5\pm\sqrt5}{2}\]
    于是$T$的特征值均非零且非负,因而它是可逆正算子.
\end{proof}
\begin{problem}[3.]
    设$n\in\N^*$,算子$T\in\L(\F^n)$关于其标准基的矩阵中的元素均为$1$.试证明:$T$是正算子.
\end{problem}
\begin{proof}
    对于任意$v:=(\li x,n)\in\F^n$有
    \[\begin{aligned}
        \inprod{Tv}{v}
        &= \inprod{T(\li x,n)}{(\li x,n)} \\
        &= \left(\li x+n\right)^2 \\
        &\geqslant 0
    \end{aligned}\]
    于是$T$是正算子.
\end{proof}
\begin{problem}[4.]
    设$n\in\N^*$且$n>1$,试证明:存在$n\times n$矩阵$A$,其所有元素都是正数且$A=A^*$,%
    但$\F^n$上关于其标准基的矩阵为$A$的算子不是正算子.
\end{problem}
\begin{proof}
    考虑矩阵
    \[A=\begin{pmatrix}
        1&2&\cdots&2\\
        2&1&\cdots&2\\
        \vdots&\vdots&\ddots&
    \end{pmatrix}\]
    令$A$为$T\in\L(\F^2)$对应于$\F^2$的标准基的矩阵.于是
    \[\inprod{T(x,y)}{(x,y)}
    =\inprod{(x+2y,2x+y)}{(x,y)}
    =(x+y)^2+2xy\]
    令$(x,y)=(1,-1)$,则有$\inprod{T(x,y)}{(x,y)}<0$,于是$T$不是正算子.
\end{proof}

\end{document}