\documentclass{ctexart}
\usepackage{geometry}
\usepackage[dvipsnames,svgnames]{xcolor}
\usepackage[strict]{changepage}
\usepackage{framed}
\usepackage{enumerate}
\usepackage{amsmath,amsthm,amssymb}
\usepackage{enumitem}
\usepackage{solution}

\allowdisplaybreaks
\geometry{left=2cm, right=2cm, top=2.5cm, bottom=2.5cm}

\begin{document}\pagestyle{empty}
\begin{center}
    \large\tbf{Linear Algebra Done Right 7C}
\end{center}
\begin{problem}[1.]
    设$T\in\L(V)$.试证明:如果$T$和$-T$都是正算子,那么$T=\mbf0$.
\end{problem}
\begin{proof}
    由题意可知对任意$v\in V$有
    \[\inprod{Tv}{v}\geqslant0\]
    且
    \[\inprod{-Tv}{v}=-\inprod{Tv}{v}\geqslant\]
    于是$\inprod{Tv}{v}=0$,因而$T=\mbf0$(在实内积空间上需要$T$自伴,这由正算子的定义可得).
\end{proof}
\begin{problem}[2.]
    设$T\in\L(\F^4)$关于其标准基的矩阵为
    \[\begin{pmatrix}
        2&-1&0&0\\
        -1&2&-1&0\\
        0&-1&2&-1\\
        0&0&-1&2
    \end{pmatrix}\]
    试证明:$T$是可逆正算子.
\end{problem}
\begin{proof}
    $\F^4$的标准基也是$\F^4$的规范正交基.观察这矩阵,可知$(\M(T))^*=\M(T)$,于是$T^*=T$,即$T$自伴.\\
    一些计算表明$T$的特征值为
    \[\dfrac{3\pi\sqrt5}{2},\dfrac{5\pm\sqrt5}{2}\]
    于是$T$的特征值均非零且非负,因而它是可逆正算子.
\end{proof}
\begin{problem}[3.]
    设$n\in\N^*$,算子$T\in\L(\F^n)$关于其标准基的矩阵中的元素均为$1$.试证明:$T$是正算子.
\end{problem}
\begin{proof}
    对于任意$v:=(\li x,n)\in\F^n$有
    \[\begin{aligned}
        \inprod{Tv}{v}
        &= \inprod{T(\li x,n)}{(\li x,n)} \\
        &= \left(\li x+n\right)^2 \\
        &\geqslant 0
    \end{aligned}\]
    于是$T$是正算子.
\end{proof}
\begin{problem}[4.]
    设$n\in\N^*$且$n>1$,试证明:存在$n\times n$矩阵$A$,其所有元素都是正数且$A=A^*$,%
    但$\F^n$上关于其标准基的矩阵为$A$的算子不是正算子.
\end{problem}
\begin{proof}
    考虑矩阵
    \[A=\begin{pmatrix}
        1&2&\cdots&2\\
        2&1&\cdots&2\\
        \vdots&\vdots&\ddots&
    \end{pmatrix}\]
    令$A$为$T\in\L(\F^2)$对应于$\F^2$的标准基的矩阵.于是
    \[\inprod{T(x,y)}{(x,y)}
    =\inprod{(x+2y,2x+y)}{(x,y)}
    =(x+y)^2+2xy\]
    令$(x,y)=(1,-1)$,则有$\inprod{T(x,y)}{(x,y)}<0$,于是$T$不是正算子.
\end{proof}
\begin{problem}[5.]
    设$T\in\L(V)$是自伴的.试证明:$T$是正算子当且仅当对于$V$的任意规范正交基$\li e,n$都有$T$关于其的矩阵的对角线元素全为非负数.
\end{problem}
\begin{proof}
    $\Rightarrow$:对于任意$V$的规范正交基$\li e,n$和任意$k\in\{1,\cdots,n\}$,由于$T$是正算子,于是
    \[\inprod{Te_k}{e_k}\geqslant0\]
    又因为$Te_k=\inprod{Te_k}{e_1}e_1+\cdots+\inprod{Te_k}{e_n}e_n$,于是
    \[\M(T,(\li e,n))_{k,k}=\inprod{Te_k}{e_k}\geqslant0\]
    于是此矩阵的对角线上均为非负数.\\
    $\Leftarrow$:由于$T$是自伴的,于是存在$V$的规范正交基$\li e,n$使得$T$关于其有对角矩阵$A$.\\
    根据题意,$A_{1,1},\cdots,A_{n,n}\geqslant0$,从而对于任意$v:=a_1e_1+\cdots+a_ne_n\in V$有
    \[\begin{aligned}
        \inprod{Tv}{v}
        &= \inprod{A_{1,1}a_1e_1}{a_1e_1}+\cdots+\inprod{A_{n,n}a_ne_n}{a_ne_n} \\
        &= A_{1,1}a_1^2+\cdots+A_{n,n}a_n^2 \\
        &\geqslant 0
    \end{aligned}\]
    从而$T$为正算子.
\end{proof}
\begin{problem}[6.]
    试证明:$V$上两正算子之和为正算子.
\end{problem}
\begin{proof}
    对于任意正的$S,T\in\L(V)$和任意$v\in V$有
    \[\inprod{(S+T)v}{v}=\inprod{Sv+Tv}{v}=\inprod{Sv}{v}+\inprod{Tv}{v}\geqslant 0\]
    于是$S+T$是正的.
\end{proof}
\begin{problem}[7.]
    设$S\in\L(V)$是可逆正算子,$T\in\L(V)$是正算子,试证明:$S+T$可逆.
\end{problem}
\begin{lemma}[Lemma.L.11]
    设$T\in\L(V)$是正算子,那么$T$可逆当且仅当$\inprod{Tv}{v}>0$对所有非零的$v\in V$成立.
\end{lemma}
\begin{proof}
    $\Rightarrow$:由于$T$可逆,于是对于任意非零的$v\in V$有$Tv\neq\mbf0$.于是$\inprod{Tv}{v}>0$.\\
    $\Leftarrow$:如果$T$不可逆,那么存在非零的$v\in V$使得$Tv=\mbf0$,从而$\inprod{Tv}{v}=0$,这与条件矛盾.于是$T$可逆.
\end{proof}
\begin{proof}
    根据\tbf{Lemma.L.11}可知$\inprod{Sv}{v}>0$对所有非零的$v\in V$成立.于是对于任意非零的$v\in V$有
    \[\inprod{(S+T)v}{v}=\inprod{Sv}{v}+\inprod{Tv}{v}>0\]
    从而$S+T$是可逆的.
\end{proof}
\begin{problem}[8.]
    设$T\in\L(V)$.试证明:$T$是正算子,当且仅当$T^\dagger$是正算子.
\end{problem}
\begin{proof}
    由于$T$是正算子,于是令$V$有一规范正交基$\li e,n$是分别对应于$T$的特征值$\li\lambda,n$的特征向量.\\
    于是$\li\lambda,n\geqslant0$.根据\tbf{7B.25}有
    \[T^\dagger e_k=\left\{\begin{array}{l}
        \frac{1}{\lambda_k}e_k,\lambda_k\neq0\\\mbf0,\lambda_k=0
    \end{array}\right.\]
    对所有$k\in\{1,\cdots,n\}$成立.于是$T^\dagger$关于$\li e,n$的矩阵是对角线上均为非负数的对角矩阵,从而$T^\dagger$是正算子.\\
    交换$T$和$T^\dagger$即可证得另一方向.于是命题得证.
\end{proof}
\begin{problem}[9.]
    设$T\in\L(V)$是正算子,$S\in\L(W,V)$.试证明:$S^*TS$是$W$上的正算子.
\end{problem}
\begin{proof}
    对于任意$w\in W$有
    \[\inprod{S^*TSw}{w}=\inprod{T(Sw)}{Sw}\geqslant0\]
    于是$S^*TS$是$W$上的正算子.
\end{proof}
\begin{problem}[10.]
    设$T$是$V$上的正算子,设$v,w\in V$满足$Tv=w$且$Tw=v$.试证明:$v=w$.
\end{problem}
\begin{proof}
    我们有
    \[\inprod{T(v-w)}{v-w}=\inprod{w-v}{v-w}=-||v-w||^2\geqslant0\]
    于是$||v-w||=0$,即$v=w$.
\end{proof}
\begin{problem}[11.]
    设$T$是$V$上的正算子,$U$是$V$在$T$下不变的子空间,试证明:$T|_U\in\L(U)$是$U$上的正算子.
\end{problem}
\begin{proof}
    对于任意$u\in U$都有$Tu\in U\subseteq V$,又因为$T$是正的,于是
    \[\inprod{T|_Uu}{u}=\inprod{Tu}{u}\geqslant0\]
    于是$T|_U$是$U$上的正算子.
\end{proof}
\begin{problem}[12.]
    设$T\in\L(V)$是正算子,试证明:对于任意$k\in\N^*$,$T^k$都是正算子.
\end{problem}
\begin{proof}
    由于$T$是正算子,于是$T$关于$V$的某组规范正交基$\li e,n$有对角矩阵
    \[\M(T)=\begin{pmatrix}
        \lambda_1&\cdots&0\\
        \vdots&\ddots&\vdots\\
        0&\cdots&\lambda_n
    \end{pmatrix}\]
    且$\li\lambda,n\geqslant0$.\\
    于是$T^k$关于这基的矩阵为
    \[\M(T^k)=(\M(T))^k=\begin{pmatrix}
        \lambda_1^k&\cdots&0\\
        \vdots&\ddots&\vdots\\
        0&\cdots&\lambda_n^k
    \end{pmatrix}\]
    这是对角线均为非负数的对角矩阵,从而$T^k$也是正算子.
\end{proof}
\begin{problem}[13.]
    设$T\in\L(V)$是自伴的,设$\alpha\in\R$.回答下列问题.
    \begin{enumerate}[label=\tbf{(\arabic*)}]
        \item 试证明:$T-\alpha I$是正算子,当且仅当$\alpha$不大于$T$的任意特征值.
        \item 试证明:$\alpha I-T$是正算子,当且仅当$\alpha$不小于$T$的任意特征值.
    \end{enumerate}
\end{problem}
\begin{proof}
    \begin{enumerate}[label=\tbf{(\arabic*)}]
        \item 由于$T$是自伴算子,于是考虑$V$的某个规范正交基$\li e,n$使得$T$关于其有对角矩阵
            \[\M(T)=\begin{pmatrix}
                \lambda_1&\cdots&0\\
                \vdots&\ddots&\vdots\\
                0&\cdots&\lambda_n
            \end{pmatrix}\]
            则$\li\lambda,n$为$T$的特征值.\\
            于是$T-\alpha I$关于这基的矩阵为
            \[\M(T-\alpha I)=\begin{pmatrix}
                \lambda_1-\alpha&\cdots&0\\
                \vdots&\ddots&\vdots\\
                0&\cdots&\lambda_n-\alpha
            \end{pmatrix}\]
            因为$\alpha\leqslant\displaystyle\min_{k\in\{1,\cdots,n\}}\{\lambda_k\}$,于是$\lambda_1-\alpha,\cdots,\lambda_n-\alpha\geqslant0$.\\
            于是$T-\alpha I$关于这基有对角线元素非负的对角矩阵,于是$T-\alpha I$是正算子.
        \item 与\tbf{(1)}类似,不再赘述.
    \end{enumerate}
\end{proof}
\begin{problem}[14.]
    设$T$是$V$上的正算子,$\li v,m\in V$.试证明
    \[\sum_{j=1}^{m}\sum_{k=1}^{m}\inprod{Tv_k}{v_j}\geqslant0\]
\end{problem}
\begin{proof}
    我们有
    \[\begin{aligned}
        \sum_{j=1}^{m}\sum_{k=1}^{m}\inprod{Tv_k}{v_j}
        &= \sum_{j=1}^{m}\inprod{Tv_1+\cdots+Tv_m}{v_j} \\
        &= \sum_{j=1}^{m}\inprod{T(\li v+m)}{v_j} \\
        &= \inprod{T(\li v+m)}{\li v+m} \\
        &\geqslant0
    \end{aligned}\]
\end{proof}
\begin{problem}[15.]
    设$T\in\L(V)$是自伴的,试证明:存在正算子$A,B\in\L(V)$使得
    \[T=A-B\text{\ \ 且\ \ }\sqrt{TT^*}=A+B\text{\ \ 且\ \ }AB=BA=\mbf0\]
\end{problem}
\begin{proof}
    由于$T$是自伴的,于是存在由$T$的特征向量$\li e,n,\li f,m$构成的$V$的规范正交基.\\
    不妨设它们对应的特征值为$\li\lambda,n,\li\mu,m$,其中$\li\lambda,n\geqslant0,\li\mu,m<0$.\\
    令$A,B\in\L(V)$满足对所有$k\in\{1,\cdots,n\}$和$j\in\{1,\cdots,m\}$有
    \[\left\{\begin{array}{l}
        Ae_k=\lambda_ke_k\\Af_j=\mbf0
    \end{array}\right.\ \ \ \ \ 
    \left\{\begin{array}{l}
        Be_k=\mbf0\\Bf_j=-\mu_jf_j
    \end{array}\right.\]
    考虑它们的矩阵,不难验证题设的条件成立.于是命题得证.
\end{proof}
\begin{problem}[16.]
    设$T$是$V$上的正算子.试证明:$\nul\sqrt{T}=\nul T$且$\range\sqrt{T}=\range T$.
\end{problem}
\begin{proof}
    注意到$\sqrt{T}$是正算子,于是$\sqrt{T}$是正规的.根据\tbf{7A.27}可知
    \[\nul T=\nul(\sqrt{T})^2=\nul\sqrt{T},\range T=\range(\sqrt{T})^2=\range\sqrt T\]
\end{proof}
\begin{problem}[17.]
    设$T\in\L(V)$是正算子,试证明:存在实系数多项式$p$使得$\sqrt{T}=p(T)$.
\end{problem}
\begin{proof}
    考虑$T$的特征向量$\li e,n$构成的$V$的规范正交基和对应的特征值$\li\lambda,n$.\\
    由于$T$是正算子,于是$\lambda_k\geqslant0$.令$p\in\P(\R)$满足
    \[p(\lambda_k)=\sqrt{\lambda_k},\forall k\in\{1,\cdots,n\}\]
    于是对于任意$k\in\{1,\cdots,n\}$有
    \[\sqrt{T}e_k=\sqrt{\lambda_k}e_k=p(\lambda_k)e_k=p(T)e_k\]
    于是$\sqrt{T}=p(T)$.
\end{proof}
\begin{problem}[18.]
    设$S$和$T$是$V$上的正算子.试证明:$ST$是正算子,当且仅当$S$和$T$可交换.
\end{problem}
\begin{proof}
    根据\tbf{7A.9},如果$S$和$T$不可交换,那么$ST$不是自伴算子,也就不是正算子.\\
    现在假设$S,T$可交换,于是根据\tbf{7B.16/17}可知存在$V$的规范正交基$\li e,n$使得$S,T$关于其有对角矩阵.\\
    不妨设
    \[\M(S)=\begin{pmatrix}
        \mu_1&\cdots&0\\
        \vdots&\ddots&\vdots\\
        0&\cdots&\mu_n
    \end{pmatrix}\ \ \ \ \ 
    \M(T)=\begin{pmatrix}
        \lambda_1&\cdots&0\\
        \vdots&\ddots&\vdots\\
        0&\cdots&\lambda_n
    \end{pmatrix}\]
    于是
    \[\M(TS)=\begin{pmatrix}
        \mu_1\lambda_1&\cdots&0\\
        \vdots&\ddots&\vdots\\
        0&\cdots&\mu_n\lambda_n
    \end{pmatrix}\]
    其中$\mu_1\lambda_1,\cdots,\mu_n\lambda_n\geqslant0$.于是$ST$是正算子.
\end{proof}
\begin{problem}[19.]
    试证明:$\F^2$上的恒等算子具有无穷多个自伴的平方根.
\end{problem}
\begin{proof}
    对于任意$t\in(0,1)$,令$T\in\L(\F^2)$关于其标准基的矩阵为
    \[\M(T)=\begin{pmatrix}
        \sqrt{1-t^2}&t\\t&-\sqrt{1-t^2}
    \end{pmatrix}\]
    不难验证$\M(T^2)=(\M(T))^2=I$,且$T$是自伴的.于是存在无穷多个$T$使得$T^2=I$.
\end{proof}
\begin{problem}[20.]
    设$T\in\L(V)$,$\li e,n$是$V$的规范正交基.试证明:$T$是正算子,当且仅当存在$\li v,n\in V$使得
    \[\inprod{Te_k}{e_j}=\inprod{v_k}{v_j}\]
    对所有$j,k=1,\cdots,n$成立.
\end{problem}
\begin{proof}
    $\Rightarrow$:设$T$是正算子,那么$\sqrt{T}$自伴.于是
    \[\inprod{Te_k}{e_j}=\inprod{\sqrt{T}\sqrt{T}e_k}{e_j}=\inprod{\sqrt{T}e_k}{\sqrt{T}e_j}\]
    对于任意$j,k\in\{1,\cdots,n\}$都成立.于是令$v_k=\sqrt{T}e_k$即可.\\
    $\Leftarrow$:设$R\in\L(V)$使得$Re_k=v_k$.于是
    \[\inprod{Te_k}{e_j}=\inprod{v_k}{v_j}=\inprod{Re_k}{Re_j}=\inprod{R^*Re_k}{e_j}\]
    于是$\M(T,(\li e,n))=\M(R^*R,(\li e,n))$,即$T=R^*R$.根据正算子的性质可知$T$是正算子.
\end{proof}
\begin{problem}[21.]
    设$n\in\N^*$.$n\times n$的\tbf{希尔伯特矩阵}是第$j$行第$k$列元素为$\dfrac{1}{j+k-1}$的$n\times n$矩阵.%
    设算子$T\in\L(V)$关于$V$的某个规范正交基的矩阵为$n\times n$阶希尔伯特矩阵,试证明:$T$是可逆正算子.
\end{problem}
\begin{proof}
    设这组规范正交基为$\li e,n$.考虑$v:=x_1e_1+\cdots+x_ne_n\in V$,我们有
    \[\begin{aligned}
        \inprod{Tv}{v}
        &= \inprod{\sum_{k=1}^{n}x_kTe_k}{\sum_{j=1}^{n}x_je_j} 
        = \sum_{j,k=1}^{n}\overline{x_j}x_k\inprod{Te_k}{e_j} \\
        &= \sum_{j,k=1}^{n}\dfrac{\overline{x_j}x_k}{j+k-1} 
        = \sum_{j,k=1}^{n}\overline{x_j}x_k\int_{0}^{1}t^{j+k-2}\di t \\
        &= \int_0^1\sum_{j,k=1}^{n}\overline{x_j}x_kt^{j+k-2}\di t \\
        &= \int_0^1\left(\sum_{j=1}^{n}\overline{x_j}t^{j-1}\right)\left(\sum_{k=1}^{n}x_kt^{k-1}\right)\di t \\
        &= \int_0^1\overline{\left(\sum_{j=1}^nx_jt^{j-1}\right)}\left(\sum_{k=1}^{n}x_kt^{k-1}\right)\di t \\
        &= \int_0^1\left|\sum_{k=1}^{n}x_kt^{k-1}\right|^2\di t
        \geqslant0
    \end{aligned}\]
    于是$T$是正算子.当$\inprod{Tv}{v}=0$时,要求
    \[\sum_{k=1}^{n}x_kt^{k-1}=0\]
    对任意$t\in[0,1]$成立.于是$\li x=n=0$,即$v=\mbf0$,从而$\nul T=\{\mbf0\}$,于是$T$可逆.\\
    因此$T$是可逆正算子.
\end{proof}
\begin{problem}[22.]
    设$T\in\L(V)$是正算子,$u\in V$满足$||u||=1$且$||Tu||\geqslant||Tv||$对所有满足$||v||=1$的$v\in V$成立.%
    试证明:$u$是$T$的特征向量,对应于$T$的最大的特征值.
\end{problem}
\begin{proof}
    考虑$T$的所有特征值$0\leqslant\li\lambda<n$.于是
    \[V=E(\lambda_1,T)\oplus E(\lambda_n,T)\]
    考虑$u=\li v+n$,其中$v_k\in E(\lambda_k,T)$.于是$1=||u||^2=||v_1||^2+\cdots+||v_n||^2$.于是
    \[||Tu||^2=\lambda_1^2||v_1||^2+\cdots+\lambda_n^2||v_n||^2\leqslant\lambda_n^2\left(||v_1||^2+\cdots+||v_n||^2\right)=\lambda_n^2\]
    现在考虑$w\in E(\lambda_n,T)$且$||w||=1$.根据题意,我们有
    \[||Tw||^2=\lambda_n^2\leqslant||Tu||^2\]
    于是$||Tu||=\lambda_n$.考虑第一个不等式的取等条件,当且仅当$\lambda_1=\cdots=\lambda_{n-1}=0$时成立,于是$u=v_n\in E(\lambda_n,T)$.\\
    于是$u$是$T$的特征向量,对应于$T$最大的特征值.
\end{proof}
\begin{problem}[23.]
    对于$T\in\L(V)$和$u,v\in V$,定义$\inprod{u}{v}_T$为$\inprod uv_T=\inprod{Tu}{v}$.回答下列问题.
    \begin{enumerate}[label=\tbf{(\arabic*)}]
        \item 设$T\in\L(V)$.试证明:$\inprod{\cdot}{\cdot}_T$是$V$上的内积,当且仅当$T$是关于原内积的可逆正算子.
        \item 试证明:$V$上的任意内积都具有$\inprod{\cdot}{\cdot}_T$的形式,其中$T$是某个$V$上的可逆正算子.
    \end{enumerate}
\end{problem}
\begin{proof}
    \begin{enumerate}[label=\tbf{(\arabic*)}]
        \item $\Leftarrow$:我们依次证明$\inprod\cdot\cdot_T$具有内积的性质.
            \begin{enumerate}[label=\tbf{(\alph*)}]
                \item \tbf{正性}:由于$T$是正算子,于是对于任意$v\in V$有
                    \[\inprod{v}{v}_T=\inprod{Tv}{v}\geqslant0\]
                \item \tbf{定性}:由于$\inprod{Tv}{v}=0$当且仅当$v=\mbf0$,于是$\inprod{v}{v}_T=0$当且仅当$v=\mbf0$.
                \item \tbf{可加性}:对于任意$u,v,w\in V$有
                    \[\inprod{u+v}{w}_T=\inprod{T(u+v)}{w}=\inprod{Tu+Tv}{w}=\inprod{Tu}{w}+\inprod{Tv}{w}=\inprod{u}{w}_T+\inprod vw_T\]
                \item \tbf{齐次性}:对于任意$\lambda\in\F$和任意$u,v\in V$有
                    \[\inprod{\lambda u}{v}_T=\inprod{T(\lambda u)}{v}=\inprod{\lambda Tu}{v}=\lambda\inprod{Tu}{v}=\lambda\inprod{u}{v}_T\]
                \item \tbf{共轭对称性}:由于$T$是正算子,于是$T$自伴,于是对于任意$u,v\in V$有
                    \[\inprod uv_T=\inprod {Tu}v=\overline{\inprod{v}{Tu}}=\overline{\inprod{Tv}{u}}=\overline{\inprod{v}{u}_T}\]
            \end{enumerate}
            综上可以得出$\inprod\cdot\cdot_T$是$V$上的内积.\\
            $\Rightarrow$:对于任意$u,v\in V$有
            \[\inprod{Tu}{v}=\inprod{u}{v}_T=\overline{\inprod vu_T}=\overline{\inprod{Tv}{u}}=\inprod{u}{Tv}\]
            从而$T$是自伴的.对于任意$v\in V$,我们有
            \[\inprod{Tv}{v}=\inprod vv_T\geqslant0\]
            当且仅当$v=\mbf0$时等式成立.于是根据\tbf{Lamma.L.11}可知$T$是可逆正算子.
        \item 考虑$V$原有的内积$\inprod\cdot\cdot_1$和任意的内积$\inprod\cdot\cdot_2$.\\
            考虑$V$关于$\inprod\cdot\cdot_1$的规范正交基$\li e,n$和关于$\inprod\cdot\cdot_2$的规范正交基$\li f,n$.\\
            定义$R\in\L(V)$使得$Rf_k=e_k$对任意$k\in\{1,\cdots,n\}$都成立.由于$R$是基到基的映射,因此$R$可逆.\\
            现在令$T=R^*R$,由于$R$可逆,那么$T$也可逆.根据正算子的性质可知$T$是正算子.\\
            现在,对于任意$u:=a_1f_1+\cdots+a_nf_n$和$v:=b_1f_1+\cdots+b_nf_n$,我们有
            \[\inprod{u}{v}_2=a_1\overline{b_1}+\cdots+a_n\overline{b_n}=\inprod{Ru}{Rv}_1=\inprod{R^*Ru}{v}_1=\inprod{Tu}{v}_1\]
            从而存在可逆正算子$T\in\L(V)$使得$\inprod\cdot\cdot_2=\inprod\cdot\cdot_T$.
    \end{enumerate}
\end{proof}
\begin{problem}[24.]
    设$S,T$是$V$上的正算子.试证明:$\nul(S+T)=\nul S\cap\nul T$.
\end{problem}
\begin{problem}
    根据\tbf{7C.6}可知$S+T$也是正算子.于是对于$v\in V$有
    \[\begin{aligned}
        v\in\nul(S+T)
        &\Leftrightarrow\inprod{(S+T)v}{v}=0
        \Leftrightarrow\inprod{Sv}{v}+\inprod{Tv}{v}=0 \\
        &\Leftrightarrow\inprod{Sv}{v}=\inprod{Tv}{v}=0
        \Leftrightarrow Sv=Tv=\mbf0 \\
        &\Leftrightarrow v\in\nul S\cap\nul T
    \end{aligned}\]
    于是$\nul(S+T)=\nul S\cap\nul T$.
\end{problem}
\begin{problem}[25.]
    令$T$是\tbf{7A.31(2)}中的二阶求导算子.试证明:$-T$是正算子.
\end{problem}
\begin{proof}
    设$D$是\tbf{7A.31(1)}中的一阶求导算子,于是$T=D^2$且$D^*=-D$.于是对于任意$f\in V$有
    \[\inprod{-Tf}{f}=\inprod{-D^2f}{f}=\inprod{D^*Df}{f}=\inprod{Df}{Df}=||Df||^2\geqslant0\]
    于是$-T$是$V$上的正算子.
\end{proof}
\end{document}