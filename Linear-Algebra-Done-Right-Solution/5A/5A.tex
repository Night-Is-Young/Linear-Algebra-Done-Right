\documentclass{ctexart}
\usepackage{geometry}
\usepackage[dvipsnames,svgnames]{xcolor}
\usepackage[strict]{changepage}
\usepackage{framed}
\usepackage{enumerate}
\usepackage{amsmath,amsthm,amssymb}
\usepackage{enumitem}
\usepackage{solution}

\allowdisplaybreaks
\geometry{left=2cm, right=2cm, top=2.5cm, bottom=2.5cm}

\begin{document}\pagestyle{empty}
\begin{center}
    \large\tbf{Linear Algebra Done Right 5A}
\end{center}
\begin{problem}[1.]
    设$T\in\mathcal{L}(V)$且$U$是$V$的子空间.
    \begin{enumerate}[label=\tbf{(\arabic*)}]
        \item 证明:如果$U\subseteq\nul T$,那么$U$在$T$下不变.
        \item 证明:如果$\range T\subseteq U$,那么$U$在$T$下不变.
    \end{enumerate}
\end{problem}
\begin{proof}
    \begin{enumerate}[label=\tbf{(\arabic*)}]
        \item 对于任意$u\in U\subseteq\nul T$,都有$Tu=\mbf{0}\in U$,于是$U$在$T$下不变.
        \item 对于任意$u\in U\subseteq V$,都有$Tu\in\range T\subseteq U$,于是$U$在$T$下不变.
    \end{enumerate}
\end{proof}
\begin{problem}[2.]
    设$T\in\mathcal{L}(V)$且$\li V,m$是$V$在$T$下的不变子空间.证明$\li V+m$在$T$下不变.
\end{problem}
\begin{proof}
    对于任意$v=\li v+m,v_k\in V_k$有
    $$Tv=T\left(\li v+m\right)=\li {Tv}+m$$
    而$Tv_k\in V_k$,于是$\li {Tv}+m\in\li V+m$.于是$\li V+m$在$T$下不变.
\end{proof}
\begin{problem}[3.]
    设$T\in\mathcal{L}(V)$.证明$V$的任意一族$T$下的不变子空间的交集在$T$下不变.
\end{problem}
\begin{proof}
    设$\li V,m$在$T$下不变.\\
    对于任意$\displaystyle v\in\bigcap_{j=1}^{m}V_j$和任意$1\leqslant k\leqslant m$有$v\in V_k$,于是$Tv\in V_k$.\\
    这表明$\displaystyle Tv\in\bigcap_{j=1}^{m}V_j$,从而$\displaystyle \bigcap_{j=1}^{m}V_j$在$T$下不变.
\end{proof}
\begin{problem}[4.]
    证明或给出一反例:若$V$是有限维的,其子空间$U$在$V$上任意算子下均不变,那么$U=\left\{\mbf{0}\right\}\text{或}V$.
\end{problem}
\begin{proof}
    当$U=\left\{\mbf{0}\right\}\text{或}V$,不难验证它们在任意算子下不变.\\
    若$U\neq\left\{\mbf{0}\right\}\text{或}V$,那么假定$U$的基为$\li u,m$,将其扩展为$V$的一组基$\li u,m,\li v,n$.\\
    根据上面的假设,$m\geqslant 1$且$n\geqslant 1$.定义$T\in\mathcal{L}(V)$为
    $$\left\{\begin{array}{l}
        Tu_1=v_1\\
        Tv_1=u_1\\
        Tu_k=u_k,2\leqslant k\leqslant m \\
        Tv_j=v_j,2\leqslant j\leqslant n
    \end{array}\right.$$
    于是$Tu_1=v_1\notin U$,从而这样的$U$不能在任意$T$下不变.\\
    于是命题得证.
\end{proof}
\begin{problem}[5.]
    设$T\in\mathcal{L}(\R^2)$定义为$T(x,y)=(-3y,x)$,求$T$的特征值.
\end{problem}
\begin{solution}[Solution.]
    设$(x,y)\neq(0,0)$满足$T(x,y)=\lambda(x,y)$,即
    $$\left\{\begin{array}{l}
        -3y=\lambda x\\
        x=\lambda y
    \end{array}\right.$$
    变形可得$\lambda^2+3=0$.这方程没有实根,于是$T$不存在实特征值.
\end{solution}
\begin{problem}[6.]
    定义$T\in\mathcal{L}(\F^2)$为$T(w,z)=(z,w)$.求$T$的所有特征值和对应的特征向量.
\end{problem}
\begin{solution}[Solution.]
    设$(w,z)\neq(0,0)$满足$T(w,z)=\lambda(z,w)$,即
    $$\left\{\begin{array}{l}
        z=\lambda w\\
        w=\lambda z
    \end{array}\right.$$
    变形可得$\lambda^2-1=0$,即$\lambda=\pm 1$.\\
    于是$T$的特征值为$1$和$-1$,对应的特征向量为$(w,w)$和$(w,-w)$,其中$w\neq 0\in\F$.
\end{solution}
\begin{problem}[7.]
    定义$T\in\mathcal{L}(\F^3)$为$T(z_1,z_2,z_3)=(2z_2,0,5z_3)$.求$T$的所有特征值和对应的特征向量.
\end{problem}
\begin{solution}[Solution.]
    设$(z_1,z_2,z_3)\neq(0,0,0)\in\F^3$满足$T(z_1,z_2,z_3)=\lambda(z_1,z_2,z_3)$,即
    $$\left\{\begin{array}{l}
        \lambda z_1=2z_2 \\
        \lambda z_2=0 \\
        \lambda z_3=5z_3
    \end{array}\right.$$
    这方程组的解为$z_1=z_2=0,\lambda=5$.于是$T$的特征值为$5$,特征向量为$(0,0,z_3)$,其中$z_3\in\F$.
\end{solution}
\begin{problem}[8.]
    设$P\in\mathcal{L}(V)$且$P^2=P$.证明:若$\lambda$是$P$的特征值,那么$\lambda=0\text{或}1$.
\end{problem}
\begin{proof}
    设$v\neq\mbf{0}\in V$使得$Pv=\lambda v$.于是$(P^2)(v)=P(P(v))=P(\lambda v)=\lambda Pv=\lambda^2 v$.\\
    由$P=P^2$可知$\lambda v=\lambda^2 v$,又$v\neq\mbf{0}$,于是$\lambda^2-\lambda=0$,即$\lambda=0\text{或}1$.
\end{proof}
\begin{problem}[9.]
    定义$T:\mathcal{P}(\R)\to\mathcal{P}(\R)$为$Tp=p'$.求出$T$的所有特征值和对应的特征向量.
\end{problem}
\begin{proof}
    设$T$的特征值为$\lambda$,特征向量为$p$,于是$Tp=\lambda p=p'$.\\
    设$p=a_0+a_1x+\cdots+a_mx^m$,其中$a_m\neq 0$.于是$p'=a_1+2a_2x+\cdots+ma_mx^{m-1}$.\\
    若$\lambda=0$,则$p'=\mbf{0}$,这要求$p$为任意常值函数.\\
    若$\lambda\neq 0$且$m\geqslant 1$,那么$\deg(\lambda p)=m>m-1=\deg p'$,于是不存在这样的$p$使得式子成立.\\
    综上可知$T$的特征值为$0$,特征向量为常值多项式$t$,$t\in\F$.
\end{proof}
\begin{problem}[10.]
    定义$T\in\mathcal{L}(\mathcal{P}_4(\R))$为$(Tp)(x)=xp'(x)$对所有$x\in\R$成立.求出$T$的所有特征值和对应的特征向量.
\end{problem}
\begin{proof}
    设$T$的特征值为$\lambda$,特征向量为$p\neq\mbf{0}$.于是对于任意$x\in\R$有$(Tp)(x)=(\lambda p)(x)=xp'(x)$.\\
    即$\lambda p(x)=xp'(x)$对所有$x\in\R$成立.设$p=a_0+a_1x+a_2x^2+a_3x^3+a_4x^4$,其中各$a$不全为$0$.\\
    于是我们有$$\lambda\left(a_0+a_1x+a_2x^2+a_3x^3+a_4x^4\right)=a_1x+2a_2x^2+3a_3x^3+4a_4x^4$$
    于是$$\left\{\begin{array}{l}
        \lambda a_0=0\\
        \lambda a_1=a_1\\
        \lambda a_2=2a_2\\
        \lambda a_3=3a_3\\
        \lambda a_4=4a_4
    \end{array}\right.$$
    于是$\lambda=1,2,3,4$.\\
    综上可知$T$的特征值为$1,2,3$和$4$,对应的特征向量分别为$a_1x,a_2x^2,a_3x^3$和$a_4x^4$,其中各$a_k\neq0\in\F$.
\end{proof}
\begin{problem}[11.]
    设$V$是有限维的,$T\in\mathcal{L}(V)$且$\alpha\in\F$.证明:存在$\delta>0$使得对所有满足$0<\left|\alpha-\lambda\right|<\delta$的$\lambda\in\F$,都有$T-\lambda I$可逆.
\end{problem}
\begin{proof}
    若$T$没有特征值,那么对于任意$\lambda\in\F$,$T-\lambda I$都是可逆的,这时任取$\delta>0$即可.\\
    若$T$有特征值,不妨设为$\li\lambda,m$,其中$m\leqslant \dim V$.\\
    令$\displaystyle\delta=\min_{1\leqslant k\leqslant m,\lambda_k\neq\alpha}\left|\alpha-\lambda_k\right|$.于是对于任意$\lambda\in(\alpha-\delta,\alpha)\cup(\alpha,\alpha+\delta)$,都有$\lambda\neq\lambda_k$.\\
    这就表明$\lambda$不是$T$的特征值,于是$T-\lambda I$可逆.
\end{proof}
\begin{problem}[12.]
    设$V=U\oplus W$,其中$U$和$W$都是$V$的非零子空间.定义$P\in\L(V)$为,对任意$u\in U$和$w\in W$都有$P(u+w)=u$.求出$P$的所有特征值和对应的特征向量.
\end{problem}
\begin{proof}
    设$P$的特征值为$\lambda$.设$v:=u+w\in V$且$v\neq\mbf{0}$是对应的特征向量,于是$Pv=\lambda v$.\\
    因此$P(u+w)=\lambda(u+w)=u$,即$(1-\lambda)u=\lambda w$.\\
    注意到$U+W$是直和,于是上式成立当且仅当$(1-\lambda)u=\lambda w=\mbf{0}$.\\
    于是$P$的特征值为$0,1$对应的特征向量为$w,u$,其中$w\in W,u\in U$且$w,u\neq\mbf{0}$.
\end{proof}
\begin{problem}[13.]
    设$T\in\L(V)$,并设$S\in\L(V)$可逆.
    \begin{enumerate}[label=\tbf{(\arabic*)}]
        \item 证明$T$和$S^{-1}TS$具有相同的特征值.
        \item 说明$T$和$S^{-1}TS$的特征向量间的关系.
    \end{enumerate}
\end{problem}
\begin{proof}
    \begin{enumerate}[label=\tbf{(\arabic*)}]
        \item 设$T$的特征值为$\lambda$,对应的特征向量为$v$.由于$S$可逆,不妨设$u\in V$使得$Su=v$.\\
            于是$(S^{-1}TS)u=S^{-1}T(Su)=S^{-1}(Tv)=S^{-1}(\lambda v)=\lambda(S^{-1}v)=\lambda u$.\\
            这表明$S^{-1}TS$也有特征值$\lambda$,对应的特征向量为$S^{-1}v$.
        \item 见\tbf{(1)}的论述.
    \end{enumerate}
\end{proof}
\begin{problem}[14.]
    给出一例$\R^4$上没有实特征值的算子.
\end{problem}
\begin{solution}[Solution.]
    定义$T\in\L(\R^4):(a,b,c,d)\mapsto(-b,a,c,d)$.这$T$就没有实特征值.
\end{solution}
\begin{problem}[15.]
    设$V$是有限维的,$T\in\L(V),\lambda\in\F$.证明$\lambda$是$T$的特征值当且仅当$\lambda$是对偶算子$T'\in\L(V')$的特征值.
\end{problem}
\begin{proof}
    $$\begin{aligned}
        T\text{有特征值}\lambda
        &\Leftrightarrow (T-\lambda I)\text{不可逆} \\
        &\Leftrightarrow (T-\lambda I)'\text{不可逆} \\
        &\Leftrightarrow T'-\lambda I'\text{不可逆} \\
        &\Leftrightarrow T'\text{有特征值}\lambda
    \end{aligned}$$
\end{proof}
\begin{problem}[16.]
    设$\li v,n$是$V$的基,$T\in\L(V)$.证明:如果$\lambda$是$T$的特征值,那么
    $$\left|\lambda\right|\leqslant n\max_{1\leqslant j,k\leqslant n}\left|\mathcal{M}(T)_{j,k}\right|$$
\end{problem}
\begin{proof}
    记$\mathcal{M}(T)=A$.设$v:=a_1v_1+\cdots+a_nv_n\in V$是$T$对应$\lambda$的特征向量.\\
    于是$$Tv=\sum_{k=1}^{n}a_kTv_k=\sum_{k=1}^{n}\left(a_k\sum_{j=1}^{n}A_{j,k}v_j\right)=\sum_{j=1}^{n}\left(\sum_{k=1}^{n}a_kA_{j,k}\right)v_j=\sum_{j=1}^{n}\lambda a_jv_j$$
    即$\lambda=\dfrac{\displaystyle\sum_{k=1}^{n}a_kA_{j,k}}{a_j}$对任意$1\leqslant j\leqslant n$都成立.%
    于是$\displaystyle\left|\lambda\right|\left|a_j\right|\leqslant\sum_{k=1}^{n}\left|A_{j,k}\right|\left|a_k\right|$.
    取$a_j$使得$\displaystyle\left|a_j\right|=\max_{1\leqslant j\leqslant n}\left|b_j\right|$.于是
    $$|\lambda|\leqslant\sum_{k=1}^{n}\left|A_{j,k}\right|\left|\dfrac{a_k}{a_j}\right|\leqslant\sum_{k=1}^{n}\left|A_{j,k}\right|\leqslant n\max_{1\leqslant j,k\leqslant n}\left|\mathcal{M}(T)_{j,k}\right|$$
\end{proof}
\begin{problem}[17.]
    设$\F=\R$,$T\in\L(V)$,$\lambda\in\R$.证明:$\lambda$是$T$的特征值,当且仅当$\lambda$是复化$T_\C$的特征值.
\end{problem}
\begin{proof}
    $\Rightarrow$:设$v\in V$是对应$\lambda$的特征向量.存在$v+v\i\in V_\C$使得
    $$T(v+v\i)=Tv+Tv\i=\lambda v+\lambda v\i=\lambda(v+v\i)$$
    于是$\lambda$为$T_\C$的特征值.\\
    $\Leftarrow$:设$v+u\i\in V_\C$是对应$\lambda$的特征向量.于是存在$u,v\in V$使得
    $$T(v+u\i)=Tv+Tu\i=\lambda(v+u\i)$$
    于是$Tv=\lambda v,Tu=\lambda u$,即$\lambda$是$T$的特征值.
\end{proof}
\begin{problem}[18.]
    设$\F=\R$,$T\in\L(V)$,$\lambda\in\C$.证明:$\lambda$是$T_\C$的特征值,当且仅当$\overline{\lambda}$是$T_\C$的特征值.
\end{problem}
\begin{proof}
    设$v+u\i$是$T_\C$对应$\lambda$的特征向量.不妨设$\lambda=a+b\i$.
    $$T(v+u\i)=Tv+Tu\i=\lambda(v+u\i)=(a+b\i)(v+u\i)=(av-bu)+(au+bv)i$$
    于是$Tv=av-bu,Tu=au+bv$.因此有
    $$T(u+v\i)=Tu+Tv\i=(au+bv)+(av-bu)\i=(a-b\i)(u+v\i)=\overline{\lambda}(u+v\i)$$
    于是$\overline{\lambda}$为$T_\C$的特征值.反之亦同理.
\end{proof}
\begin{problem}[19.]
    证明:定义为$T(z_1,z_2,\cdots)=(0,z_1,z_2,\cdots)$的前向位移算子$T\in\L(\F^\infty)$没有特征值.
\end{problem}
\begin{proof}
    假定$T$有特征值$\lambda\in\F$,对应的特征向量为$v$.那么有$Tz_1=0=\lambda z_1,Tz_k=z_{k-1}=\lambda z_k(\forall k\geqslant 2)$.\\
    若$z_1=0$,则对任意$k\in\N^*$都有$z_k=0$,于是$v=\mbf{0}$,舍去.\\
    若$\lambda=0$,则对于任意$k\in\N^*$有$z_k=\lambda z_{k+1}=0$,于是$v=\mbf{0}$,舍去.\\
    综上可知$T$没有特征值.
\end{proof}
\begin{problem}[20.]
    定义后向位移算子$S\in\L(\F^\infty)$为$S(z_1,z_2,z_3,\cdots)=(z_2,z_3,\cdots)$.
    \begin{enumerate}[label=\tbf{(\arabic*)}]
        \item 证明$\F$的任意元素均为$S$的特征值.
        \item 求出$S$的所有特征向量.
    \end{enumerate}
\end{problem}
\begin{proof}
    \begin{enumerate}[label=\tbf{(\arabic*)}]
        \item 对任意$\lambda\in\F$,都有$v=(1,\lambda,\lambda^2,\cdots)$满足
            $$Tv=(\lambda,\lambda^2,\cdots)=\lambda v$$
            于是命题得证.
        \item 对于任意$c\in\F$且$c\neq 0$和任意$k\in\mathbf{Z}$,$v=(c\lambda^k,c\lambda^{k+1},\cdots)$均为$S$对应$\lambda$的特征向量.
    \end{enumerate}
\end{proof}
\begin{problem}[21.]
    设$T\in\L(V)$可逆.
    \begin{enumerate}[label=\tbf{(\arabic*)}]
        \item 设$\lambda\in\F$且$\lambda\neq0$.证明:$\lambda$是$T$的特征值当且仅当$\dfrac{1}{\lambda}$是$T^{-1}$的特征值.
        \item 证明:$T$和$T^{-1}$的特征向量相同.
    \end{enumerate}
\end{problem}
\begin{proof}
    \begin{enumerate}[label=\tbf{(\arabic*)}]
        \item 设$v$是$T$对应$\lambda$的特征向量,即$Tv=\lambda v$.于是
            $$T^{-1}v=\dfrac{1}{\lambda}T^{-1}(\lambda v)=\dfrac{1}{\lambda}T^{-1}Tv=\dfrac{1}{\lambda}v$$
            于是$\dfrac{1}{\lambda}$是$T^{-1}$的特征值,特征向量为$v$.\\
            至于另一方向的蕴含关系,注意到$\dfrac{1}{\frac{1}{\lambda}}=\lambda,\left(T^{-1}\right)^{-1}=T$即可得证.
        \item 见\tbf{(1)}.
    \end{enumerate}
\end{proof}
\begin{problem}[22.]
    设$T\in\L(V)$且存在$u,w\in V(v,w\neq\mbf{0})$使得$Tu=3w,Tw=3u$.证明$3$或$-3$是$T$的特征值.
\end{problem}
\begin{proof}
    由题意$Tu+Tw=T(u+w)=3(u+w)$.\\
    若$u+w\neq\mbf{0}$,那么$T$的特征值为$3$,对应的特征向量为$u+w$.\\
    若$u+w=\mbf{0}$,那么$Tu=3w=-3u$,从而$T$的特征值为$-3$,对应的特征向量为$u$和$w$.
\end{proof}
\begin{problem}[23.]
    设$V$是有限维的,$S,T\in\L(V)$.证明:$ST$与$TS$的特征值相同.
\end{problem}
\begin{proof}
    设$\lambda$为$ST$的特征值,对应的特征向量为$v$.于是$STv=\lambda v$.\\
    于是$TS(Tv)=T(STv)=T(\lambda v)=\lambda Tv$.于是$TS$的特征值为$\lambda$,对应的特征向量为$Tv$.
\end{proof}
\begin{problem}[24.]
    设$A\in\F^{n,n}$,定义$T\in\L(\F^{n,1})$为$Tx=Ax$.
    \begin{enumerate}[label=\tbf{(\arabic*)}]
        \item 设$A$的每一行元素之和均为$1$.证明:$1$是$T$的特征值.
        \item 设$A$的每一列元素之和均为$1$.证明:$1$是$T$的特征值.
    \end{enumerate}
\end{problem}
\begin{proof}
    \begin{enumerate}[label=\tbf{(\arabic*)}]
        \item 取$v=\begin{pmatrix}1\\\vdots\\1\end{pmatrix}\in\F^{n,1}$.于是
            \[Tv=Av=\begin{pmatrix}
                \displaystyle\sum_{k=1}^{n}A_{1,k}\\
                \vdots\\
                \displaystyle\sum_{k=1}^{n}A_{n,k}
            \end{pmatrix}
            =\begin{pmatrix}
                1\\\vdots\\1
            \end{pmatrix}
            =v\]
            于是$1$是$T$的特征值.
        \item 记$B=\mathcal{M}(T')=A^\text{t}$.根据\tbf{(1)}可知$T'$有特征值$1$,根据\tbf{15.}可知$T$有特征值$1$.
    \end{enumerate}
\end{proof}
\begin{problem}[25.]
    设$T\in\L(V)$,$u,w$是$T$的特征向量且$u+w$也是$T$的特征向量.证明:$u,w$是$T$对应于同一特征值的特征向量.
\end{problem}
\begin{proof}
    假设$u,w,u+w$对应的特征值为$\lambda,\mu,\xi$.于是$Tu=\lambda u,Tw=\mu w$.\\
    于是$T(u+w)=\lambda u+\mu w=\xi(u+w)$,即$(\lambda-\xi)u=(\xi-\mu)w$.\\
    又$u,w\neq\mbf{0}$.若$\lambda-\xi,\xi-\mu\neq0$,那么有$w=\dfrac{\lambda-\xi}{\xi-\mu}u$.于是
    $$Tw=\mu w=\mu\cdot\dfrac{\lambda-\xi}{\xi-\mu}u=T\left(\dfrac{\lambda-\xi}{\xi-\mu}u\right)=\lambda\cdot\dfrac{\lambda-\xi}{\xi-\mu}u$$
    于是$\lambda=\mu$,从而$u,w$对应于同一特征值.\\
    若$\lambda-\xi=\xi-\mu=0$,即$\lambda=\xi=\mu$,于是$u,w$对应于同一特征值.\\
    命题得证.
\end{proof}
\begin{problem}[26.]
    设$T\in\L(V)$使得$V$中任意非零向量都是$T$的特征向量.证明:$T$是恒等算子的标量倍.
\end{problem}
\begin{proof}
    任取$u,w\neq\mbf{0}\in V$且$u+w\neq\mbf{0}$.则$u+w$也是$T$的特征向量.据\tbf{25.}可知$u,w$对应于同一特征值$\lambda$.\\
    于是对于任意$v\in V$有$Tv=\lambda v$,从而$(T-\lambda I)v=\mbf{0}$,于是$T=\lambda I$.命题得证.
\end{proof}
\begin{problem}[27.]
    设$V$是有限维的,且$k\in\left\{1,\cdots,\dim V-1\right\}$.设$T\in\L(V)$使得$V$的任意$k$维子空间都在$T$下不变.证明:$T$是恒等算子的标量倍.
\end{problem}
\begin{proof}
    
\end{proof}
\begin{problem}[28.]
    设$V$是有限维的,$T\in\L(V)$.证明:$T$最多有$1+\dim\range T$个不同的特征值.
\end{problem}
\begin{proof}
    设$T$的特征值为$\li \lambda,n$.设$\li v,n$满足$Tv_k=\lambda_kv_k$.\\
    由于不同特征值对应的向量组线性无关,于是$\li v,k$线性无关.\\
    由于至多存在一个$1\leqslant k\leqslant n$使得$\lambda_k=0$,于是$\span(\lambda_1v_1,\cdots,\lambda_nv_n)$至少是维度为$n-1$的空间.\\
    于是$n-1\leqslant\dim\range T$,即$n\leqslant1+\dim\range T$.
\end{proof}
\begin{problem}[29.]
    设$T\in\L(\R^3)$,且有特征值$-4,5,\sqrt{7}$.证明:存在$x\in\R^3$使得$Tx-9x=(-4,5\sqrt{7})$.
\end{problem}
\begin{proof}
    由于$T\in\L(\R^3)$最多有$\dim\R^3=3$个不同的特征值,于是$9$不是$T$的特征值.\\
    这等价于$T-9I$可逆,于是一定存在$x\in\R^3$使得$(T-9I)x=(-4,5,\sqrt{7})$.
\end{proof}
\begin{problem}[30.]
    设$T\in\L(V)$,且$(T-2I)(T-3I)(T-4I)=\mbf{0}$.设$\lambda$为$T$的特征值,证明:$\lambda=2,3\text{或}4$.
\end{problem}
\begin{proof}
    由$(T-2I)(T-3I)(T-4I)=\mbf{0}$可知对于任意$v\in V$有$(T-2I)(T-3I)(T-4I)v=\mbf{0}$.\\
    若$(T-4I)v=\mbf{0}$,则$\lambda=4$为$T$的特征值,对应的特征向量为$v$.\\
    若$(T-4I)v\neq\mbf{0}$,不妨令$u=(T-4I)v$.\\
    若$(T-3I)u=\mbf{0}$,则$\lambda=3$为$T$的特征值,对应的特征向量为$u$.\\
    若$(T-3I)u\neq\mbf{0}$,则$\lambda=2$为$T$的特征值,对应的特征向量为$(T-3I)u=(T-3I)(T-2I)v$.\\
    综上,命题得证.
\end{proof}
\begin{problem}[31.]
    给出一例$T\in\L(\R^2)$使得$T^4=-I$.
\end{problem}
\begin{solution}[Solution.]
    令$T$满足对任意$(x,y)\in\R^2$有$T(x,y)=\left(\dfrac{x-y}{\sqrt{2}},\dfrac{x+y}{\sqrt{2}}\right)$即可.这实际上是将向量逆时针旋转$\dfrac{\pi}{4}$.
\end{solution}
\begin{problem}[32.]
    设$T\in\L(V)$没有特征值且$T^4=I$.证明$T^2=-I$.
\end{problem}
\begin{proof}
    由题意$T^4-I=(T^2+I)(T+I)(T-I)=\mbf{0}$.\\
    由于$T$没有特征值,于是$T+I,T-I$均为可逆算子.于是$T^2+I=\mbf{0}$,即$T^2=-I$.
\end{proof}
\begin{problem}[33.]
    设$T\in\L(V)$且$m\in\N^*$.
    \begin{enumerate}[label=\tbf{(\arabic*)}]
        \item 证明:$T$是单射,当且仅当$T^m$是单射.
        \item 证明:$T$是满射,当且仅当$T^m$是满射.
    \end{enumerate}
\end{problem}
\begin{proof}
    \begin{enumerate}[label=\tbf{(\arabic*)}]
        \item $\Rightarrow$:由于$T$是单射,于是$\nul T=\{\mbf0\}$.于是$T^2v=T(Tv)=\mbf{0}$当且仅当$Tv=\mbf0$,即$v=\mbf{0}$.这表明$T^2$是单射.依次类推可知$T^m$是单射.\\
            $\Leftarrow$:若$T$不是单射,于是存在$v\neq\mbf0$使得$Tv=\mbf0$,即$T^mv=T^{m-1}(Tv)=T^{m-1}\mbf0=\mbf0$,进而$T^m$不是单射,这与假设不符,于是$T$是单射.
        \item $T\text{是满射}\Leftrightarrow T\text{是单射}\Leftrightarrow T^m\text{是单射}\Leftrightarrow T^m\text{是满射}$.
    \end{enumerate}
\end{proof}
\begin{problem}[34.]
    设$V$是有限维的,$\li v,m\in V$.证明:$\li v,m$线性无关当且仅当存在$T\in\L(V)$使得$\li v,m$是$T$对应于不同特征值的特征向量.
\end{problem}
\begin{proof}
    $\Rightarrow$:设$\li\lambda,m$满足$Tv_k=\lambda_kv_k$,各$\lambda_k$不相同.\\
    将$\li v,m$扩展为$V$的一组基$\li v,m,\li u,n$,令任意$1\leqslant j\leqslant n$有$Tu_j=\mbf0$.\\
    这就定义了$T\in\L(V)$满足题意.\\
    $\Leftarrow$:我们已经证明了不同特征值对应的特征向量构成的组线性无关.
\end{proof}
\begin{problem}[35.]
    设$\li\lambda,n$是一组相异实数.证明:$\e^{\lambda_1 x},\cdots,\e^{\lambda_n x}$在由$\R$上的实值函数构成的向量空间中线性无关.
\end{problem}
\begin{proof}
    令$V=\span(\e^{\lambda_1 x},\cdots,\e^{\lambda_n x})$.定义$D\in\L(V)$为$Df=f'$.\\
    于是对于任意$1\leqslant k\leqslant n$有$D\e^{\lambda_k x}=\lambda_k\e^{\lambda_k x}$.故$\lambda_k$是$T$的特征值,特征向量为$\e^{\lambda_k x}$.\\
    又$\li\lambda,n$相异,于是$\e^{\lambda_1 x},\cdots,\e^{\lambda_n x}$线性无关.
\end{proof}
\begin{problem}[36.]
    设$\li\lambda,n$是一组相异正数.证明:$\cos{\lambda_1 x},\cdots,\cos{\lambda_n x}$在由$\R$上的实值函数构成的向量空间中线性无关.
\end{problem}
\begin{proof}
    令$V=\span(\cos{\lambda_1 x},\cdots,\cos{\lambda_n x})$.定义$D\in\L(V)$为$Df=f''$.\\
    于是对于任意$1\leqslant k\leqslant n$有$D\cos{\lambda_k x}=-\lambda_k^2\cos{\lambda_k x}$.故$-\lambda_k^2$是$T$的特征值,特征向量为$\cos{\lambda_k x}$.\\
    又$\li\lambda,n$是相异的正数,于是$-\lambda_1^2,\cdots,-\lambda_n^2$相异,于是$\cos{\lambda_1 x},\cdots,\cos{\lambda_n x}$线性无关.
\end{proof}
\begin{problem}[37.]
    设$V$是有限维的,$T\in\L(V)$.定义$\mathcal{A}\in\L(\L(V))$为$\mathcal{A}(S)=TS$对任意$S\in\L(V)$都成立.证明:$T$与$\mathcal{A}$的特征值相同.
\end{problem}
\begin{proof}
    设$\lambda$是$T$的特征值,对应的特征向量为$v$.于是存在$S\in\L(V)$使得对任意$u\in V,Su=v$.\\
    于是对于任意$u\in V$有$(\mathcal{A}(S))u=TSu=Tv=\lambda v=\lambda Su=(\lambda S)u$.\\
    于是$\lambda$是$\mathcal{A}$的特征值,对应的特征向量$S$如上定义.
\end{proof}
\begin{problem}[38.]
    设$V$是有限维的,$T\in\L(V)$,且$U$是$V$在$T$下的不变子空间.\tbf{商算子}$T/U\in\L(V/U)$定义为$(T/U)(v+U)=Tv+U$对任意$v\in V$成立.
    \begin{enumerate}[label=\tbf{(\arabic*)}]
        \item 证明:$T/U$的定义是有意义的,并且$T/U$是$V/U$上的算子.
        \item 证明:$T/U$的每个特征值都是$T$的特征值.
    \end{enumerate}
\end{problem}
\begin{proof}
    \begin{enumerate}[label=\tbf{(\arabic*)}]
        \item 考虑$v,w\in V$使得$v+U=w+U$,于是$v-w\in U$.由于$U$在$T$下不变,于是$T(v-w)=Tv-Tw\in U$.\\
            于是$(T/U)(v+U)=Tv+U=Tw+U=(T/U)(w+U)$,因而$T/U$的定义是有意义的.\\
            考虑$v+U,w+U\in V/U$,则有$$(T/U)(v+w+U)=T(v+w)+U=Tv+Tw+U=(Tv+U)+(Tw+U)=(T/U)(v+U)+(T/U)(w+U)$$
            又$v+w+U=(v+U)+(w+U)$,于是$T/U$满足可加性.齐次性的证明亦同理,不再赘述.\\
            于是$T/U$是$V/U$上的算子.
        \item 考虑$T/U$的特征值$\lambda$和对应的特征向量$v+U$,显然$v\notin U$.于是
            $$(T/U)(v+U)=Tv+U=\lambda(v+U)=\lambda v+U$$
            于是$Tv-\lambda v\in U$.设$Tv-\lambda v=u\in U$.考虑$R:=(T-\lambda I)|_U$.\\
            若$\nul R\neq\{\mbf0\}$,则存在$w\in U$使得$(T-\lambda I)w=\mbf0$,于是$\lambda$为$T$的特征值,对应的特征向量为$w$.\\
            若$\nul R=\{\mbf0\}$,则$R$是单射,于是$R$可逆.设$w\in U$满足$Rw=-u$,即$Tw=\lambda w-u$,于是
            $$T(v+w)=Tv+Tw=\lambda v+u+Tw=\lambda v+\lambda w=\lambda(v+w)$$
            于是$\lambda$为$T$的特征值,对应的特征向量为$v+w$.
    \end{enumerate}
\end{proof}
\begin{problem}[39.]
    设$V$是有限维的,$T\in\L(V)$.证明:$T$有特征值,当且仅当存在$V$的$\dim-1$维子空间在$T$下不变.
\end{problem}
\begin{proof}
    $\Rightarrow$:设$T$的特征值为$\lambda$,对应的特征向量为$v_1$.将其扩展为$V$的一组基$\li v,n$.\\
    不难得出$\dim\nul(T-\lambda I)\geqslant 1$而$\dim\range(T-\lambda I)\leqslant\dim V-1$.
    设$\range(T-\lambda I)$的一组基$\li v,m$,将其扩展为线性无关组$\li v,m,\li u,{\dim V-1-m}$,令$n=\dim V-1-m$,记$U=\span(\li v,m,\li u,n)$.\\
    于是对于任意$w:=a_1v_1+\cdots+a_mv_m+b_1u_1+\cdots+b_nu_n\in U$有
    $$Tw=\sum_{k=1}^{m}a_kTv_k\in\range(T-\lambda I)\subseteq U$$
    于是$U$在$(T-\lambda I)$下不变.对于任意$u\in U$有$Tu=(T-\lambda I)u+\lambda u\in U$,于是$U$在$T$下不变.\\
    $\Leftarrow$:设$U$在$T$下不变且$\dim U=\dim V-1$.考虑$T/U\in\L(V/U)$.\\
    由于$\dim\L(V/U)=\dim V-\dim U=1$,于是据\tbf{3A.7.}可知$T/U=\lambda I$.据\tbf{38.}可知$T$的特征值为$\lambda$.
\end{proof}
\begin{problem}[40.]
    设$S,T\in\L(V)$且$S$可逆.设$p\in\P(\F)$.证明:$p(STS^{-1})=Sp(T)S^{-1}$.
\end{problem}
\begin{proof}
    设$p(z)=\displaystyle\sum_{k=0}^{m}c_kz^k$.于是$\displaystyle p(STS^{-1})=\sum_{k=0}^{m}c_k(STS^{-1})^k$.\\
    注意到$(STS^{-1})^{0}=I=SS^{-1}=ST^0S^{-1}$.假定$(STS^{-1})^k=ST^kS^{-1}$,那么
    $$(STS^{-1})^{k+1}=(STS^{-1})^k(STS^{-1})=ST^{k}S^{-1}STS^{-1}=ST^{k}ITS^{-1}=ST^{k+1}S^{-1}$$
    于是对任意$k\in\N$有$(STS^{-1})^k=ST^kS^{-1}$.\\
    于是$\displaystyle p(STS^{-1})=\sum_{k=0}^{m}c_k(STS^{-1})^k=\sum_{k=0}^mc_kST^{k}S^{-1}=Sp(T)S^{-1}$.
\end{proof}
\begin{problem}[41.]
    设$T\in\L(V)$且$U$是$V$在$T$下的不变子空间.证明:对任意$p\in\P(\F)$,均有$U$在$p(T)$下不变.
\end{problem}
\begin{proof}
    首先$T^0u=Iu=u\in U$.若$T^ku\in U$,那么$T^{k+1}u=T(T^k u)\in U$.\\
    于是对于任意$k\in\N$有$T^ku\in U$,即$U$在$T^k$下不变.\\
    于是$\displaystyle p(T)u=\sum_{k=0}^{m}c_kT^ku\in U$,于是$U$在$p(T)$下不变.
\end{proof}
\begin{problem}[42.]
    定义$T\in\L(\F^n)$为$T(\li x,n)=(x_1,2x_2,\cdots,nx_n)$.
    \begin{enumerate}[label=\tbf{(\arabic*)}]
        \item 求出$T$的所有特征值和对应的特征向量.
        \item 求出$\F^n$的所有在$T$下不变的子空间.
    \end{enumerate}
\end{problem}
\begin{lemma}[Lemma.L.4]
    设$T\in\L(V)$,且$U$是$V$在$T$下的不变子空间.若$\li\lambda,m$是$T$的互异特征值且对应特征向量为$\li v,m$,那么%
    $\li v+m\in U$当且仅当$v_k\in U$对任意$k\in\{1,\cdots,m\}$成立.
\end{lemma}
\begin{proof}
    $\Leftarrow$:由于$v_k\in U$对所有$k$成立,又因为$U$是子空间,于是$\li v+m\in U$.\\
    $\Rightarrow$:我们采取归纳证明的方式.当$m=1$时,结论是显然的.\\
    若结论对于某个$m\in\N^*$时成立,那么取$T$的互异特征值$\li\lambda,{m+1}$和对应的特征向量$\li v,{m+1}$.\\
    令$v=\li v+{m+1}\in U$,则有
    \[Tv=T(\li v+{m=1})=\lambda_1v_1+\cdots+\lambda_{m+1}v_{m+1}\in U\]
    于是
    \[Tv-\lambda_{m+1}v_{m+1}=(\lambda_1-\lambda_{m+1})v_1+\cdots+(\lambda_m-\lambda_{m+1})v_m\in U\]
    由于$\li\lambda,{m+1}$互异,因而对于所有$k\in\{1,\cdots,m\}$都有$\lambda_k-\lambda_{m+1}\neq0$.\\
    于是$(\lambda_k-\lambda_{m+1})v_k$是$T$对应$\lambda_k$的特征向量.\\
    我们的假设表明$(\lambda_k-\lambda_{m+1})v_k\in U$,于是$v_k\in U$.又因为$v=\li v+{m+1}\in U$,于是$v_{m+1}\in U$.\\
    归纳可知命题成立.
\end{proof}
\begin{solution}[Solution.]
    \begin{enumerate}[label=\tbf{(\arabic*)}]
        \item 观察可得$T$的特征值为$1,2,\cdots,n$,对应的特征向量为$\F^n$的标准基.
        \item 设$\F^n$的标准基为$\li e,n$.$\F^n$的所有在$T$下不变的子空间为任意选取这些标准基构成的向量组张成的空间.
    \end{enumerate}
\end{solution}
\begin{problem}[43.]
    设$V$是有限维的,$\dim V>1$,且$T\in\L(V)$.证明:$\left\{p(T):p\in\P(\F)\right\}\neq\L(V)$.
\end{problem}
\begin{proof}
    对于任意$p,q\in\P(\F)$,都有$p(T)q(T)=q(T)p(T)$,于是$\left\{p(T):p\in\P(\F)\right\}$中的任意两个算子都是可交换的.\\
    然而\tbf{3A.16.}表明$\L(V)$中存在两个不可交换的算子,于是$\left\{p(T):p\in\P(\F)\right\}\neq\L(V)$.
\end{proof}
\end{document}