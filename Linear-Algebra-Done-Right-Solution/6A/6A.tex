\documentclass{ctexart}
\usepackage{geometry}
\usepackage[dvipsnames,svgnames]{xcolor}
\usepackage[strict]{changepage}
\usepackage{framed}
\usepackage{enumerate}
\usepackage{amsmath,amsthm,amssymb}
\usepackage{enumitem}
\usepackage{solution}

\allowdisplaybreaks
\geometry{left=2cm, right=2cm, top=2.5cm, bottom=2.5cm}

\begin{document}\pagestyle{empty}
\begin{center}
    \large\tbf{Linear Algebra Done Right 6A}
\end{center}
\begin{problem}[1.]
    证明:如果$\li v,m\in V$,那么
    \[\sum_{j=1}^{m}\sum_{k=1}^{m}\langle v_j,v_k\rangle\geqslant 0\]
\end{problem}
\begin{proof}
    我们有
    \[\begin{aligned}
        ||\li v+m||^2
        &= \sum_{j=1}^m||v_j||^2+2\sum_{j,k\in\{1,\cdots,m\},j\neq k}\langle v_j,v_k\rangle \\
        &= \sum_{j=1}^{m}\sum_{k=1}^{m}\langle v_j,v_k\rangle
    \end{aligned}\]
    于是
    \[\sum_{j=1}^{m}\sum_{k=1}^{m}\langle v_j,v_k\rangle=||\li v+m||^2\geqslant0\]
\end{proof}
\begin{problem}[2.]
    设$S\in\L(V)$,定义$\langle\cdot,\cdot\rangle_S$为
    \[\langle u,v\rangle_S=\langle Su,Sv\rangle\]
    对所有$u,v\in V$成立.试证明:$\langle\cdot,\cdot\rangle_S$是$V$上的内积,当且仅当$S$是单射.
\end{problem}
\begin{proof}
    我们有
    \[\langle\cdot,\cdot\rangle_S\text{是内积}\Leftrightarrow \langle v,v\rangle_S=0\text{当且仅当}v=\mbf0\Leftrightarrow \langle Sv,Sv\rangle=0\text{当且仅当}Sv=\mbf0\Leftrightarrow\nul S=\{\mbf0\}\Leftrightarrow S\text{是单射}\]
\end{proof}
\begin{problem}[3.]
    证明下列命题.
    \begin{enumerate}[label=\tbf{(\arabic*)}]
        \item 证明:将$\R^2$中的有序对$\left((x_1,x_2),(y_1,y_2)\right)$映射到$|x_1y_1|+|x_2y_2|$的函数不是$\R^2$上的内积.
        \item 证明:将$\R^3$中的有序对$\left((x_1,x_1,x_3),(y_1,y_2,y_3)\right)$映射到$x_1y_1+x_3y_3$的函数不是$\R^3$上的内积.
    \end{enumerate}
\end{problem}
\begin{proof}
    \begin{enumerate}[label=\tbf{(\arabic*)}]
        \item 令$u=(1,0),v=(-1,0),w=(1,0)$,于是
            \[f(u,w)=1\ \ \ \ \ f(v,w)=1\ \ \ \ \ f(u+v,w)=0\]
            于是$f(u+v,w)\neq f(u,w)+f(v,w)$,因而这映射$f$不满足第一位上的可加性,不是$\R^2$上的内积.
        \item 令$v=(0,1,0)$,则$g(v,v)=0$,而$v\neq\mbf0$,因而这映射$g$不满足定性,不是$\R^3$上的内积.
    \end{enumerate}
\end{proof}
\begin{problem}[4.]
    设$T\in\L(V)$使得$||Tv||\leqslant||v||$对任意$v\in V$成立.试证明:$T-\sqrt2I$是单射.
\end{problem}
\begin{proof}
    若$T-\sqrt2I$不是单射,则存在$v\in V$且$v\neq\mbf0$使得$Tv=\sqrt2v$.于是
    \[||Tv||=||\sqrt2v||=\sqrt2||v||>||v||\]
    这与题设矛盾,从而$T-\sqrt2I$是单射.
\end{proof}
\begin{problem}[5.]
    设$V$是实内积空间.证明下列命题.
    \begin{enumerate}[label=\tbf{(\arabic*)}]
        \item 证明:$\langle u+v,u-v\rangle=||u||^2-||v||^2$对任意$u,v\in V$成立.
        \item 证明:若$u,v\in V$满足$||u||=||v||$,那么$u+v$正交于$u-v$.
        \item 证明:菱形的对角线相互垂直.
    \end{enumerate}
\end{problem}
\begin{proof}
    \begin{enumerate}[label=\tbf{(\arabic*)}]
        \item 我们有
            \[\begin{aligned}
                \langle u+v,u-v\rangle
                &= \langle u,u-v\rangle+\langle v,u-v\rangle \\
                &= \langle u,u\rangle-\langle u,v\rangle+\langle v,u\rangle-\langle v,v\rangle \\
                &= \langle u,u\rangle-\inprod vv \\
                &= ||u||^2-||v||^2
            \end{aligned}\]
        \item 我们有
            \[||u||=||v||\Rightarrow \inprod{u+v}{u-v}=0\Rightarrow u+v\bot u-v\]
        \item 令$V=\R^2$,考虑菱形$ABCD$,令$u=\overrightarrow{BA},v=\overrightarrow{BC}$,则有$u+v=\overrightarrow{BD},u-v=\overrightarrow{CA}$.\\
            由于$BA=BC$,则$||u||=||v||$,由\tbf{(2)}可知$\overrightarrow{BD}$和$\overrightarrow{CA}$正交,因而$BD\bot AC$,命题得证.
    \end{enumerate}
\end{proof}
\begin{problem}[6.]
    设$u,v\in V$,试证明:$\inprod{u}{v}=0$当且仅当对任意$a\in\F$都有$||u||\leqslant||u+av||$.
\end{problem}
\begin{proof}
    $\Rightarrow$:对任意$a\in\F$有$\inprod{u}{av}=\bar{a}\inprod{u}{v}=0$,于是$u$和$av$正交.于是
    \[||u+av||^2=||u||^2+||av||^2\geqslant||u||^2\]
    于是$||u||\leqslant||u+av||$.\\
    $\Leftarrow$:考虑$u$的正交分解,取$c=\dfrac{\inprod{u}{v}}{||v||^2}$.取$w=u-cv$,则$\inprod{w}{v}=0$.于是
    \[||u+cv||^2=||w||^2=||u||^2-||cv||^2=||u||^2-c^2||v||^2\geqslant||u||^2\]
    于是$c=0$,因而$\inprod{u}{v}=\inprod{w}{v}=0$.
\end{proof}
\begin{problem}[7.]
    设$u,v\in V$.试证明:$||au+bv||=||bu+av||$对任意$a,b\in\R$成立,当且仅当$||u||=||v||$.
\end{problem}
\begin{proof}
    $\Rightarrow$:取$a=1,b=0$即有$||u||=||v||$.\\
    $\Leftarrow$:当$||u||=||v||$时,我们有
    \[\begin{aligned}
        ||au+bv||^2
        &= \inprod{au+bv}{au+bv} \\
        &= a^2||u||^2+b^2||v||^2+2ab\inprod uv \\
        &= a^2||v||^2+b^2||u||^2+2ab\inprod uv \\
        &= \inprod{bu+av}{bu+av} \\
        &= ||bu+av||^2
    \end{aligned}\]
    于是$||au+bv||=||bu+av||$.
\end{proof}
\begin{problem}[8.]
    设$a,b,c,x,y\in\R$且$a^2+b^2+c^2+x^2+y^2\leqslant1$.试证明:$a+b+c+4x+9y\leqslant10$.
\end{problem}
\begin{proof}
    我们有
    \[(a+b+c+4x+9y)^2\leqslant(a^2+b^2+c^2+x^2+y^2)(1^2+1^2+1^2+4^2+9^2)\leqslant100\]
    两边开平方即得\[a+b+c+4x+9y\leqslant10\]
\end{proof}
\begin{problem}[9.]
    设$u,v\in V$,$||u||=||v||=\inprod uv=1$.试证明:$u=v$.
\end{problem}
\begin{proof}
    据Cauchy-Schwarz不等式,$|\langle u,v\rangle|\leqslant||u||||v||$当且仅当$u=\lambda v,\lambda\in\F$时成立.\\
    又$1=\inprod uv=\inprod{v}{\lambda v}=\lambda||v||^2=\lambda$,于是$\lambda=1$,即$u=v$.
\end{proof}
\begin{problem}[10.]
    设$u,v\in V$,$||u||<1$且$||v||<1$.试证明
    \[\sqrt{1-||u||^2}\sqrt{1-||v||^2}\leqslant1-\left|\inprod uv\right|\]
\end{problem}
\begin{proof}
    我们有
    \[\left(||u||-||v||\right)^2\geqslant0\]
    于是
    \[||u||^2+||v||^2\geqslant2||u||||v||\]
    变形可得
    \[\left(1-||u||^2\right)\left(1-||v||^2\right)\leqslant\left(1-||u||||v||\right)^2\]
    而$||u||,||v||<1$,于是
    \[\sqrt{1-||u||^2}\sqrt{1-||v||^2}\leqslant1-||u||||v||\]
    据Cauchy-Schwarz不等式有
    \[\left|\inprod uv\right|\leqslant ||u||||v||\]
    于是
    \[\sqrt{1-||u||^2}\sqrt{1-||v||^2}\leqslant1-\left|\inprod uv\right|\]
\end{proof}
\begin{problem}[11.]
    求向量$u,v\in\R^2$使得$u$是$(1,3)$的标量倍,$v$正交于$(1,3)$,且$u+v=(1,2)$.
\end{problem}
\begin{solution}
    考虑$(1,2)$在$(1,3)$上的正交分解$(1,2)=c(1,3)+w$,其中$c=\dfrac{\inprod{(1,2)}{(1,3)}}{||(1,3)||^2}=\dfrac{7}{10}$.\\
    于是令$u=\left(\dfrac{7}{10},\dfrac{21}{10}\right),v=\left(\dfrac{3}{10},-\dfrac{1}{10}\right)$.
\end{solution}
\begin{problem}[12.]
    设$a,b,c,d\in\R^*$.
    \begin{enumerate}[label=\tbf{(\arabic*)}]
        \item 试证明:$\left(a+b+c+d\right)\left(\dfrac1a+\dfrac1b+\dfrac1c+\dfrac1d\right)\geqslant16$.
        \item 求上述不等式的取等条件.
    \end{enumerate}
\end{problem}
\begin{solution}
    \begin{enumerate}[label=\tbf{(\arabic*)}]
        \item 令$u=\left(\sqrt{a},\sqrt{b},\sqrt{c},\sqrt{d}\right),v=\left(\sqrt{\dfrac1a},\sqrt{\dfrac1b},\sqrt{\dfrac1c},\sqrt{\dfrac1d}\right)$.据Cauchy-Schwarz不等式有
            \[\left|\inprod uv\right|\leqslant||u||||v||\]
            即
            \[4\leqslant\sqrt{a+b+c+d}\cdot\sqrt{\dfrac1a+\dfrac1b+\dfrac1c+\dfrac1d}\]
            两边平方即得
            \[\left(a+b+c+d\right)\left(\dfrac1a+\dfrac1b+\dfrac1c+\dfrac1d\right)\geqslant16\]
        \item 据Cauchy-Schwarz不等式的取等条件,当且仅当$u=\lambda v$时取等,即$a=b=c=d$.
    \end{enumerate}
\end{solution}
\begin{problem}[13.]
    证明:如果$\li a,n\in\R$,那么
    \[\left(\dfrac{\li a+n}{n}\right)^2\leqslant \dfrac{a_1^2+\cdots+a_n^2}{n}\]
\end{problem}
\begin{proof}
    据Cauchy-Schwarz不等式有
    \[\left(\dfrac{\li a+n}{n}\right)^2\leqslant\left(n\cdot\dfrac{1}{n^2}\right)\left(a_1^2+\cdots+a_n^2\right)=\dfrac{a_1^2+\cdots+a_n^2}{n}\]
    于是命题得证.
\end{proof}
\begin{problem}[14.]
    设$v\in V$且$v\neq\mbf0$.试证明:如果$u\in V$且$||u||=1$,那么
    \[\left|\left|v-\dfrac{v}{||v||}\right|\right|\leqslant||v-u||\]
    当且仅当$u=\dfrac{v}{||v||}$时等号成立.
\end{problem}
\begin{proof}
    我们有
    \[||v-u||^2-\left(||v||-||u||\right)^2=2\inprod vu+2||u||||v||\geqslant0\]
    即
    \[\left|||v||-||u||\right|\leqslant||v-u||^2\]
    当且仅当$v,u$成标量倍关系时等式成立,即$u=\dfrac{v}{||v||}$或$u=-\dfrac{v}{||v||}$时等号成立.\\
    我们有
    \[\left|1-\dfrac{1}{||v||}\right|<1+\dfrac{1}{||v||}\]
    于是
    \[\min||v-u||=\left|\left|v-\dfrac{v}{||v||}\right|\right|\]
    命题得证.
\end{proof}
\begin{problem}[15.]
    设$u,v\in\R^2$且$u,v\neq\mbf0$,试证明:
    \[\inprod uv=||u||||v||\cos\theta\]
    其中$\theta$为向量$u,v$的夹角.
\end{problem}
\begin{proof}
    对于$u,v,u-v$构成的三角形使用余弦定理,则有
    \[\cos\theta=\dfrac{||u||^2+||v||^2-||u-v||^2}{2||u||||v||}\]
    而
    \[||u-v||^2=||u||^2+||v^2||-2\inprod uv\]
    代入上式有
    \[\cos\theta=\dfrac{\inprod uv}{||u||||v||}\]
    变形可得
    \[\inprod uv=||u||||v||\cos\theta\]
    于是命题得证.
\end{proof}
\begin{problem}[16.]
    $\R^2$和$\R^3$中向量的夹角可以用几何方法定义,然而对于更高维的空间$\R^n$中的几何则并不明晰.
    因此,定义$x,y\in\R^n$的夹角$\theta$为
    \[\arccos\dfrac{\inprod xy}{||x||||y||}\]
    这一定义的动机来源于$\R^2$和$\R^3$中夹角的几何意义.试解释:证明这一定义的成立需要用到Cauchy-Schwarz不等式.
\end{problem}
\begin{solution}
    为保证$\arccos$函数有意义,这一定义的成立至少要求
    \[-1\leqslant\dfrac{\inprod xy}{||x||||y||}\leqslant 1\]
    即
    \[\left|\inprod xy\right|\leqslant||x||||y||\]
    这就需要用到Cauchy-Schwarz不等式.
\end{solution}
\begin{problem}[17.]
    试证明:对于任意$\li a,n,\li b,n\in\R$,都有
    \[\left(\sum_{k=1}^{n}a_kb_k\right)^2\leqslant\left(\sum_{k=1}^{n}ka_k^2\right)\left(\sum_{k=1}^{n}\dfrac{b_k^2}{k}\right)\]
\end{problem}
\begin{proof}
    令$u=\left(a_1,\sqrt2a_2,\cdots,\sqrt{n}a_n\right),v=\left(b_1,\dfrac{b_2}{\sqrt{2}},\cdots,\dfrac{b_n}{\sqrt{n}}\right)$.据Cauchy-Schwarz不等式有
    \[\left|\inprod uv\right|^2\leqslant||u||^2||v||^2\]
    即
    \[\left(\sum_{k=1}^{n}a_kb_k\right)^2\leqslant\left(\sum_{k=1}^{n}ka_k^2\right)\left(\sum_{k=1}^{n}\dfrac{b_k^2}{k}\right)\]
\end{proof}
\begin{problem}[18.]
    设函数$f(x):[1,+\infty)\to[0,+\infty)$连续.
    \begin{enumerate}[label=\tbf{(\arabic*)}]
        \item 试证明:
            \[\left(\int_{1}^{+\infty}f(x)\di x\right)^2\leqslant\int_1^{+\infty}x^2\left(f(x)\right)^2\di x\].
        \item 试给出上述不等式两边均有限且取等时$f(x)$满足的条件.
    \end{enumerate}
\end{problem}
\begin{proof}
    \begin{enumerate}[label=\tbf{(\arabic*)}]
        \item 考虑$\R^{[1,t]}$上的内积
            \[\inprod uv=\int_{1}^{t}uv\]
            据Cauchy-Schwarz不等式有
            \[\left(\int_1^tuv\right)^2\leqslant\left(\int_1^t u^2\right)\left(\int_1^t v^2\right)\]
            于是
            \[\left(\int_1^tf(x)\di x\right)^2=\left(\int_1^t\dfrac xxf(x)\di x\right)\leqslant\left(\int_1^tx^2\left(f(x)\right)^2\di x\right)\left(\int_1^t\dfrac{1}{x^2}\di x\right)=\left(\int_1^tx^2\left(f(x)\right)^2\di x\right)\left(1-\dfrac 1t\right)\]
            而$\displaystyle\lim_{t\to+\infty}\left(1-\dfrac1t\right)=1$,又$f(x)$非负,于是
            \[\left(\int_1^{+\infty}f(x)\di x\right)^2\leqslant\int_{1}^{+\infty}x^2\left(f(x)\right)^2\di x\]
        \item 当且仅当$xf(x)$是$\dfrac1x$的标量倍,即$f(x)=\dfrac{\lambda}{x^2}$,其中$\lambda\in\R^*$.此时
            \[\left(\int_1^{+\infty}f(x)\di x\right)^2=\int_{1}^{+\infty}x^2\left(f(x)\right)^2\di x=\lambda^2\]
    \end{enumerate}
\end{proof}
\begin{problem}[19.]
    设$\li v,n$是$V$的一个基,且$T\in\L(V)$.试证明:如果$\lambda$是$T$的特征值,那么
    \[|\lambda|^2\leqslant\sum_{j=1}^n\sum_{k=1}^n\left|\mathcal{M}(T)_{j,k}\right|^2\]
\end{problem}
\begin{proof}
    考虑$V$上的内积
    \[\inprod{a_1v_1+\cdots+a_nv_n}{b_1v_1+\cdots+b_nv_n}=a_1\overline{b_1}+\cdots+a_n\overline{b_n}\]
    考虑$T$的特征向量$v:=a_1v_1+\cdots+a_nv_n$使得$Tv=\lambda v$,则有
    \[Tv=\sum_{k=1}^{n}a_kTv_k=\sum_{k=1}^n\left(a_k\sum_{j=1}^n\mathcal{M}(T)_{j,k}v_j\right)=\sum_{j=1}^n\left(\sum_{k=1}^{n}a_k\mathcal{M}(T)_{j,k}\right)v_j\]
    于是
    \[||Tv||^2=\sum_{j=1}^{n}\left|\sum_{k=1}^{n}a_k\mathcal{M}(T)_{j,k}\right|^2\]
    又因为$||Tv||^2=|\lambda|^2||v||^2$,于是
    \[|\lambda|^2=\dfrac{\displaystyle\sum_{j=1}^{n}\left|\sum_{k=1}^{n}a_k\mathcal{M}(T)_{j,k}\right|^2}{||v||^2}=\dfrac{\displaystyle\sum_{j=1}^{n}\left|\sum_{k=1}^{n}a_k\mathcal{M}(T)_{j,k}\right|^2}{\displaystyle\sum_{k=1}^{n}|a_k|^2}\]
    据Cauchy-Schwarz不等式,对于任意$j\in\{1,\cdots,n\}$有
    \[\left|\sum_{k=1}^{n}a_k\mathcal{M}(T)_{j,k}\right|^2\leqslant\left(\sum_{k=1}^{n}\left|a_k\right|^2\right)\left(\sum_{k=1}^{n}\left|\mathcal{M}(T)_{j,k}\right|^2\right)\]
    于是
    \[|\lambda|^2\leqslant\sum_{j=1}^n\sum_{k=1}^n\left|\mathcal{M}(T)_{j,k}\right|^2\]
\end{proof}
\begin{problem}[20.]
    试证明:如果$u,v\in V$,那么$\left|||u||-||v||\right|\leqslant||u-v||$.
\end{problem}
\begin{proof}
    我们有
    \[||u-v||^2-\left(||u||-||v||\right)^2=2\inprod uv+2||u||||v||\geqslant0\]
    即
    \[\left|||u||-||v||\right|\leqslant||u-v||^2\]
    当且仅当$v,u$成标量倍关系时等式成立.
\end{proof}
\begin{problem}[21.]
    设$u,v\in V$使得
    \[||u||=3\ \ \ \ \ ||u+v||=4\ \ \ \ \ ||u-v||=6\]
    求$||v||$.
\end{problem}
\begin{solution}
    根据平行四边形不等式有
    \[4^2+6^2=2\left(3^2+||v||^2\right)\]
    于是$||v||=\sqrt{17}$.
\end{solution}
\begin{problem}[22.]
    试证明:如果$u,v\in V$,那么
    \[||u+v||||u-v||\leqslant||u||^2+||v||^2\]
\end{problem}
\begin{proof}
    我们有
    \[\begin{aligned}
        &\left(||u||^2+||v||^2\right)^2-\left(||u+v||||u-v||\right)^2\\
        =&\left(||u||^2+||v||^2\right)^2-||u+v||^2||u-v||^2\\
        =&\left(||u||^2+||v||^2\right)^2-\left(||u||^2+||v||^2+2\left|\inprod uv\right|\right)\left(||u||^2+||v||^2-2\left|\inprod uv\right|\right) \\
        =&\left(||u||^2+||v||^2\right)^2-\left(||u||^2+||v||^2\right)^2+4\left|\inprod uv\right|^2 \\
        =&4\left|\inprod uv\right|^2\geqslant0
    \end{aligned}\]
    即
    \[\left(||u+v||||u-v||\right)^2\leqslant\left(||u||^2+||v||^2\right)^2\]
    两边开平方即得
    \[||u+v||||u-v||\leqslant||u||^2+||v||^2\]
\end{proof}
\begin{problem}[23.]
    设$\li v,m\in V$使得对任意$k\in\{1,\cdots,m\}$都有$||v_k||\leqslant1$.试证明:存在$\li a,m\in\{-1,1\}$使得
    \[||a_1v_1+\cdots+a_mv_m||\leqslant\sqrt m\]
\end{problem}
\begin{proof}
    对$m$使用归纳法.当$m=1$时,命题的成立是显然的,因为$||v_1||\leqslant1$.\\
    现在假设$m>1$,且命题对所有小于$m$的正整数均成立.令$v=a_1v_1+\cdots+a_{m-1}v_{m-1}$,于是$||v||\leqslant\sqrt{m-1}$.\\
    我们有
    \[||v+a_mv_m||^2=||v||^2+||v_m||^2+2a_m\inprod{v}{v_m}=m+2a_m\inprod{v}{v_m}\]
    若$\inprod{v}{v_m}\geqslant0$,则令$a_m=-1$,否则令$a_m=1$,则有
    \[||v+a_mv_m||^2\leqslant m\]
    两边开平方即得
    \[||v+a_mv_m||=||a_1v_1+\cdots+a_mv_m||\leqslant\sqrt{m}\]
    归纳可知命题成立.
\end{proof}
\begin{problem}
    
\end{problem}
\end{document}