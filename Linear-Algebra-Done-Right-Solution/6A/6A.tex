\documentclass{ctexart}
\usepackage{geometry}
\usepackage[dvipsnames,svgnames]{xcolor}
\usepackage[strict]{changepage}
\usepackage{framed}
\usepackage{enumerate}
\usepackage{amsmath,amsthm,amssymb}
\usepackage{enumitem}
\usepackage{solution}

\allowdisplaybreaks
\geometry{left=2cm, right=2cm, top=2.5cm, bottom=2.5cm}

\begin{document}\pagestyle{empty}
\begin{center}
    \large\tbf{Linear Algebra Done Right 6A}
\end{center}
\begin{problem}[1.]
    证明:如果$\li v,m\in V$,那么
    \[\sum_{j=1}^{m}\sum_{k=1}^{m}\langle v_j,v_k\rangle\geqslant 0\]
\end{problem}
\begin{proof}
    我们有
    \[\begin{aligned}
        ||\li v+m||^2
        &= \sum_{j=1}^m||v_j||^2+2\sum_{j,k\in\{1,\cdots,m\},j\neq k}\langle v_j,v_k\rangle \\
        &= \sum_{j=1}^{m}\sum_{k=1}^{m}\langle v_j,v_k\rangle
    \end{aligned}\]
    于是
    \[\sum_{j=1}^{m}\sum_{k=1}^{m}\langle v_j,v_k\rangle=||\li v+m||^2\geqslant0\]
\end{proof}
\begin{problem}[2.]
    设$S\in\L(V)$,定义$\langle\cdot,\cdot\rangle_S$为
    \[\langle u,v\rangle_S=\langle Su,Sv\rangle\]
    对所有$u,v\in V$成立.试证明:$\langle\cdot,\cdot\rangle_S$是$V$上的内积,当且仅当$S$是单射.
\end{problem}
\begin{proof}
    我们有
    \[\langle\cdot,\cdot\rangle_S\text{是内积}\Leftrightarrow \langle v,v\rangle_S=0\text{当且仅当}v=\mbf0\Leftrightarrow \langle Sv,Sv\rangle=0\text{当且仅当}Sv=\mbf0\Leftrightarrow\nul S=\{\mbf0\}\Leftrightarrow S\text{是单射}\]
\end{proof}
\begin{problem}[3.]
    证明下列命题.
    \begin{enumerate}[label=\tbf{(\arabic*)}]
        \item 证明:将$\R^2$中的有序对$\left((x_1,x_2),(y_1,y_2)\right)$映射到$|x_1y_1|+|x_2y_2|$的函数不是$\R^2$上的内积.
        \item 证明:将$\R^3$中的有序对$\left((x_1,x_1,x_3),(y_1,y_2,y_3)\right)$映射到$x_1y_1+x_3y_3$的函数不是$\R^3$上的内积.
    \end{enumerate}
\end{problem}
\begin{proof}
    \begin{enumerate}[label=\tbf{(\arabic*)}]
        \item 令$u=(1,0),v=(-1,0),w=(1,0)$,于是
            \[f(u,w)=1\ \ \ \ \ f(v,w)=1\ \ \ \ \ f(u+v,w)=0\]
            于是$f(u+v,w)\neq f(u,w)+f(v,w)$,因而这映射$f$不满足第一位上的可加性,不是$\R^2$上的内积.
        \item 令$v=(0,1,0)$,则$g(v,v)=0$,而$v\neq\mbf0$,因而这映射$g$不满足定性,不是$\R^3$上的内积.
    \end{enumerate}
\end{proof}
\begin{problem}[4.]
    设$T\in\L(V)$使得$||Tv||\leqslant||v||$对任意$v\in V$成立.试证明:$T-\sqrt2I$是单射.
\end{problem}
\begin{proof}
    若$T-\sqrt2I$不是单射,则存在$v\in V$且$v\neq\mbf0$使得$Tv=\sqrt2v$.于是
    \[||Tv||=||\sqrt2v||=\sqrt2||v||>||v||\]
    这与题设矛盾,从而$T-\sqrt2I$是单射.
\end{proof}
\begin{problem}[5.]
    设$V$是实内积空间.证明下列命题.
    \begin{enumerate}[label=\tbf{(\arabic*)}]
        \item 证明:$\langle u+v,u-v\rangle=||u||^2-||v||^2$对任意$u,v\in V$成立.
        \item 证明:若$u,v\in V$满足$||u||=||v||$,那么$u+v$正交于$u-v$.
        \item 证明:菱形的对角线相互垂直.
    \end{enumerate}
\end{problem}
\begin{proof}
    \begin{enumerate}[label=\tbf{(\arabic*)}]
        \item 我们有
            \[\begin{aligned}
                \langle u+v,u-v\rangle
                &= \langle u,u-v\rangle+\langle v,u-v\rangle \\
                &= \langle u,u\rangle-\langle u,v\rangle+\langle v,u\rangle-\langle v,v\rangle \\
                &= \langle u,u\rangle-\inprod vv \\
                &= ||u||^2-||v||^2
            \end{aligned}\]
        \item 我们有
            \[||u||=||v||\Rightarrow \inprod{u+v}{u-v}=0\Rightarrow u+v\bot u-v\]
        \item 令$V=\R^2$,考虑菱形$ABCD$,令$u=\overrightarrow{BA},v=\overrightarrow{BC}$,则有$u+v=\overrightarrow{BD},u-v=\overrightarrow{CA}$.\\
            由于$BA=BC$,则$||u||=||v||$,由\tbf{(2)}可知$\overrightarrow{BD}$和$\overrightarrow{CA}$正交,因而$BD\bot AC$,命题得证.
    \end{enumerate}
\end{proof}
\begin{problem}[6.]
    设$u,v\in V$,试证明:$\inprod{u}{v}=0$当且仅当对任意$a\in\F$都有$||u||\leqslant||u+av||$.
\end{problem}
\begin{proof}
    $\Rightarrow$:对任意$a\in\F$有$\inprod{u}{av}=\bar{a}\inprod{u}{v}=0$,于是$u$和$av$正交.于是
    \[||u+av||^2=||u||^2+||av||^2\geqslant||u||^2\]
    于是$||u||\leqslant||u+av||$.\\
    $\Leftarrow$:考虑$u$的正交分解,取$c=\dfrac{\inprod{u}{v}}{||v||^2}$.取$w=u-cv$,则$\inprod{w}{v}=0$.于是
    \[||u+cv||^2=||w||^2=||u||^2-||cv||^2=||u||^2-c^2||v||^2\geqslant||u||^2\]
    于是$c=0$,因而$\inprod{u}{v}=\inprod{w}{v}=0$.
\end{proof}
\begin{problem}[7.]
    设$u,v\in V$.试证明:$||au+bv||=||bu+av||$对任意$a,b\in\R$成立,当且仅当$||u||=||v||$.
\end{problem}
\begin{proof}
    $\Rightarrow$:取$a=1,b=0$即有$||u||=||v||$.\\
    $\Leftarrow$:当$||u||=||v||$时,我们有
    \[\begin{aligned}
        ||au+bv||^2
        &= \inprod{au+bv}{au+bv} \\
        &= a^2||u||^2+b^2||v||^2+2ab\inprod uv \\
        &= a^2||v||^2+b^2||u||^2+2ab\inprod uv \\
        &= \inprod{bu+av}{bu+av} \\
        &= ||bu+av||^2
    \end{aligned}\]
    于是$||au+bv||=||bu+av||$.
\end{proof}
\begin{problem}[8.]
    设$a,b,c,x,y\in\R$且$a^2+b^2+c^2+x^2+y^2\leqslant1$.试证明:$a+b+c+4x+9y\leqslant10$.
\end{problem}
\begin{proof}
    我们有
    \[(a+b+c+4x+9y)^2\leqslant(a^2+b^2+c^2+x^2+y^2)(1^2+1^2+1^2+4^2+9^2)\leqslant100\]
    两边开平方即得\[a+b+c+4x+9y\leqslant10\]
\end{proof}
\begin{problem}[9.]
    设$u,v\in V$,$||u||=||v||=\inprod uv=1$.试证明:$u=v$.
\end{problem}
\begin{proof}
    据Cauchy-Schwarz不等式,$|\langle u,v\rangle|\leqslant||u||||v||$当且仅当$u=\lambda v,\lambda\in\F$时成立.\\
    又$1=\inprod uv=\inprod{v}{\lambda v}=\lambda||v||^2=\lambda$,于是$\lambda=1$,即$u=v$.
\end{proof}
\begin{problem}[10.]
    设$u,v\in V$,$||u||<1$且$||v||<1$.试证明
    \[\sqrt{1-||u||^2}\sqrt{1-||v||^2}\leqslant1-\left|\inprod uv\right|\]
\end{problem}
\begin{proof}
    我们有
    \[\left(||u||-||v||\right)^2\geqslant0\]
    于是
    \[||u||^2+||v||^2\geqslant2||u||||v||\]
    变形可得
    \[\left(1-||u||^2\right)\left(1-||v||^2\right)\leqslant\left(1-||u||||v||\right)^2\]
    而$||u||,||v||<1$,于是
    \[\sqrt{1-||u||^2}\sqrt{1-||v||^2}\leqslant1-||u||||v||\]
    据Cauchy-Schwarz不等式有
    \[\left|\inprod uv\right|\leqslant ||u||||v||\]
    于是
    \[\sqrt{1-||u||^2}\sqrt{1-||v||^2}\leqslant1-\left|\inprod uv\right|\]
\end{proof}
\begin{problem}[11.]
    求向量$u,v\in\R^2$使得$u$是$(1,3)$的标量倍,$v$正交于$(1,3)$,且$u+v=(1,2)$.
\end{problem}
\begin{solution}
    考虑$(1,2)$在$(1,3)$上的正交分解$(1,2)=c(1,3)+w$,其中$c=\dfrac{\inprod{(1,2)}{(1,3)}}{||(1,3)||^2}=\dfrac{7}{10}$.\\
    于是令$u=\left(\dfrac{7}{10},\dfrac{21}{10}\right),v=\left(\dfrac{3}{10},-\dfrac{1}{10}\right)$.
\end{solution}
\begin{problem}[12.]
    设$a,b,c,d\in\R^*$.
    \begin{enumerate}[label=\tbf{(\arabic*)}]
        \item 试证明:$\left(a+b+c+d\right)\left(\dfrac1a+\dfrac1b+\dfrac1c+\dfrac1d\right)\geqslant16$.
        \item 求上述不等式的取等条件.
    \end{enumerate}
\end{problem}
\begin{solution}
    \begin{enumerate}[label=\tbf{(\arabic*)}]
        \item 令$u=\left(\sqrt{a},\sqrt{b},\sqrt{c},\sqrt{d}\right),v=\left(\sqrt{\dfrac1a},\sqrt{\dfrac1b},\sqrt{\dfrac1c},\sqrt{\dfrac1d}\right)$.据Cauchy-Schwarz不等式有
            \[\left|\inprod uv\right|\leqslant||u||||v||\]
            即
            \[4\leqslant\sqrt{a+b+c+d}\cdot\sqrt{\dfrac1a+\dfrac1b+\dfrac1c+\dfrac1d}\]
            两边平方即得
            \[\left(a+b+c+d\right)\left(\dfrac1a+\dfrac1b+\dfrac1c+\dfrac1d\right)\geqslant16\]
        \item 据Cauchy-Schwarz不等式的取等条件,当且仅当$u=\lambda v$时取等,即$a=b=c=d$.
    \end{enumerate}
\end{solution}
\end{document}