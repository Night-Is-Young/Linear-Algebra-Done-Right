\documentclass{ctexart}
\usepackage{geometry}
\usepackage[dvipsnames,svgnames]{xcolor}
\usepackage[strict]{changepage}
\usepackage{framed}
\usepackage{enumerate}
\usepackage{amsmath,amsthm,amssymb}
\usepackage{enumitem}
\usepackage{solution}

\allowdisplaybreaks
\geometry{left=2cm, right=2cm, top=2.5cm, bottom=2.5cm}

\begin{document}\pagestyle{empty}
\begin{center}
    \large\tbf{Linear Algebra Done Right 7F}
\end{center}
\begin{problem}[1.]
    设$S,T\in\L(V,W)$.试证明:$\left|||S||-||T||\right|\leqslant||S-T||$.
\end{problem}
\begin{proof}
    由线性映射的范数的定义可知存在$v\in V$且$||v||\leqslant1$使得$||S-T||=||(S-T)v||$.于是
    \[||S-T||=||(S-T)v||=||Sv-Tv||\geqslant\left|||Sv||-||Tv||\right|\geqslant\left|||S||-||T||\right|\]
    以上不等式即\tbf{反向三角不等式}.
\end{proof}
\begin{problem}[2.]
    设$T\in\L(V)$是自伴的(如果$\F=\C$则令$T$是正规的),试证明:$||T||=\max\{|\lambda|:\lambda\text{是}T\text{的特征值}\}$.
\end{problem}
\begin{proof}
    由\tbf{7.88}和\tbf{7E.7}可知
    \[||T||=T\text{的最大奇异值}=T\text{的绝对值最大的特征值}\]
\end{proof}
\begin{problem}[3.]
    设$T\in\L(V,W)$且$v\in V$.试证明:$||Tv||=||T||||v||$当且仅当$T^*Tv=||T||^2v$.
\end{problem}
\begin{proof}
    $\Leftarrow$:由\tbf{7.88(c)}和\tbf{7.91}可得
    \[||T||^2||v||=||T^*Tv||\leqslant||T^*||||Tv||=||T||||Tv||\Rightarrow||Tv||\geqslant||T||||v||\]
    而$||Tv||\leqslant||T||||v||$,于是$||Tv||=||T||||v||$.\\
    $\Rightarrow$:我们有
    \[\begin{aligned}
        ||T^*Tv-||T||^2v||^2
        &= \inprod{T^*Tv-||T||^2v}{T^*Tv-||T||^2v} \\
        &= ||T^*Tv||^2+||T||^4||v||^2-2\text{Re}\inprod{T^*Tv}{||T||^2v} \\
        &\leqslant ||T^*||^2||Tv||^2+||T||^4||v||^2-2||T||^2||Tv||^2 \\
        &= 0
    \end{aligned}\]
    于是$T^*Tv=||T||^2v$.
\end{proof}
\begin{problem}[4.]
    设$T\in\L(V,W),v\in V$且$||Tv||=||T||||v||$.试证明:如果$u\in V$且$\inprod uv=0$,那么$\inprod{Tu}{Tv}=0$.
\end{problem}
\begin{proof}
    根据\tbf{7F.3}有
    \[\inprod{Tu}{Tv}=\inprod{u}{T^*Tv}=\inprod{u}{||T||^2v}=||T||^2\inprod uv=0\]
\end{proof}
\begin{problem}[5.]
    设$U$是有限维内积空间,$T\in\L(V,U)$且$S\in\L(U,W)$.试证明:$||ST||\leqslant||S||||T||$.
\end{problem}
\begin{proof}
    由线性映射的范数的定义可知存在$v\in V$且$||v||\leqslant1$使得$||ST||=||STv||$.于是
    \[||ST||=||STv||=||Tv||\left|\left|S\left(\dfrac{Tv}{||Tv||}\right)\right|\right|\leqslant||Tv||||S||\leqslant||S||||T||\]
\end{proof}
\begin{problem}[6.]
    证明或给出一反例:如果$S,T\in\L(V)$,那么$||ST||=||TS||$.
\end{problem}
\begin{solution}
    令$V=\F^2$,$S(x,y)=(x,0),T(x,y)=(y,0)$.于是$ST=\mbf0\neq TS$.根据\tbf{7.87(b)},即$||ST||=0\neq||TS||$.
\end{solution}
\begin{problem}[8.]
    回答下列问题.
    \begin{enumerate}[label=\tbf{(\arabic*)}]
        \item 试证明:如果$T\in\L(V)$且$||I-T||<1$,那么$T$可逆.
        \item 设$S\in\L(V)$可逆.试证明:如果$T\in\L(V)$且$||S-T||<\dfrac{1}{||S^{-1}||}$,那么$T$是可逆的.
    \end{enumerate}
\end{problem}
\begin{proof}
    \begin{enumerate}[label=\tbf{(\arabic*)}]
        \item 如果$T$不可逆,那么存在$v\in V$使得$Tv=\mbf0$.令$u=\dfrac{v}{||v||}$,于是$||u||=1$.于是
            \[||I-T||\geqslant||(I-T)u||=||u-\mbf0||=||u||=1\]
            这与$||I-T||<1$矛盾,于是$T$可逆.
        \item 取与\tbf{(1)}同样的$v,u$,我们有$||S-T||\geqslant||(S-T)u|||=||Su||$.\\
            而$1=||u||=||S^{-1}Su||\leqslant||S^{-1}||||Su||$,即$||Su||\geqslant\dfrac{1}{||S^{-1}||}$.即$||S-T||\geqslant\dfrac{1}{||S^{-1}||}$.\\
            这与题设不符,于是$T$可逆.
    \end{enumerate}
\end{proof}
\begin{problem}[9.]
    设$T\in\L(V)$.试证明:对于任意$\ep>0$,都存在可逆的$S\in\L(V)$使得$0<||T-S||<\ep$.
\end{problem}
\begin{proof}
    取$\delta\in(0,\ep)$且$\delta$不是$T$的特征值,于是$T-\delta I$可逆.令$S=T-\delta I$,则有
    \[||T-S||=||\delta I||=|\delta|\in(0,\ep)\]
    于是命题得证.
\end{proof}
\begin{problem}[10.]
    设$\dim V>1$,$T\in\L(V)$不可逆.试证明:对于任意$\ep>0$,都存在$S\in\L(V)$使得$0<||T-S||<\ep$且$S$不可逆.
\end{problem}
\begin{proof}
    由于$T$不可逆,于是存在$e_1\in V$且$||e_1||=1$使得$Te_1=\mbf0$.将$e_1$扩展为$V$的规范正交基$\li e,n$.\\
    令$Se_1=\mbf0,Se_k=Te_k-\dfrac{\ep}{2}e_k(k=2,\cdots,n)$,于是$S$不可逆.对于任意$v\in V$且$v\notin\span(e_1)$有
    \[0<||(T-S)v||^2=\left|\left|\dfrac{\ep}{2}\sum_{k=2}^{n}\inprod{v}{e_k}e_k\right|\right|^2\leqslant\dfrac{\ep^2}{4}\]
    于是$0<||T-S||\leqslant\dfrac\ep2<\ep$
\end{proof}
\begin{problem}[11.]
    设$\F=\C$,$T\in\L(V)$.试证明:对于任意$\ep>0$,都存在可对角化的$S\in\L(V)$使得$0<||T-S||<\ep$.
\end{problem}
\begin{proof}
    根据Schur定理,存在$V$的规范正交基$\li e,n$使得$T$关于其有上三角矩阵$A$,其对角线元素为$\li\lambda,n$.\\
    考虑$D\in\L(V)$满足$De_k=ke_k$,显然$||D||=n$.\\
    由于$\li\lambda,n$的数目是有限的,于是存在$\delta\in\left(0,\dfrac{\ep}{n}\right)$使得$\lambda_k+k\delta$互异.\\
    令$S=T+\delta D$,于是
    \[0<||T-S||=||\delta D||=\delta||D||<\dfrac{\ep}{n}\cdot n=\ep\]
    于是命题得证.
\end{proof}
\begin{problem}[12.]
    设$T\in\L(V)$是正算子,试证明:$\left|\left|\sqrt{T}\right|\right|=\sqrt{\left|\left|T\right|\right|}$.
\end{problem}
\begin{proof}
    不妨设$T$的最大特征值为$\lambda$,于是$\sqrt{T}$的最大特征值为$\sqrt{\lambda}$.\\
    根据\tbf{7E.7}可知$T$和$\sqrt{T}$的最大奇异值分别为$\lambda$和$\sqrt{\lambda}$.于是命题得证.
\end{proof}
\begin{problem}[13.]
    设$S,T\in\L(V)$是正算子.试证明$||S-T||\leqslant\max\{||S||,||T||\}\leqslant||S+T||$.
\end{problem}
\begin{lemma}[Lemma.L.13]
    如果$A\in\L(V)$是自伴的,那么$||A||I-A$是正算子.
\end{lemma}
\begin{proof}
    考虑到$A,I$均为自伴算子,于是$||A||I-A$也是自伴的.\\
    考虑$||A||I-A$的特征值$\lambda$,则存在非零的$v\in V$使得$||A||v-Av=\lambda v$,从而$||A||-\lambda$是$A$的特征值.\\
    根据\tbf{7F.2}可得$\left|||A||-\lambda\right|\leqslant||A||$,从而$\lambda\geqslant0$,即$||A||I-A$是正算子.
\end{proof}
\begin{lemma}[Lemma.L.14]
    如果$A,B\in\L(V)$且$A,B-A$均为正算子,那么$||A||\leqslant||B||$.
\end{lemma}
\begin{proof}
    根据\tbf{7C.6}可知$B=A+(B-A)$是正算子,于是根据\tbf{Lemma.L.13}可知$||B||I-B$是正算子.于是
    \[||B||I-A=(||B||I-B)+(B-A)\]
    是正算子.考虑$A$的任意特征值$\lambda\geqslant0$和对应的特征向量$v$,我们有
    \[\left(||B||I-A\right)v=\left(||B||-\lambda\right)v\]
    由于$B-A$是正算子,于是$||B||-\lambda\geqslant0$,从而$||B||\geqslant\lambda$.\\
    于是$||A||=\max\{\lambda:\lambda\text{为}A\text{的特征值}\}\leqslant||B||$,命题得证.
\end{proof}
\begin{proof}
    回到我们的命题.设$\lambda$为自伴算子$S-T$的特征值,其特征向量为$v$.于是
    \[\left(||S||I-(S-T)\right)v=\left(||S||-\lambda\right)v\]
    根据\tbf{Lemma.L.13}和\tbf{7C.6}可知$||S||I-(S-T)$是正算子,于是$||S||-\lambda\geqslant0$.\\
    同理可知$||T||+\lambda\geqslant0$.由于上述两条等式对于所有$S-T$的特征值$\lambda$都成立,于是$||S-T||\leqslant\max\{||S||,||T||\}$.\\
    根据\tbf{Lemma.L.14},令$A=S,B=S+T$可知$||S||\leqslant||S+T||$,同理有$||T||\leqslant||S+T||$.\\
    于是$\max\{||S||,||T||\}\leqslant||S+T||$.
\end{proof}
\begin{problem}[14.]
    设$U,W$是$V$的子空间且$||P_U-P_W||<1$.试证明$\dim U=\dim W$.
\end{problem}
\begin{proof}
    注意到$P_U=I-P_{U^\bot},P_W=I-P_{W^\bot}$.根据\tbf{7F.8(1)}可知$P_U+P_{W^\bot}$和$P_W+P_{U^\bot}$都是可逆的.于是
    \[U\cap W^\bot=(\nul P_{U^{\bot}})\cap(\nul P_W)\subseteq\nul(P_{U^\bot}+P_W)=\{\mbf0\}\Rightarrow U\cap W^\bot=\{\mbf0\}\]
    \[V=\range(P_U+P_{W^\bot})\supseteq(\range P_U)+(\range P_{W^\bot})=U+W^\bot\Rightarrow U+W^\bot=V\]
    于是
    \[\dim V=\dim(U+W^\bot)=\dim U+\dim W^\bot-\dim(U\cap W^\bot)=\dim U+\dim V-\dim W\]
    于是$\dim U=\dim W$.
\end{proof}
\begin{problem}[15.]
    定义$T\in\L(\F^3)$为$T(z_1,z_2,z_3)=(z_3,2z_1,3z_2)$.求使得$T=S\sqrt{T^*T}$的$S\in\L(\F^3)$的表达式.
\end{problem}
\begin{solution}
    考虑$T$关于$\F^3$的标准基的矩阵
    \[\M(T)=\begin{pmatrix}
        0&0&1\\2&0&0\\0&3&0
    \end{pmatrix}\]
    于是
    \[\M(T^*T)=\begin{pmatrix}
        4&0&0\\0&9&0\\0&0&1
    \end{pmatrix}\]
    于是$\sqrt{T^*T}e_1=2e_1,\sqrt{T^*T}e_2=3e_2,\sqrt{T^*T}e_3=e_3$.于是$S$应满足
    \[Te_1=S\sqrt{T^*T}e_1=2Se_1=2e_2\Rightarrow Se_1=e_2\]
    \[Te_2=S\sqrt{T^*T}e_2=3Se_2=3e_3\Rightarrow Se_2=e_3\]
    \[Te_3=S\sqrt{T^*T}e_3=Se_3=e_1\Rightarrow Se_3=e_1\]
    于是
    \[S(z_1,z_2,z_3)=\left(z_3,z_1,z_2\right)\]
\end{solution}
\begin{problem}[16.]
    设$S\in\L(V)$是可逆正算子.试证明:存在$\delta>0$使得任意满足$||S-T||<\delta$的自伴算子$T$都是正算子.
\end{problem}
\begin{proof}
    设$\mu$是$S$的最小奇异值.由于$S$可逆,于是$\mu>0$.\\
    根据\tbf{7E.12(2)}可知$\sqrt{\mu}$是$\sqrt{T}$的最小奇异值,于是由\tbf{7E.14}可得
    \[\inprod{Sv}{v}=\inprod{\sqrt{S}v}{\sqrt{S}v}=\left|\left|\sqrt{S}v\right|\right|^2\geqslant\mu||v||^2\]
    设自伴算子$T\in\L(V)$满足$||S-T||<\mu$.由Cauchy不等式可得
    \[\inprod{(S-T)v}{v}\leqslant||(S-T)v||||v||\leqslant||S-T||||v||^2\leqslant\mu||v||^2\leqslant\inprod{Sv}{v}\]
    于是$\inprod{Tv}{v}\geqslant0$对任意$v\in V$都成立,因而$T$是正算子.
\end{proof}
\begin{problem}[17.]
    试证明:如果$u\in V$而$\varphi_u$是$V$上定义为$\varphi_u(v)=\inprod{v}{u}$的线性泛函,那么$||\varphi_u||=||u||$.
\end{problem}
\begin{proof}
    对于任意$v\in V$,由Cauchy不等式可得
    \[\left|\varphi_u(v)\right|=\left|\inprod{u}{v}\right|\leqslant||u||||v||\]
    当且仅当$v=\lambda u$时等号成立.对于所有满足$||v||=1$的$v\in V$则有
    \[\left|\varphi_u(v)\right|\leqslant||u||\]
    根据线性映射的范数的定义可知$||\varphi_u||=||u||$.
\end{proof}
\begin{problem}[18.]
    设$\li e,n$是$V$的规范正交基,$T\in\L(V,W)$.试证明下列命题.
    \begin{enumerate}[label=\tbf{(\arabic*)}]
        \item $\displaystyle\max_{1\leqslant k\leqslant n}\{||Te_k||\}\leqslant||T||\leqslant\sqrt{\sum_{k=1}^{n}||Te_k||^2}$.
        \item $\displaystyle||T||=\sqrt{\sum_{k=1}^{n}||Te_k||^2}$当且仅当$\dim\range T\leqslant1$.
    \end{enumerate}
\end{problem}
\begin{proof}
    \begin{enumerate}[label=\tbf{(\arabic*)}]
        \item 根据$||T||$的定义,对任意$v\in V$且$||v||=1$都有$||T||\geqslant||Tv||$,于是
            \[\max_{1\leqslant k\leqslant n}\{||Te_k||\}\leqslant||T||\]
            令$s_1,\cdots,s_n$为$T$的(按降序排列的)奇异值,根据\tbf{7E.11(1)}有
            \[\sum_{k=1}^{n}||Te_k||^2=\sum_{k=1}^{n}s_k^2\]
            由\tbf{7.88(a)}可知$||T||=s_1$,于是
            \[||T||=s_1=\sqrt{s_1^2}\leqslant\sqrt{\sum_{k=1}^{n}s_k^2}=\sqrt{\sum_{k=1}^{n}||Te_k||^2}\]
            于是命题得证.
        \item 这要求$T$除了其最大奇异值外其余所有奇异值为$0$.于是$\dim\range T\leqslant1$.
    \end{enumerate}
\end{proof}
\begin{problem}[19.]
    试证明:如果$T\in\L(V,W)$,那么$||T^*T||=||T||^2$.
\end{problem}
\begin{proof}
    考虑$||T||$的最大奇异值$s_1$,根据\tbf{7.88(a)}有$||T||=s_1$.\\
    而$T^*T$是正算子,其关于$V$的某规范正交基$\li e,n$由对角矩阵.不妨设对角线元素为$\li \lambda,n$.于是
    \[||T^*Tv||^2=\left|\left|\sum_{k=1}^{n}\lambda_k\inprod{v}{e_k}e_k\right|\right|
    =\sum_{k=1}^{n}\lambda_k^2\left|\inprod{v}{e_k}\right|^2
    \leqslant\left(\max_{1\leqslant k\leqslant n}\{\lambda_k\}\right)^2\sum_{k=1}^{n}\left|\inprod{v}{e_k}\right|
    \leqslant\left(\max_{1\leqslant k\leqslant n}\{\lambda_k\}\right)^2||v||^2\]
    于是$||T^*Tv||\leqslant\max_{1\leqslant k\leqslant n}\{\lambda_k\}||v||$,取$v=e_j$使得$\lambda_k$最大即可取等.\\
    因而$\displaystyle||T^*T||=\max_{1\leqslant k\leqslant n}\{\lambda_k\}=s_1^2=||T||^2$.
\end{proof}
\begin{proof}
    对于任意$v\in V$有
    \[\left|\left|\sqrt{T^*T}v\right|\right|^2=\inprod{T^*Tv}{v}=\inprod{Tv}{Tv}=||Tv||^2\leqslant||T||^2||v||^2\]
    于是$\left|\left|\sqrt{T^*T}\right|\right|\leqslant||T||$.考虑$T$的极分解$T=S\sqrt{T^*T}$,由于$S$是酉算子,于是$||S||=1$.于是根据\tbf{7F.5}有
    \[||T||=\left|\left|S\sqrt{T^*T}\right|\right|\leqslant||S||\left|\left|\sqrt{T^*T}\right|\right|=\left|\left|\sqrt{T^*T}\right|\right|\]
    于是$\left|\left|\sqrt{T^*T}\right|\right|=||T||$.根据\tbf{7F.12}有$||T^*T||=||T||^2$.
\end{proof}
\begin{problem}[20.]
    设$T\in\L(V)$是正规的,试证明:对于任意$k\in\N^*$都有$||T||^k=||T^k||$.
\end{problem}
\begin{proof}
    设$\li s,n$为$T$的奇异值,那么$||T||=s_1$.\\
    根据\tbf{7E.12(2)},考虑$V$的规范正交基$\li e,n$使得$T^*Te_j=s_j^2e_j(j=1,\cdots,n)$.于是根据$T$的正规性可得
    \[\left(T^k\right)^*T^ke_j=(T^*T)^ke_j=s_j^{2k}e_j\]
    从而$T^k$的奇异值为$s_1^k,\cdots,s_n^k$,于是$||T^k||=s_1^k=||T||^k$.
\end{proof}
\begin{problem}[21.]
    设$\dim V>1$且$\dim W>1$.试证明:$\L(V,W)$上的范数并不来自内积.换言之,试证明:不存在$\L(V,W)$上的内积使得
    \[\max\left\{||Tv||:v\in V\text{且}||v||=1\right\}=\sqrt{\inprod TT}\]
    对所有$T\in\L(V,W)$都成立.
\end{problem}
\begin{proof}
    考虑$V$的规范正交基$\li e,n$和$W$的规范正交基$\li f,m$,其中$m,n\geqslant 2$.\\
    令$T,S\in\L(V,W)$为
    \[Se_1=f_1,Se_2=f_2,Se_k=\mbf0(k>2)\]
    \[Te_1=-f_1,Te_2=f_2,Te_k=\mbf0(k>2)\]
    于是$||S+T||=||S-T||=2$且$||S||=||T||=1$,则有
    \[||S+T||^2+||S-T||^2=8\neq4=2\left(||S||^2+||T||^2\right)\]
    这不符合平行四边形等式,于是不存在这样的内积.
\end{proof}
\begin{problem}[22.]
    设$T\in\L(V,W)$.令$n=\dim V$,$\li s,n$为$T$的按降序排列的奇异值.试证明:如果$1\leqslant k\leqslant n$,%
    那么$\min\left\{||T|_U||:U\text{是}V\text{的子空间且}\dim U=k\right\}=s_{n-k+1}$.
\end{problem}\noindent
当你做到这里的时候,是否感到了绝望?
\begin{proof}
    对于给定的$k\in\{1,\cdots,n\}$,令$E_k=\min\left\{||T|_U||:U\text{是}V\text{的子空间且}\dim U=k\right\}$.\\
    如果$T$没有正奇异值,也即$T=\mbf0$,我们有$T|_U=\mbf0$,于是$E_k=0=s_{n-k+1}$.\\
    当$k=n$时,唯一可取的子空间为$V$,于是$E_n=||T||=s_1$.\\
    现在,假定$1\leqslant k<n$,并假设$\li s,m$为$T$的正奇异值.\\
    考虑$T$的奇异值分解
    \[Tv=\sum_{j=1}^{m}s_j\inprod{v}{e_j}f_j\]
    将$\li e,m$扩展为$V$的规范正交基$\li e,n$,于是
    \[Te_j=\left\{\begin{array}{l}
        s_jf_j,1\leqslant j\leqslant m\\
        \mbf0,m<j\leqslant n
    \end{array}\right.\]
    令$X=\span(e_{n-k+1},\cdots,e_k)$,那么$\dim X=k$.考虑以下两种情况.\\
    如果$1\leqslant k\leqslant n-m$,那么$s_{n-k+1}=0$.另外,$T|_X=\mbf0$,从而$E_k=||\mbf0||=0=s_{n-k+1}$.\\
    如果$n-m<k<n$,那么$1\leqslant n-k<m$.考虑$V$的任意一个$k$维子空间$U$.\\
    对于任意$v\in V$且$||v||\leqslant1$都有$P_Uv\in U$且$||P_Uv||\leqslant||v||\leqslant1$.\\
    于是根据$||T_U||$的定义可知$||TP_Uv||\leqslant||T|_U||$,即$||TP_U||\leqslant||T|_U||$.
    现在,注意到
    \[\dim\range(TP_{U^\bot})\leqslant\dim\range P_{U^\bot}=\dim U^\bot=n-k\]
    根据\tbf{7.92}可知$||T-TP_{U^\bot}||\geqslant s_{n-k+1}$.另一方面,有$P_U=I-P_{U^\bot}$,于是$||TP_U||\geqslant s_{n-k+1}$.\\
    于是$||T|_U||\geqslant||TP_U||\geqslant s_{n-k+1}$,我们得到了$||T|_U||$的下界.\\
    现在,对任意$x\in X$有
    \[||T|_Xx||^2=\sum_{j=n-k+1}^{m}s_j^2\left|\inprod{x}{e_j}\right|^2\leqslant s_{n-k+1}^2||x||^2\]
    从而$||T|_X||\leqslant s_{n-k+1}$.结合$T|_X$的下界可知$||T|_X||=s_{n-k+1}$,于是$E_k=s_{n-k+1}$.\\
    综上可知命题得证.
\end{proof}
\begin{problem}[24.]
    设$T\in\L(V)$可逆.试证明:$||T^{-1}||=\dfrac{1}{||T||}$当且仅当$\dfrac{T}{||T||}$是酉算子.
\end{problem}
\begin{proof}
    设$T$的奇异值为$\li s,n$.由于$T$可逆,于是$\li s,n>0$.\\
    根据伪逆的奇异值分解和伪逆的性质可知$T^{-1}$的奇异值(按降序排列)为$\dfrac{1}{s_n},\cdots,\dfrac{1}{s_1}$.\\
    于是$||T^{-1}||=\dfrac{1}{s_n}$.于是
    \[||T^{-1}||=\dfrac{1}{||T||}\Leftrightarrow\dfrac{1}{s_n}=\dfrac{1}{s_1}\Leftrightarrow\li s=n\Leftrightarrow\dfrac{T}{||T||}\text{的所有奇异值均为}1\Leftrightarrow \dfrac{T}{||T||}\text{是酉算子}\]
\end{proof}
\begin{problem}[25.]
    取定$u,x\in V$,其中$u\neq\mbf0$.定义$T\in\L(V)$为对任意$v\in V$有$Tv=\inprod{v}{u}x$.%
    试证明:\[\sqrt{T^*T}v=\dfrac{||x||}{||u||}\inprod{v}{u}u\]对任意$v\in V$成立.
\end{problem}
\begin{proof}
    我们有
    \[\inprod{Tv}{v}=\inprod{v}{u}\cdot\inprod{x}{v}=\inprod{v}{\inprod{v}{x}u}=\inprod{v}{T^*v}\]
    于是$T^*v=\inprod{v}{x}u$.令$R=\dfrac{||x||}{||u||}\inprod{v}{u}u$,我们有
    \[T^*Tv=\inprod{v}{u}T^*x=||x||^2\inprod{v}{u}u\]
    而
    \[R^2v=\dfrac{||x||^2}{||u||^2}\inprod{v}{u}||u||^2u=||x||^2\inprod vuu=T^*Tv\]
    于是$R^2=T^*T$.对于任意$v\in V$有
    \[\inprod{Rv}{v}=\dfrac{||x||}{||u||}\inprod{v}{u}\inprod{u}{v}=\dfrac{||x||}{||u||}\left|\inprod vu\right|^2\geqslant0\]
    于是$R$是正算子,因而$R=\sqrt{T^*T}$.
\end{proof}
\begin{problem}[26.]
    设$T\in\L(V)$.试证明:$T$可逆,当且仅当存在唯一的酉算子$S\in\L(V)$使得$T=S\sqrt{T^*T}$.
\end{problem}
\begin{proof}
    $\Rightarrow$:设$T$可逆,那么根据$T$的极分解可知存在酉算子$S$使得$T=S\sqrt{T^*T}$.\\
    又因为$T$可逆,于是$\nul\sqrt{T^*T}=\nul T^*T=\nul T=\{\mbf0\}$,进而$\sqrt{T^*T}$可逆.于是
    \[S=T\left(\sqrt{T^*T}\right)^{-1}\]
    是唯一的.\\
    $\Leftarrow$:如果$T$不可逆,那么考虑$T$的奇异值分解
    \[Tv=\sum_{k=1}^{m}s_k\inprod{v}{e_k}f_k\]
    将$\li e,m$和$\li f,m$扩展为$V$的规范正交基$\li e,n$和$\li f,n$.\\
    由于$T$不可逆,那么其存在为$0$的奇异值,则有$n>m$.\\
    令$S,R\in\L(V)$,满足$Se_k=Re_k=f_k(k=1,\cdots,n-1)$和$Se_n=-Re_n=f_n$.\\
    显然$R\neq S$,并且它们都是酉算子.而
    \[S\sqrt{T^*T}v=S\left(\sum_{k=1}^{m}s_k\inprod{v}{e_k}e_k\right)=Tv\]
    \[R\sqrt{T^*T}v=R\left(\sum_{k=1}^{m}s_k\inprod{v}{e_k}e_k\right)=Tv\]
    于是存在互异的$S,R$使得$T=S\sqrt{T^*T}=R\sqrt{T^*T}$,这与假设矛盾,于是$T$可逆.
\end{proof}
\begin{problem}[27.]
    设$T\in\L(V)$,$T$的奇异值为$\li s,n$.令$\li e,n$和$\li f,n$为$V$的规范正交基,使得
    \[Tv=\sum_{k=1}^{n}s_k\inprod{v}{e_k}f_k\]
    对任意$v\in V$成立.定义$S\in\L(V)$为
    \[Sv=\sum_{k=1}^n\inprod{v}{e_k}f_k\]
    试证明下列命题.
    \begin{enumerate}[label=\tbf{(\arabic*)}]
        \item $S$是酉算子且$\displaystyle||T-S||=\max_{k=1,\cdots,n}\{|s_k-1|\}$.
        \item 如果$E\in\L(V)$是酉算子,那么$||T-E||\geqslant||T-S||$.
    \end{enumerate}
\end{problem}
\begin{proof}
    \begin{enumerate}[label=\tbf{(\arabic*)}]
        \item 注意到$Se_k=f_k$对任意$k=1,\cdots,n$都成立,于是$S$是酉算子.我们有
            \[(T-S)v=\sum_{k=1}^{n}\left(s_k-1\right)\inprod{v}{e_k}f_k=\sum_{k=1}^{n}\left|s_k-1\right|\inprod{v}{e_k}\dfrac{f_k}{\text{sgn}\left(s_k-1\right)}\]
            上式即为$T-S$的奇异值分解.于是$T-S$的奇异值为$\left|s_1-1\right|,\cdots,\left|s_n-1\right|$.\\
            于是$||T-S||=\displaystyle\max_{k=1,\cdots,n}\{|s_k-1|\}$.
        \item 对于任意$k=1,\cdots,n$都有$||Ee_k||=||e_k||=1$,于是
            \[||T-E||\geqslant||(T-E)e_k||\geqslant\left|||Te_k||-||Ee_k||\right|\geqslant\left|||s_kf_k||-1\right|=\left|s_k-1\right|\]
            于是$||T-E||\geqslant\displaystyle\max_{k=1,\cdots,n}\{|s_k-1|\}=||T-S||$.
    \end{enumerate}
\end{proof}
\begin{problem}[28.]
    设$T\in\L(V)$.试证明:存在酉算子$S\in\L(V)$使得$T=\sqrt{TT^*}S$.
\end{problem}
\begin{proof}
    考虑$T$的奇异值分解
    \[Tv=\sum_{k=1}^{m}s_k\inprod{v}{e_k}f_k\]
    根据$T^*$的奇异值分解,代入可得
    \[\sqrt{TT^*}v=\sum_{k=1}^{m}s_k\inprod{v}{f_k}f_k\]
    将$\li e,m$和$\li f,m$扩展为$V$的规范正交基$\li e,n$和$\li f,n$.\\
    定义$S\in\L(V)$为
    \[Sv=\sum_{k=1}^{n}\inprod{v}{e_k}f_k\]
    则有$\inprod{Sv}{f_k}=\inprod{v}{e_k}$对于任意$k=1,\cdots,m$成立.于是
    \[\sqrt{TT^*}Sv=\sum_{k=1}^{m}s_k\inprod{Sv}{f_k}f_k=\sum_{k=1}^{m}s_k\inprod{v}{f_k}f_k=Tv\]
    于是$T=\sqrt{TT^*}S$.
\end{proof}
\begin{problem}[29.]
    设$T\in\L(V)$.试证明下列命题.
    \begin{enumerate}[label=\tbf{(\arabic*)}]
        \item 存在酉算子$S\in\L(V)$使得$TT^*=ST^*TS^*$.
        \item 上一问的结论蕴含$T$和$T^*$的奇异值相同.
    \end{enumerate}
\end{problem}
\begin{proof}
    \begin{enumerate}[label=\tbf{(\arabic*)}]
        \item 根据\tbf{7.93}和\tbf{7F.28}可知存在酉算子$S$使得$T=S\sqrt{T^*T}=\sqrt{TT^*}S$.\\
            由于$S$是酉算子,于是$S$可逆,并且$S=S^*$.在上式右乘$S^*$可得
            \[\sqrt{TT^*}=S\sqrt{T^*T}S^*\]
            于是
            \[TT^*=S\sqrt{T^*T}S^*S\sqrt{T^*T}S^*=S\sqrt{T^*T}\sqrt{T^*T}S^*=ST^*TS^*\]
        \item 由\tbf{5A.13}可得$T^*T$和$TT^*$的特征值相同,于是$T$和$T^*$的奇异值相同.
    \end{enumerate}
\end{proof}
\begin{problem}[30.]
    设$T\in\L(V)$,酉算子$S\in\L(V)$,正算子$R\in\L(V)$使得$T=SR$.试证明:$R=\sqrt{T^*T}$.
\end{problem}
\begin{proof}
    注意到
    \[T^*T=(SR)^*(SR)=R^*S^*SR=R^*R=R^2\]
    由于$R$是正算子,于是$R=\sqrt{T^*T}$.
\end{proof}
\begin{problem}[31.]
    设$\F=\C$且$T\in\L(V)$是正规的.试证明:存在酉算子$S\in\L(V)$使得$T=S\sqrt{T^*T}$,并且$S$和$\sqrt{T^*T}$关于$V$的同一个规范正交基有对角矩阵.
\end{problem}
\begin{proof}
    根据\tbf{7.93}和\tbf{7F.28}可知存在酉算子$S$使得$T=S\sqrt{T^*T}=\sqrt{TT^*}S$.\\
    由于$T$正规,于是$T^*T=TT^*$,因此$S\sqrt{T^*T}=\sqrt{T^*T}S$,即$S$和$\sqrt{T^*T}$可交换.\\
    根据\tbf{7B.16}可知$S$和$\sqrt{T^*T}$关于$V$的同一个规范正交基有对角矩阵.
\end{proof}
\begin{problem}[32.]
    设$T\in\L(V,W)$且$T\neq\mbf0$.令$\li s,m$为$T$的正奇异值,试证明:存在$(\nul T)^\bot$的规范正交基$\li e,m$使得
    \[T\left(E\left(\dfrac{e_1}{s_1},\cdots,\dfrac{e_m}{s_m}\right)\right)\]
    为$\range T$中以$\mbf0$为中心,半径为$1$的球.
\end{problem}
\begin{proof}
    考虑$T$的奇异值分解
    \[T=\sum_{k=1}^ms_k\inprod{v}{e_k}f_k\]
    其中$\li e,m$和$\li f,m$分别为$V$和$W$中的规范正交组.\\
    又因为$Tv\neq\mbf0$当且仅当$v\in\span(\li e,m)$,于是$\li e,m$是$(\nul T)^\bot$的规范正交基.\\
    对于任意$v\in E\left(\dfrac{e_1}{s_1},\cdots,\dfrac{e_m}{s_m}\right)$有
    \[s_1^2\left|\inprod{v}{e_1}\right|^2+\cdots+s_m^2\left|\inprod{v}{e_m}\right|^2<1
    \Leftrightarrow \left|\inprod{Tv}{f_1}\right|^2+\cdots+\left|\inprod{Tv}{f_m}\right|^2<1
    \Leftrightarrow ||Tv||<1\]
    由于$\li f,m$为$\range T$的基,于是
    \[T\left(E\left(\dfrac{e_1}{s_1},\cdots,\dfrac{e_m}{s_m}\right)\right)\]
    为$\range T$中的单位球.
\end{proof}
\end{document}