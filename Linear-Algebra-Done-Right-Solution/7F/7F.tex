\documentclass{ctexart}
\usepackage{geometry}
\usepackage[dvipsnames,svgnames]{xcolor}
\usepackage[strict]{changepage}
\usepackage{framed}
\usepackage{enumerate}
\usepackage{amsmath,amsthm,amssymb}
\usepackage{enumitem}
\usepackage{solution}

\allowdisplaybreaks
\geometry{left=2cm, right=2cm, top=2.5cm, bottom=2.5cm}

\begin{document}\pagestyle{empty}
\begin{center}
    \large\tbf{Linear Algebra Done Right 7F}
\end{center}
\begin{problem}[1.]
    设$S,T\in\L(V,W)$.试证明:$\left|||S||-||T||\right|\leqslant||S-T||$.
\end{problem}
\begin{proof}
    由线性映射的范数的定义可知存在$v\in V$且$||v||\leqslant1$使得$||S-T||=||(S-T)v||$.于是
    \[||S-T||=||(S-T)v||=||Sv-Tv||\geqslant\left|||Sv||-||Tv||\right|\geqslant\left|||S||-||T||\right|\]
    以上不等式即\tbf{反向三角不等式}.
\end{proof}
\begin{problem}[2.]
    设$T\in\L(V)$是自伴的(如果$\F=\C$则令$T$是正规的),试证明:$||T||=\max\{|\lambda|:\lambda\text{是}T\text{的特征值}\}$.
\end{problem}
\begin{proof}
    由\tbf{7.88}和\tbf{7E.7}可知
    \[||T||=T\text{的最大奇异值}=T\text{的绝对值最大的特征值}\]
\end{proof}
\begin{problem}[3.]
    设$T\in\L(V,W)$且$v\in V$.试证明:$||Tv||=||T||||v||$当且仅当$T^*Tv=||T||^2v$.
\end{problem}
\begin{proof}
    $\Leftarrow$:由\tbf{7.88(c)}和\tbf{7.91}可得
    \[||T||^2||v||=||T^*Tv||\leqslant||T^*||||Tv||=||T||||Tv||\Rightarrow||Tv||\geqslant||T||||v||\]
    而$||Tv||\leqslant||T||||v||$,于是$||Tv||=||T||||v||$.\\
    $\Rightarrow$:我们有
    \[\begin{aligned}
        ||T^*Tv-||T||^2v||^2
        &= \inprod{T^*Tv-||T||^2v}{T^*Tv-||T||^2v} \\
        &= ||T^*Tv||^2+||T||^4||v||^2-2\text{Re}\inprod{T^*Tv}{||T||^2v} \\
        &\leqslant ||T^*||^2||Tv||^2+||T||^4||v||^2-2||T||^2||Tv||^2 \\
        &= 0
    \end{aligned}\]
    于是$T^*Tv=||T||^2v$.
\end{proof}
\begin{problem}[4.]
    设$T\in\L(V,W),v\in V$且$||Tv||=||T||||v||$.试证明:如果$u\in V$且$\inprod uv=0$,那么$\inprod{Tu}{Tv}=0$.
\end{problem}
\begin{proof}
    根据\tbf{7F.3}有
    \[\inprod{Tu}{Tv}=\inprod{u}{T^*Tv}=\inprod{u}{||T||^2v}=||T||^2\inprod uv=0\]
\end{proof}
\begin{problem}[5.]
    设$U$是有限维内积空间,$T\in\L(V,U)$且$S\in\L(U,W)$.试证明:$||ST||\leqslant||S||||T||$.
\end{problem}
\begin{proof}
    由线性映射的范数的定义可知存在$v\in V$且$||v||\leqslant1$使得$||ST||=||STv||$.于是
    \[||ST||=||STv||=||Tv||\left|\left|S\left(\dfrac{Tv}{||Tv||}\right)\right|\right|\leqslant||Tv||||S||\leqslant||S||||T||\]
\end{proof}
\begin{problem}[6.]
    证明或给出一反例:如果$S,T\in\L(V)$,那么$||ST||=||TS||$.
\end{problem}
\begin{solution}
    令$V=\F^2$,$S(x,y)=(x,0),T(x,y)=(y,0)$.于是$ST=\mbf0\neq TS$.根据\tbf{7.87(b)},即$||ST||=0\neq||TS||$.
\end{solution}
\begin{problem}[8.]
    回答下列问题.
    \begin{enumerate}[label=\tbf{(\arabic*)}]
        \item 试证明:如果$T\in\L(V)$且$||I-T||<1$,那么$T$可逆.
        \item 设$S\in\L(V)$可逆.试证明:如果$T\in\L(V)$且$||S-T||<\dfrac{1}{||S^{-1}||}$,那么$T$是可逆的.
    \end{enumerate}
\end{problem}
\begin{proof}
    \begin{enumerate}[label=\tbf{(\arabic*)}]
        \item 如果$T$不可逆,那么存在$v\in V$使得$Tv=\mbf0$.令$u=\dfrac{v}{||v||}$,于是$||u||=1$.于是
            \[||I-T||\geqslant||(I-T)u||=||u-\mbf0||=||u||=1\]
            这与$||I-T||<1$矛盾,于是$T$可逆.
        \item 取与\tbf{(1)}同样的$v,u$,我们有$||S-T||\geqslant||(S-T)u|||=||Su||$.\\
            而$1=||u||=||S^{-1}Su||\leqslant||S^{-1}||||Su||$,即$||Su||\geqslant\dfrac{1}{||S^{-1}||}$.即$||S-T||\geqslant\dfrac{1}{||S^{-1}||}$.\\
            这与题设不符,于是$T$可逆.
    \end{enumerate}
\end{proof}
\begin{problem}[9.]
    设$T\in\L(V)$.试证明:对于任意$\ep>0$,都存在可逆的$S\in\L(V)$使得$0<||T-S||<\ep$.
\end{problem}
\begin{proof}
    取$\delta\in(0,\ep)$且$\delta$不是$T$的特征值,于是$T-\delta I$可逆.令$S=T-\delta I$,则有
    \[||T-S||=||\delta I||=|\delta|\in(0,\ep)\]
    于是命题得证.
\end{proof}
\begin{problem}[10.]
    设$\dim V>1$,$T\in\L(V)$不可逆.试证明:对于任意$\ep>0$,都存在$S\in\L(V)$使得$0<||T-S||<\ep$且$S$不可逆.
\end{problem}
\begin{proof}
    由于$T$不可逆,于是存在$e_1\in V$且$||e_1||=1$使得$Te_1=\mbf0$.将$e_1$扩展为$V$的规范正交基$\li e,n$.\\
    令$Se_1=\mbf0,Se_k=Te_k-\dfrac{\ep}{2}e_k(k=2,\cdots,n)$,于是$S$不可逆.对于任意$v\in V$且$v\notin\span(e_1)$有
    \[0<||(T-S)v||^2=\left|\left|\dfrac{\ep}{2}\sum_{k=2}^{n}\inprod{v}{e_k}e_k\right|\right|^2\leqslant\dfrac{\ep^2}{4}\]
    于是$0<||T-S||\leqslant\dfrac\ep2<\ep$
\end{proof}
\begin{problem}[11.]
    设$\F=\C$,$T\in\L(V)$.试证明:对于任意$\ep>0$,都存在可对角化的$S\in\L(V)$使得$0<||T-S||<\ep$.
\end{problem}
\begin{proof}
    根据Schur定理,存在$V$的规范正交基$\li e,n$使得$T$关于其有上三角矩阵$A$,其对角线元素为$\li\lambda,n$.\\
    考虑$D\in\L(V)$满足$De_k=ke_k$,显然$||D||=n$.\\
    由于$\li\lambda,n$的数目是有限的,于是存在$\delta\in\left(0,\dfrac{\ep}{n}\right)$使得$\lambda_k+k\delta$互异.\\
    令$S=T+\delta D$,于是
    \[0<||T-S||=||\delta D||=\delta||D||<\dfrac{\ep}{n}\cdot n=\ep\]
    于是命题得证.
\end{proof}
\begin{problem}[12.]
    设$T\in\L(V)$是正算子,试证明:$\left|\left|\sqrt{T}\right|\right|=\sqrt{\left|\left|T\right|\right|}$.
\end{problem}
\begin{proof}
    不妨设$T$的最大特征值为$\lambda$,于是$\sqrt{T}$的最大特征值为$\sqrt{\lambda}$.\\
    根据\tbf{7E.7}可知$T$和$\sqrt{T}$的最大奇异值分别为$\lambda$和$\sqrt{\lambda}$.于是命题得证.
\end{proof}
\begin{problem}[13.]
    设$S,T\in\L(V)$是正算子.试证明$||S-T||\leqslant\max\{||S||,||T||\}\leqslant||S+T||$.
\end{problem}
\begin{lemma}[Lemma.L.13]
    如果$A\in\L(V)$是自伴的,那么$||A||I-A$是正算子.
\end{lemma}
\begin{proof}
    考虑到$A,I$均为自伴算子,于是$||A||I-A$也是自伴的.\\
    考虑$||A||I-A$的特征值$\lambda$,则存在非零的$v\in V$使得$||A||v-Av=\lambda v$,从而$||A||-\lambda$是$A$的特征值.\\
    根据\tbf{7F.2}可得$\left|||A||-\lambda\right|\leqslant||A||$,从而$\lambda\geqslant0$,即$||A||I-A$是正算子.
\end{proof}
\begin{lemma}[Lemma.L.14]
    如果$A,B\in\L(V)$且$A,B-A$均为正算子,那么$||A||\leqslant||B||$.
\end{lemma}
\begin{proof}
    根据\tbf{7C.6}可知$B=A+(B-A)$是正算子,于是根据\tbf{Lemma.L.13}可知$||B||I-B$是正算子.于是
    \[||B||I-A=(||B||I-B)+(B-A)\]
    是正算子.考虑$A$的任意特征值$\lambda\geqslant0$和对应的特征向量$v$,我们有
    \[\left(||B||I-A\right)v=\left(||B||-\lambda\right)v\]
    由于$B-A$是正算子,于是$||B||-\lambda\geqslant0$,从而$||B||\geqslant\lambda$.\\
    于是$||A||=\max\{\lambda:\lambda\text{为}A\text{的特征值}\}\leqslant||B||$,命题得证.
\end{proof}
\begin{proof}
    回到我们的命题.设$\lambda$为自伴算子$S-T$的特征值,其特征向量为$v$.于是
    \[\left(||S||I-(S-T)\right)v=\left(||S||-\lambda\right)v\]
    根据\tbf{Lemma.L.13}和\tbf{7C.6}可知$||S||I-(S-T)$是正算子,于是$||S||-\lambda\geqslant0$.\\
    同理可知$||T||+\lambda\geqslant0$.由于上述两条等式对于所有$S-T$的特征值$\lambda$都成立,于是$||S-T||\leqslant\max\{||S||,||T||\}$.\\
    根据\tbf{Lemma.L.14},令$A=S,B=S+T$可知$||S||\leqslant||S+T||$,同理有$||T||\leqslant||S+T||$.\\
    于是$\max\{||S||,||T||\}\leqslant||S+T||$.
\end{proof}
\begin{problem}[14.]
    设$U,W$是$V$的子空间且$||P_U-P_W||<1$.试证明$\dim U=\dim W$.
\end{problem}
\begin{proof}
    注意到$P_U=I-P_{U^\bot},P_W=I-P_{W^\bot}$.根据\tbf{7F.8(1)}可知$P_U+P_{W^\bot}$和$P_W+P_{U^\bot}$都是可逆的.于是
    \[U\cap W^\bot=(\nul P_{U^{\bot}})\cap(\nul P_W)\subseteq\nul(P_{U^\bot}+P_W)=\{\mbf0\}\Rightarrow U\cap W^\bot=\{\mbf0\}\]
    \[V=\range(P_U+P_{W^\bot})\supseteq(\range P_U)+(\range P_{W^\bot})=U+W^\bot\Rightarrow U+W^\bot=V\]
    于是
    \[\dim V=\dim(U+W^\bot)=\dim U+\dim W^\bot-\dim(U\cap W^\bot)=\dim U+\dim V-\dim W\]
    于是$\dim U=\dim W$.
\end{proof}
\end{document}