\documentclass{ctexart}
\usepackage{geometry}
\usepackage[dvipsnames,svgnames]{xcolor}
\usepackage[strict]{changepage}
\usepackage{framed}
\usepackage{enumerate}
\usepackage{amsmath,amsthm,amssymb}
\usepackage{enumitem}
\usepackage{solution}

\allowdisplaybreaks
\geometry{left=2cm, right=2cm, top=2.5cm, bottom=2.5cm}

\begin{document}\pagestyle{empty}
\begin{problem}[1.]
    设$T\in\mathcal{L}(V,W)$可逆,证明$T^{-1}$可逆,且有$\left(T^{-1}\right)^{-1}=T$.
\end{problem}
\begin{proof}
    假设$v_1,v_2\in V$且$Tv_1=w_1,Tv_2=w_2$.于是$T^{-1}w_1=v_1,T^{-1}w_2=v_2$.\\
    从而$T^{-1}w_1=T^{-1}w_2$必有$v_1=v_2$,于是必有$w_1=Tv_1=Tv_2=w_2$,即$T^{-1}$是单射.\\
    对于任意$v\in V$都有$T^{-1}(Tv)=v$,于是$\range T^{-1}=V$,进而$T^{-1}$是满射.\\
    综上,$T^{-1}$可逆.下面证明$\left(T^{-1}\right)^{-1}=T$.注意到$T^{-1}T=I,TT^{-1}=I$.\\
    于是根据定义可知$T$是$T^{-1}$的逆,进而$T=\left(T^{-1}\right)^{-1}$.\\
    命题得证.
\end{proof}
\end{document}